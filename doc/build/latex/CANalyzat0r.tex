%% Generated by Sphinx.
\def\sphinxdocclass{report}
\documentclass[letterpaper,10pt,english]{sphinxmanual}
\ifdefined\pdfpxdimen
   \let\sphinxpxdimen\pdfpxdimen\else\newdimen\sphinxpxdimen
\fi \sphinxpxdimen=.75bp\relax

\usepackage[utf8]{inputenc}
\ifdefined\DeclareUnicodeCharacter
 \ifdefined\DeclareUnicodeCharacterAsOptional\else
  \DeclareUnicodeCharacter{00A0}{\nobreakspace}
\fi\fi
\usepackage{cmap}
\usepackage[T1]{fontenc}
\usepackage{amsmath,amssymb,amstext}
\usepackage{babel}
\usepackage{times}
\usepackage[Bjarne]{fncychap}
\usepackage[dontkeepoldnames]{sphinx}

\usepackage{geometry}

% Include hyperref last.
\usepackage{hyperref}
% Fix anchor placement for figures with captions.
\usepackage{hypcap}% it must be loaded after hyperref.
% Set up styles of URL: it should be placed after hyperref.
\urlstyle{same}
\addto\captionsenglish{\renewcommand{\contentsname}{Contents:}}

\addto\captionsenglish{\renewcommand{\figurename}{Fig.}}
\addto\captionsenglish{\renewcommand{\tablename}{Table}}
\addto\captionsenglish{\renewcommand{\literalblockname}{Listing}}

\addto\extrasenglish{\def\pageautorefname{page}}

\setcounter{tocdepth}{3}
\setcounter{secnumdepth}{3}


\title{CANalyzat0r Documentation}
\date{Jul 21, 2017}
\release{1.0}
\author{pschmied (SCHUTZWERK GmbH)}
\newcommand{\sphinxlogo}{\vbox{}}
\renewcommand{\releasename}{Release}
\makeindex

\begin{document}

\maketitle
\sphinxtableofcontents
\phantomsection\label{\detokenize{index::doc}}



\chapter{Requirements}
\label{\detokenize{requirements:requirements}}\label{\detokenize{requirements::doc}}\label{\detokenize{requirements:welcome-to-canalyzat0r-s-documentation}}\begin{description}
\item[{To install all requirements, please execute the following commands:}] \leavevmode\begin{itemize}
\item {} 
sudo apt-get install python3.5 python3-pip python3-pyside can-utils ffmpeg

\item {} 
sudo pip3 install pyvit sphinx\_rtd\_theme

\end{itemize}

\end{description}

Note: You can just execute \sphinxcode{install.requirements.sh}.


\section{Docker}
\label{\detokenize{requirements:docker}}
There’s also a docker container available. Check the docker folder.


\chapter{CANAlyzat0r manual}
\label{\detokenize{manual:canalyzat0r-manual}}\label{\detokenize{manual::doc}}

\section{Introduction}
\label{\detokenize{manual:introduction}}
You can use \sphinxstylestrong{CANAlyzat0r} to quickly analyze the CAN bus in many ways.
It’s great.

\noindent\sphinxincludegraphics{{icon}.png}

This documentation will guide you through the usage of the application.
Also, you can find the code documentation in this document if you want
to extend and/or contribute to this project.


\section{Usage: Tab by tab}
\label{\detokenize{manual:usage-tab-by-tab}}

\subsection{Main Tab}
\label{\detokenize{manual.maintab::doc}}\label{\detokenize{manual.maintab:main-tab}}
Welcome to CANalyzat0rs main tab!
Here you can change interface settings and creat/remove virtual CAN
interfaces. Don’t worry, the kernel modules should aready be loaded for you.


\subsubsection{Where’s my interface?!?!1!}
\label{\detokenize{manual.maintab:where-s-my-interface-1}}
If you can’t find your attached CAN interface in the ComboBox, please
check the output of \sphinxcode{ifconfig -a}. In order to use your interface
with CANAlyzat0r, a SocketCAN device must be present. Maybe you have to
load another kernel module/driver?


\subsubsection{Creating and selecting projects}
\label{\detokenize{manual.maintab:creating-and-selecting-projects}}
On a fresh startup, you should encounter a message saying that a new
project should be created. You can still use this application without a
selected project. However, one can’t save dumps or known packets.
To create a project, please refer to the manager tab. After you
have created a project there, you can set it as active project in the
main tab.


\subsubsection{Log levels}
\label{\detokenize{manual.maintab:log-levels}}
You can set the minimum log level for which messages will be printed
to the log box in this tab.


\subsubsection{Where’s my data being saved to?!!?}
\label{\detokenize{manual.maintab:where-s-my-data-being-saved-to}}
By default, CANalyzat0r creates a SQLite database called “database.db”
in the data folder. Please take care of this file as everything you
discover is saved here.


\subsubsection{But what if i want to export my data?}
\label{\detokenize{manual.maintab:but-what-if-i-want-to-export-my-data}}
Please check the manager tab and learn on how to export projects and
dumps.


\subsection{General}
\label{\detokenize{manual.general::doc}}\label{\detokenize{manual.general:general}}

\subsubsection{Why can I change interface settings in every tab and why is there a global interface?}
\label{\detokenize{manual.general:why-can-i-change-interface-settings-in-every-tab-and-why-is-there-a-global-interface}}
You can set a global interface in order to set the selected interface
for every inactive tab. On top of that, you can override this setting
for every tab individually using the button. This allows you to e.g.
fuzz on \sphinxcode{can0} and sniff on \sphinxcode{vcan1} at the same time.


\subsubsection{Well, how can I manage packets in between tabs?}
\label{\detokenize{manual.general:well-how-can-i-manage-packets-in-between-tabs}}\begin{description}
\item[{You can:}] \leavevmode\begin{itemize}
\item {} 
Select rows and copy them to another tab (if allowed)

\item {} 
Delete rows by selecting rows and pressing \sphinxcode{Del} on your keyboard

\end{itemize}

\end{description}


\subsubsection{Ok Ok, but how can I import my SocketCAN dumps?}
\label{\detokenize{manual.general:ok-ok-but-how-can-i-import-my-socketcan-dumps}}
Just copy and paste them into the GUI tables \textless{}:


\subsubsection{What are known packets?}
\label{\detokenize{manual.general:what-are-known-packets}}
Once you discovered that packet XY does Action ZZ on your car or
setup, you can add this knowledge to the database using the manager tab.
This adds the discovered information globally for a specific project.
Using this, the “Description” column in the GUI tables in filled with
data, so you can recognize a re-occuring packet.


\subsubsection{I’ve discovered a bug, pls fix!}
\label{\detokenize{manual.general:i-ve-discovered-a-bug-pls-fix}}
Pleae report bugs using GitHub issues, Thanks.


\subsection{Sniffer Tab}
\label{\detokenize{manual.sniffertab:sniffer-tab}}\label{\detokenize{manual.sniffertab::doc}}

\subsubsection{How to sniff?}
\label{\detokenize{manual.sniffertab:how-to-sniff}}
It’s simple. For every discoverd interface you can find a sniffer tab
here. If no tab for your desired interface is displayed, please
re-check all available interfaces using the button on the main tab.
Once you find your desired interface, just click on start.


\subsubsection{Ignoring packets}
\label{\detokenize{manual.sniffertab:ignoring-packets}}
You can add tab specific packets to the ignore list.
Hint: Use a blank data field if you want to exclude whole IDs.

Another Hint: You can use the fitering buttons to remove background noise. It’s great.


\subsubsection{I get errors/no data when I try to sniff!!1!}
\label{\detokenize{manual.sniffertab:i-get-errors-no-data-when-i-try-to-sniff-1}}\begin{itemize}
\item {} 
Maybe your SocketCAN interface disappeared?

\item {} 
Maybe you have selected a wrong bitrate?

\item {} 
Have you tried turning it off and on again?

\end{itemize}


\subsubsection{The sniffer tab doesn’t list the packets I’m sending}
\label{\detokenize{manual.sniffertab:the-sniffer-tab-doesn-t-list-the-packets-i-m-sending}}
Thats normal. You will only see packets that you are receiving on a
specific interface. This prevents the packets that you generate via
fuzzing from being in your sniffed dump. If you really need all
packets in one dump, you can use \sphinxcode{candump}.


\subsection{Sender Tab}
\label{\detokenize{manual.sendertab:sender-tab}}\label{\detokenize{manual.sendertab::doc}}

\subsubsection{What do the sub tabs do?}
\label{\detokenize{manual.sendertab:what-do-the-sub-tabs-do}}
You can create multiple sender tabs to handle sending various packet
sets.


\subsubsection{Why does \sphinxstyleliteralintitle{Loop} do?!?!}
\label{\detokenize{manual.sendertab:why-does-loop-do}}
One can either send packets once or in a loop with a specific packet gap.
Imagine you have to send a packet every X seconds to keep a CAN device
online: It’s handy.


\subsection{Fuzzer Tab}
\label{\detokenize{manual.fuzzertab:fuzzer-tab}}\label{\detokenize{manual.fuzzertab::doc}}

\subsubsection{What does this thing to?}
\label{\detokenize{manual.fuzzertab:what-does-this-thing-to}}
Using this tab you can send random packets into the CAN bus to discover
things. You can tune settings that control random packet generation
using the GUI elements.


\subsubsection{What are masks?}
\label{\detokenize{manual.fuzzertab:what-are-masks}}
You can write static values into the masks or put an X if that character
should be randomized. Using this, you can freely control the payload
of generated packets.
Hint: You can change masks and lengths \sphinxstylestrong{while fuzzing}.


\subsubsection{Other fuzzers are much faster!!1!}
\label{\detokenize{manual.fuzzertab:other-fuzzers-are-much-faster-1}}
This is a python based fuzzer which also displays the packets on the GUI.
This convenience costs performance. If you want the best performance
you can use \sphinxcode{cangen} of the \sphinxcode{can-utils} package and import the
created packets later.


\subsubsection{What are the modes?}
\label{\detokenize{manual.fuzzertab:what-are-the-modes}}\begin{itemize}
\item {} 
User specified: You can freely specify ID and data masks

\item {} 
11 bit IDs / 29 bit IDs: Only short/extended IDs will be used

\end{itemize}


\subsection{Comparer Tab}
\label{\detokenize{manual.comparertab:comparer-tab}}\label{\detokenize{manual.comparertab::doc}}

\subsubsection{What can I compare?}
\label{\detokenize{manual.comparertab:what-can-i-compare}}
You can compare two sets of packets. You will get all packets
they have in common.


\subsection{Searcher Tab}
\label{\detokenize{manual.searchertab::doc}}\label{\detokenize{manual.searchertab:searcher-tab}}

\subsubsection{What can I search for?}
\label{\detokenize{manual.searchertab:what-can-i-search-for}}
Using this tab you can perform a binary packet search for a
specific packet or a whole packet set that cause an effect.
Let’s suppose you’ve fuzzed and got a large packet dump that, when
replayed, causes an effect on your CAN device / car. You now want to
extract the relevant packet(s) out of that dump. Searcher tab to the
rescue \textendash{} Load the whole packet dump and let the analyzer routine
guide you.
Note: This first tries to search for \sphinxstylestrong{1} packet that causes an action.
It this fails, the searcher tries to continously minimize the packet set.


\subsubsection{It doesn’t work!!1!}
\label{\detokenize{manual.searchertab:it-doesn-t-work-1}}
Don’t give up too fast, try the following things:
- Set the packet gap to a lower value, you can even try 0
- Just try again and hope for better shuffling
- Use another dump/fuzz again, …
- Wait a few seconds after each chunk


\subsubsection{It still doesn’t work :(}
\label{\detokenize{manual.searchertab:it-still-doesn-t-work}}
CAN devices can be extremely tricky, for example spedometers. Depending
on your dump, you may have to try it multiple times with the same dump
because of packet timings and/or bad luck. If you replay your whole
dump and see the desired action, you will be able to find it using
the searchter tab.


\subsubsection{I want to do it manually, how can this tool help me?}
\label{\detokenize{manual.searchertab:i-want-to-do-it-manually-how-can-this-tool-help-me}}
Create a new sender, add the dump to it and send them in a loop.
Minimize the packet set from the bottom using your “CTRL+C” and “DEL” and try
again. If it didn’t perform the desired action, paste the packets again
and delete other packets.


\subsection{Manager Tab}
\label{\detokenize{manual.managertab:manager-tab}}\label{\detokenize{manual.managertab::doc}}

\subsubsection{What can I do using the dumps tab?}
\label{\detokenize{manual.managertab:what-can-i-do-using-the-dumps-tab}}
You can save a set of packets that you want to keep (e.g. for further
analysis) and save it to the database. This allows you to load the
dump again at a later point.
Hint: You can edit the values in the GUI table and update the values
in the database using the update button.


\subsubsection{I know packet XY has effect ZZ, do I create a dump or a known packet?}
\label{\detokenize{manual.managertab:i-know-packet-xy-has-effect-zz-do-i-create-a-dump-or-a-known-packet}}
Just create a dump with one packet entry and the application will
handle the rest for you.


\subsubsection{Importing/Exporting projects}
\label{\detokenize{manual.managertab:importing-exporting-projects}}
If you want to import/export projects, use the manager tab. It exports
all saved data of a project to a editable textfile in JSON format.
Go ahead and edit values if you want, but be careful and don’t mess
with the data integrity \textless{}:


\subsection{Filter Tab}
\label{\detokenize{manual.filtertab::doc}}\label{\detokenize{manual.filtertab:filter-tab}}

\subsubsection{What can I filter?}
\label{\detokenize{manual.filtertab:what-can-i-filter}}
Let’s suppose you want to find (a) specific packet(s) that get
generated when you (for example) press a button, accelerate or lock
the cars doors. The captured CAN traffic contains so much data that
you can’t seem to find the packets easily. Let’s use the filter tab:
\begin{itemize}
\item {} 
You can collect background noise containing CAN packets that
are sent on the bus without any user interaction

\item {} 
After that, a variable amount of samples get captured. You have to
perform the desired action \sphinxstylestrong{in every sample} - e.g. lock the
doors in every sample.

\item {} 
As soon as all data has been captured, the filter tab begins to
analyze it. It filters background noise out of each sample and tries
to find packets that occur in every sample. These are most likely
the packets your are looking for.

\end{itemize}


\subsubsection{How can I try this without a car or CAN device?}
\label{\detokenize{manual.filtertab:how-can-i-try-this-without-a-car-or-can-device}}
Use ICSim!


\chapter{Contributing}
\label{\detokenize{contributing:contributing}}\label{\detokenize{contributing::doc}}
Here’s some useful info if you want to contribute.


\section{Guidelines}
\label{\detokenize{contributing:guidelines}}\begin{itemize}
\item {} 
Each tab has its own Class. If possible, inherit from {\hyperref[\detokenize{src:src.AbstractTab.AbstractTab}]{\sphinxcrossref{\sphinxcode{AbstractTab}}}}.

\end{itemize}
\begin{quote}
\begin{itemize}
\item {} 
To provide comatibility:

\end{itemize}
\begin{itemize}
\item {} 
The displayed data should also be in a raw data list called \sphinxcode{rawData} which is \sphinxstyleemphasis{always} up to date

\item {} 
\sphinxcode{prepareUI} initializes all GUI elements

\item {} 
\sphinxcode{active} manages the status of a tab

\item {} 
Tab specific CANData instances are called \sphinxcode{CANData}

\end{itemize}
\end{quote}
\begin{itemize}
\item {} 
Please log useful information using an own logger instance

\item {} 
Use existing Toolbox methods if possible

\item {} 
Use batch database operations using raw lists (not objects) for better performance

\item {} 
Use docstrings

\item {} 
Keep the \sphinxcode{.ui} files clean: Always name new GUI elements properly according to existing ones

\item {} 
Put new strings in the Strings file and reference it

\end{itemize}


\section{I want to add a new tab, what do I have to do?}
\label{\detokenize{contributing:i-want-to-add-a-new-tab-what-do-i-have-to-do}}\begin{itemize}
\item {} 
Create a new tab on the GUI and stick to the already existing naming conventions

\item {} 
Add a QTableView to display your data and other GUI elements

\item {} 
Update \sphinxtitleref{mainWindow.py} using \sphinxtitleref{pyside-uic mainWindow.ui \textgreater{} mainWindow.py}.

\item {} 
Add a new File and a new class which inherits from {\hyperref[\detokenize{src:src.AbstractTab.AbstractTab}]{\sphinxcrossref{\sphinxcode{AbstractTab}}}}

\item {} 
Call the parents constructor in your \sphinxcode{\_\_init\_\_}

\item {} 
Add the GUI elements from the \sphinxcode{.ui} file to your code. You can refer to the other tabs
to see how it’s done. Also, add the click handlers here.

\item {} 
Call \sphinxcode{prepareUI} as last action in \sphinxcode{\_\_init\_\_\_}

\item {} 
If your tab needs an interface or displays interface values: Add your tab
class or instance to {\hyperref[\detokenize{src:src.Toolbox.Toolbox.updateInterfaceLabels}]{\sphinxcrossref{\sphinxcode{updateInterfaceLabels()}}}} and/or {\hyperref[\detokenize{src:src.Toolbox.Toolbox.updateCANDataInstances}]{\sphinxcrossref{\sphinxcode{updateCANDataInstances()}}}}.

\item {} 
If your tab uses an instance: Add an instance to \sphinxtitleref{Globals.py} and create one at startup (see \sphinxtitleref{CANalyzat0r.py}).

\item {} 
If your tab uses a static class: Call \sphinxtitleref{prepareUI} at startup (see \sphinxtitleref{CANalyzat0r.py}).

\end{itemize}


\chapter{src package}
\label{\detokenize{src::doc}}\label{\detokenize{src:src-package}}

\section{Subpackages}
\label{\detokenize{src:subpackages}}

\subsection{src.ui package}
\label{\detokenize{src.ui:src-ui-package}}\label{\detokenize{src.ui::doc}}

\subsubsection{Submodules}
\label{\detokenize{src.ui:submodules}}

\subsubsection{CANalyzat0r.ui.mainWindow module}
\label{\detokenize{src.ui:module-src.ui.mainWindow}}\label{\detokenize{src.ui:canalyzat0r-ui-mainwindow-module}}\index{src.ui.mainWindow (module)}\index{Ui\_CANalyzatorMainWindow (class in src.ui.mainWindow)}

\begin{fulllineitems}
\phantomsection\label{\detokenize{src.ui:src.ui.mainWindow.Ui_CANalyzatorMainWindow}}\pysigline{\sphinxstrong{class }\sphinxcode{src.ui.mainWindow.}\sphinxbfcode{Ui\_CANalyzatorMainWindow}}
Bases: \sphinxcode{object}
\index{retranslateUi() (src.ui.mainWindow.Ui\_CANalyzatorMainWindow method)}

\begin{fulllineitems}
\phantomsection\label{\detokenize{src.ui:src.ui.mainWindow.Ui_CANalyzatorMainWindow.retranslateUi}}\pysiglinewithargsret{\sphinxbfcode{retranslateUi}}{\emph{CANalyzatorMainWindow}}{}
\end{fulllineitems}

\index{setupUi() (src.ui.mainWindow.Ui\_CANalyzatorMainWindow method)}

\begin{fulllineitems}
\phantomsection\label{\detokenize{src.ui:src.ui.mainWindow.Ui_CANalyzatorMainWindow.setupUi}}\pysiglinewithargsret{\sphinxbfcode{setupUi}}{\emph{CANalyzatorMainWindow}}{}
\end{fulllineitems}


\end{fulllineitems}



\subsubsection{Module contents}
\label{\detokenize{src.ui:module-src.ui}}\label{\detokenize{src.ui:module-contents}}\index{src.ui (module)}

\section{Submodules}
\label{\detokenize{src:submodules}}

\section{CANalyzat0r.AboutTab module}
\label{\detokenize{src:canalyzat0r-abouttab-module}}\label{\detokenize{src:module-src.AboutTab}}\index{src.AboutTab (module)}
Created on Jun 26, 2017

@author: pschmied
\index{AboutTab (class in src.AboutTab)}

\begin{fulllineitems}
\phantomsection\label{\detokenize{src:src.AboutTab.AboutTab}}\pysigline{\sphinxstrong{class }\sphinxcode{src.AboutTab.}\sphinxbfcode{AboutTab}}
This class handles the logic of the about tab.
\index{browseGitHub() (src.AboutTab.AboutTab static method)}

\begin{fulllineitems}
\phantomsection\label{\detokenize{src:src.AboutTab.AboutTab.browseGitHub}}\pysiglinewithargsret{\sphinxstrong{static }\sphinxbfcode{browseGitHub}}{\emph{event}}{}
Opens the SCHUTZWERK website.
\begin{quote}\begin{description}
\item[{Parameters}] \leavevmode
\sphinxstyleliteralstrong{event} \textendash{} Dummy, not used.

\end{description}\end{quote}

\end{fulllineitems}

\index{browseSW() (src.AboutTab.AboutTab static method)}

\begin{fulllineitems}
\phantomsection\label{\detokenize{src:src.AboutTab.AboutTab.browseSW}}\pysiglinewithargsret{\sphinxstrong{static }\sphinxbfcode{browseSW}}{\emph{event}}{}
Opens the SCHUTZWERK website.
\begin{quote}\begin{description}
\item[{Parameters}] \leavevmode
\sphinxstyleliteralstrong{event} \textendash{} Dummy, not used.

\end{description}\end{quote}

\end{fulllineitems}

\index{prepareUI() (src.AboutTab.AboutTab static method)}

\begin{fulllineitems}
\phantomsection\label{\detokenize{src:src.AboutTab.AboutTab.prepareUI}}\pysiglinewithargsret{\sphinxstrong{static }\sphinxbfcode{prepareUI}}{}{}
This just sets up the GUI elements.

\end{fulllineitems}


\end{fulllineitems}



\section{CANalyzat0r.CANData module}
\label{\detokenize{src:module-src.CANData}}\label{\detokenize{src:canalyzat0r-candata-module}}\index{src.CANData (module)}
Created on May 17, 2017

@author: pschmied
\index{CANData (class in src.CANData)}

\begin{fulllineitems}
\phantomsection\label{\detokenize{src:src.CANData.CANData}}\pysiglinewithargsret{\sphinxstrong{class }\sphinxcode{src.CANData.}\sphinxbfcode{CANData}}{\emph{ifaceName}, \emph{bitrate=500000}}{}~\index{CANDataInstances (src.CANData.CANData attribute)}

\begin{fulllineitems}
\phantomsection\label{\detokenize{src:src.CANData.CANData.CANDataInstances}}\pysigline{\sphinxbfcode{CANDataInstances}\sphinxstrong{ = \{\}}}
This dictionary stores all currently available CANData instances. The key
of the dictionary is the interface name

\end{fulllineitems}

\index{\_\_init\_\_() (src.CANData.CANData method)}

\begin{fulllineitems}
\phantomsection\label{\detokenize{src:src.CANData.CANData.__init__}}\pysiglinewithargsret{\sphinxbfcode{\_\_init\_\_}}{\emph{ifaceName}, \emph{bitrate=500000}}{}
This method initializes the pyvit CAN interface using the passed parameters and the start() method.
Please note the active-flag which protects the object from being deleted while being in use.
\begin{quote}\begin{description}
\item[{Parameters}] \leavevmode\begin{itemize}
\item {} 
\sphinxstyleliteralstrong{ifaceName} \textendash{} Name of the interface as displayed by \sphinxcode{ifconfig -a}

\item {} 
\sphinxstyleliteralstrong{bitrate} \textendash{} Desired bitrate of the interface

\end{itemize}

\end{description}\end{quote}

\end{fulllineitems}

\index{checkVCAN() (src.CANData.CANData method)}

\begin{fulllineitems}
\phantomsection\label{\detokenize{src:src.CANData.CANData.checkVCAN}}\pysiglinewithargsret{\sphinxbfcode{checkVCAN}}{}{}
Checks if the SocketCAN device is phyisical or virtual using a \sphinxcode{ls} call to \sphinxcode{/sys/devices/virtual/net}.
\begin{quote}\begin{description}
\item[{Returns}] \leavevmode
A boolean value indicating if the device is virtual (True) or not (False)

\end{description}\end{quote}

\end{fulllineitems}

\index{createCANDataInstance() (src.CANData.CANData static method)}

\begin{fulllineitems}
\phantomsection\label{\detokenize{src:src.CANData.CANData.createCANDataInstance}}\pysiglinewithargsret{\sphinxstrong{static }\sphinxbfcode{createCANDataInstance}}{\emph{ifaceName}, \emph{bitrate=500000}, \emph{returnObject=False}}{}
Creates a CANData instance with the desired data and either returns the object
or adds it to the CANDataInstances dictionary.
\begin{quote}\begin{description}
\item[{Parameters}] \leavevmode\begin{itemize}
\item {} 
\sphinxstyleliteralstrong{ifaceName} \textendash{} The desired interface name

\item {} 
\sphinxstyleliteralstrong{bitrate} \textendash{} The bitrate

\item {} 
\sphinxstyleliteralstrong{returnObject} \textendash{} Boolean value indicating whether the created object will be
returned or appended to the dictionary (XOR)

\end{itemize}

\item[{Returns}] \leavevmode
The created CANData object if returnObject is True

\end{description}\end{quote}

\end{fulllineitems}

\index{deleteCANDataInstance() (src.CANData.CANData static method)}

\begin{fulllineitems}
\phantomsection\label{\detokenize{src:src.CANData.CANData.deleteCANDataInstance}}\pysiglinewithargsret{\sphinxstrong{static }\sphinxbfcode{deleteCANDataInstance}}{\emph{ifaceName}}{}
Removes a CANData object from the CANDataInstances dictionary.
Note: CANData objects only will be deleted if the active flag is set to False
to prevent running operations from failing.
\begin{quote}\begin{description}
\item[{Parameters}] \leavevmode
\sphinxstyleliteralstrong{ifaceName} \textendash{} The name of the interface that will be deleted

\item[{Returns}] \leavevmode
A Boolean value indicating the success of the delete operation

\end{description}\end{quote}

\end{fulllineitems}

\index{getGlobalOrFirstInstance() (src.CANData.CANData static method)}

\begin{fulllineitems}
\phantomsection\label{\detokenize{src:src.CANData.CANData.getGlobalOrFirstInstance}}\pysiglinewithargsret{\sphinxstrong{static }\sphinxbfcode{getGlobalOrFirstInstance}}{}{}
Tries to return an available CANData instance (e.g. for startup of the application).
\begin{quote}\begin{description}
\item[{Returns}] \leavevmode
\begin{itemize}
\item {} 
The global CANData instance if available.

\item {} 
Else: The first element of all available instances.

\item {} 
Else: None of no interface is present

\end{itemize}


\end{description}\end{quote}

\end{fulllineitems}

\index{logger (src.CANData.CANData attribute)}

\begin{fulllineitems}
\phantomsection\label{\detokenize{src:src.CANData.CANData.logger}}\pysigline{\sphinxbfcode{logger}\sphinxstrong{ = \textless{}logging.Logger object\textgreater{}}}
Class specific logger instance

\end{fulllineitems}

\index{parseSocketCANLines() (src.CANData.CANData static method)}

\begin{fulllineitems}
\phantomsection\label{\detokenize{src:src.CANData.CANData.parseSocketCANLines}}\pysiglinewithargsret{\sphinxstrong{static }\sphinxbfcode{parseSocketCANLines}}{\emph{lines}}{}
Parses a list of SocketCAN lines  and generates a list of SocketCANPackets.
Note: The expected line format is e.g.:

\sphinxcode{(1493280437.565631) can0 1FD\#0000000000000000}
\begin{quote}\begin{description}
\item[{Parameters}] \leavevmode
\sphinxstyleliteralstrong{lines} \textendash{} List of lines in SocketCAN format

\item[{Returns}] \leavevmode
List of SocketCANPacket objects

\end{description}\end{quote}

\end{fulllineitems}

\index{readCANFile() (src.CANData.CANData static method)}

\begin{fulllineitems}
\phantomsection\label{\detokenize{src:src.CANData.CANData.readCANFile}}\pysiglinewithargsret{\sphinxstrong{static }\sphinxbfcode{readCANFile}}{\emph{filePath}}{}
Reads a file in SocketCAN format (as generated by candump from can-utils)
and returns a list of SocketCANPacket objects (see {\hyperref[\detokenize{src:src.CANData.SocketCANPacket}]{\sphinxcrossref{\sphinxcode{SocketCANPacket}}}}).
\begin{quote}\begin{description}
\item[{Parameters}] \leavevmode
\sphinxstyleliteralstrong{filePath} \textendash{} The path of the dump file that has to be read

\item[{Returns}] \leavevmode
A list of SocketCANPackets

\end{description}\end{quote}

\end{fulllineitems}

\index{readPacket() (src.CANData.CANData method)}

\begin{fulllineitems}
\phantomsection\label{\detokenize{src:src.CANData.CANData.readPacket}}\pysiglinewithargsret{\sphinxbfcode{readPacket}}{}{}
Read a packet from the queue using the SocketCAN interface.
Note: This blocks as long as no packet is being received.
You can use {\hyperref[\detokenize{src:src.CANData.CANData.readPacketAsync}]{\sphinxcrossref{\sphinxcode{readPacketAsync()}}}} to read
packets with a timeout.
\begin{quote}\begin{description}
\item[{Returns}] \leavevmode
A packet as pyvit frame

\end{description}\end{quote}

\end{fulllineitems}

\index{readPacketAsync() (src.CANData.CANData method)}

\begin{fulllineitems}
\phantomsection\label{\detokenize{src:src.CANData.CANData.readPacketAsync}}\pysiglinewithargsret{\sphinxbfcode{readPacketAsync}}{}{}
Read a packet from the queue using the SocketCAN interface \sphinxstylestrong{and a timeout}.
\begin{quote}\begin{description}
\item[{Returns}] \leavevmode
A packet as pyvit frame or None of no packet was received

\end{description}\end{quote}

\end{fulllineitems}

\index{rebuildCANDataInstances() (src.CANData.CANData static method)}

\begin{fulllineitems}
\phantomsection\label{\detokenize{src:src.CANData.CANData.rebuildCANDataInstances}}\pysiglinewithargsret{\sphinxstrong{static }\sphinxbfcode{rebuildCANDataInstances}}{\emph{CANIfaceNameList}}{}
Refreshes the CANDataInstances dictionary with up-to-date values using the parameter CANIfaceNameList.
Old objects will be kept, new ones will be created and missing ones will be deleted.
\begin{quote}\begin{description}
\item[{Parameters}] \leavevmode
\sphinxstyleliteralstrong{CANIfaceNameList} \textendash{} Names of interfaces that must be present in the dictionary after this method

\item[{Returns}] \leavevmode
A list of removed interface names to handle the consequences of deleting an object.

\end{description}\end{quote}

\end{fulllineitems}

\index{sendPacket() (src.CANData.CANData method)}

\begin{fulllineitems}
\phantomsection\label{\detokenize{src:src.CANData.CANData.sendPacket}}\pysiglinewithargsret{\sphinxbfcode{sendPacket}}{\emph{packet}}{}
Sends a packet using the SocketCAN interface.
\begin{quote}\begin{description}
\item[{Parameters}] \leavevmode
\sphinxstyleliteralstrong{packet} \textendash{} A packet as pyvit frame (see {\hyperref[\detokenize{src:src.CANData.CANData.tryBuildPacket}]{\sphinxcrossref{\sphinxcode{tryBuildPacket()}}}})

\end{description}\end{quote}

\end{fulllineitems}

\index{timeout (src.CANData.CANData attribute)}

\begin{fulllineitems}
\phantomsection\label{\detokenize{src:src.CANData.CANData.timeout}}\pysigline{\sphinxbfcode{timeout}\sphinxstrong{ = None}}
1 second read timeout for async reads (see {\hyperref[\detokenize{src:src.CANData.CANData.readPacketAsync}]{\sphinxcrossref{\sphinxcode{readPacketAsync()}}}})

\end{fulllineitems}

\index{toString() (src.CANData.CANData method)}

\begin{fulllineitems}
\phantomsection\label{\detokenize{src:src.CANData.CANData.toString}}\pysiglinewithargsret{\sphinxbfcode{toString}}{}{}
Return a string to display the currently used settings and interface name on the GUI
\begin{quote}\begin{description}
\item[{Returns}] \leavevmode
A string which consists of either: The interface name (for virtual CAN devices where no bitrate is available)
or the interface name along with the currently used bitrate in kBit/s

\end{description}\end{quote}

\end{fulllineitems}

\index{tryBuildPacket() (src.CANData.CANData static method)}

\begin{fulllineitems}
\phantomsection\label{\detokenize{src:src.CANData.CANData.tryBuildPacket}}\pysiglinewithargsret{\sphinxstrong{static }\sphinxbfcode{tryBuildPacket}}{\emph{CANID}, \emph{data}}{}
Builds a pyvit frame using the passed parameters.
This method automatically uses the extended CAN format if needed.
\begin{quote}\begin{description}
\item[{Parameters}] \leavevmode\begin{itemize}
\item {} 
\sphinxstyleliteralstrong{CANID} \textendash{} The CAN ID as hex string

\item {} 
\sphinxstyleliteralstrong{data} \textendash{} The desired packet data (length must be even)

\end{itemize}

\item[{Returns}] \leavevmode
A packet as pyvit frame containing the passed data or None of no frame can be created

\end{description}\end{quote}

\end{fulllineitems}

\index{updateBitrate() (src.CANData.CANData method)}

\begin{fulllineitems}
\phantomsection\label{\detokenize{src:src.CANData.CANData.updateBitrate}}\pysiglinewithargsret{\sphinxbfcode{updateBitrate}}{\emph{bitrate}}{}
Updates the bitrate of the SocketCAN interface (if possible).
\begin{quote}\begin{description}
\item[{Parameters}] \leavevmode
\sphinxstyleliteralstrong{bitrate} \textendash{} The desired bitrate in bit/s

\item[{Returns}] \leavevmode
A boolean value indicating success of updating the bitrate value

\end{description}\end{quote}

\end{fulllineitems}

\index{writeCANFile() (src.CANData.CANData static method)}

\begin{fulllineitems}
\phantomsection\label{\detokenize{src:src.CANData.CANData.writeCANFile}}\pysiglinewithargsret{\sphinxstrong{static }\sphinxbfcode{writeCANFile}}{\emph{filePath}, \emph{packets}}{}
Writes/Exports SocketCANPacket objects to a textfile.
\begin{quote}\begin{description}
\item[{Parameters}] \leavevmode\begin{itemize}
\item {} 
\sphinxstyleliteralstrong{filePath} \textendash{} Path of the file to be saved to

\item {} 
\sphinxstyleliteralstrong{packets} \textendash{} List of SocketCANPacket objects

\end{itemize}

\end{description}\end{quote}

\end{fulllineitems}


\end{fulllineitems}

\index{SocketCANPacket (class in src.CANData)}

\begin{fulllineitems}
\phantomsection\label{\detokenize{src:src.CANData.SocketCANPacket}}\pysiglinewithargsret{\sphinxstrong{class }\sphinxcode{src.CANData.}\sphinxbfcode{SocketCANPacket}}{\emph{timestamp}, \emph{iface}, \emph{id}, \emph{data}}{}
This class is used to manage data from/to SocketCAN format in a nice manner \textless{}:
\index{\_\_init\_\_() (src.CANData.SocketCANPacket method)}

\begin{fulllineitems}
\phantomsection\label{\detokenize{src:src.CANData.SocketCANPacket.__init__}}\pysiglinewithargsret{\sphinxbfcode{\_\_init\_\_}}{\emph{timestamp}, \emph{iface}, \emph{id}, \emph{data}}{}
This just sets data.
\begin{quote}\begin{description}
\item[{Parameters}] \leavevmode\begin{itemize}
\item {} 
\sphinxstyleliteralstrong{timestamp} \textendash{} Timestamp of the packet

\item {} 
\sphinxstyleliteralstrong{iface} \textendash{} Interface the packet was captured from

\item {} 
\sphinxstyleliteralstrong{id} \textendash{} CAN ID

\item {} 
\sphinxstyleliteralstrong{data} \textendash{} Data payload

\end{itemize}

\end{description}\end{quote}

\end{fulllineitems}

\index{toString() (src.CANData.SocketCANPacket method)}

\begin{fulllineitems}
\phantomsection\label{\detokenize{src:src.CANData.SocketCANPacket.toString}}\pysiglinewithargsret{\sphinxbfcode{toString}}{}{}
Returns the string representation of the current object.
ID lengths will be padded:
\begin{itemize}
\item {} 
length \textless{}= 3 \textendash{}\textgreater{} length = 3

\item {} 
length \textgreater{} 3 \textendash{}\textgreater{} length = 8

\end{itemize}
\begin{quote}\begin{description}
\item[{Returns}] \leavevmode
A string with the data of the current object

\end{description}\end{quote}

\end{fulllineitems}


\end{fulllineitems}



\section{CANalyzat0r.CANalyzat0r module}
\label{\detokenize{src:canalyzat0r-canalyzat0r-module}}\label{\detokenize{src:module-src.CANalyzat0r}}\index{src.CANalyzat0r (module)}
Created on May 17, 2017

@author: pschmied
\index{MainWindow (class in src.CANalyzat0r)}

\begin{fulllineitems}
\phantomsection\label{\detokenize{src:src.CANalyzat0r.MainWindow}}\pysigline{\sphinxstrong{class }\sphinxcode{src.CANalyzat0r.}\sphinxbfcode{MainWindow}}
Bases: \sphinxcode{PySide.QtGui.QMainWindow}, {\hyperref[\detokenize{src.ui:src.ui.mainWindow.Ui_CANalyzatorMainWindow}]{\sphinxcrossref{\sphinxcode{src.ui.mainWindow.Ui\_CANalyzatorMainWindow}}}}
\index{\_\_init\_\_() (src.CANalyzat0r.MainWindow method)}

\begin{fulllineitems}
\phantomsection\label{\detokenize{src:src.CANalyzat0r.MainWindow.__init__}}\pysiglinewithargsret{\sphinxbfcode{\_\_init\_\_}}{}{}~\begin{description}
\item[{This method has to take care of the following things:}] \leavevmode\begin{enumerate}
\setcounter{enumi}{-1}
\item {} 
Lazy import tab modules

\item {} 
Initialize the main UI

\item {} 
Setup the database connection and ensure all tables are present

\item {} 
Detect all currently attached CAN devices

\item {} 
Call the prepareUI()-method of every tab

\item {} 
\{En, Dis\}able GUI elements based on the presence of a CAN device

\item {} 
Setup logging

\item {} 
Install a globlal exception hook that will catch all “remaining” exceptions
to log it to the GUI

\item {} 
Load the CAN kernel modules

\item {} 
Check if superuser privileges are present - exit if not present

\item {} 
Install event handlers for GUI elements (assignWidgets())

\end{enumerate}

\end{description}

\end{fulllineitems}

\index{assignWidgets() (src.CANalyzat0r.MainWindow method)}

\begin{fulllineitems}
\phantomsection\label{\detokenize{src:src.CANalyzat0r.MainWindow.assignWidgets}}\pysiglinewithargsret{\sphinxbfcode{assignWidgets}}{}{}
This method connects all GUI elements of the static tabs to their event handlers.
For all other tabs (those that inherit from {\hyperref[\detokenize{src:src.AbstractTab.AbstractTab}]{\sphinxcrossref{\sphinxcode{AbstractTab}}}}, this is
done in the constructor

\end{fulllineitems}

\index{checkSU() (src.CANalyzat0r.MainWindow method)}

\begin{fulllineitems}
\phantomsection\label{\detokenize{src:src.CANalyzat0r.MainWindow.checkSU}}\pysiglinewithargsret{\sphinxbfcode{checkSU}}{}{}
This method gets the effective UID and returns a
boolean value indicating if root privileges are available.
\begin{description}
\item[{Returns:}] \leavevmode
A boolean value indicating the superuser status

\end{description}

\end{fulllineitems}

\index{cleanup() (src.CANalyzat0r.MainWindow static method)}

\begin{fulllineitems}
\phantomsection\label{\detokenize{src:src.CANalyzat0r.MainWindow.cleanup}}\pysiglinewithargsret{\sphinxstrong{static }\sphinxbfcode{cleanup}}{}{}
This gets called when exiting.
This cleans up everything \textless{}:

\end{fulllineitems}

\index{staticMetaObject (src.CANalyzat0r.MainWindow attribute)}

\begin{fulllineitems}
\phantomsection\label{\detokenize{src:src.CANalyzat0r.MainWindow.staticMetaObject}}\pysigline{\sphinxbfcode{staticMetaObject}\sphinxstrong{ = \textless{}PySide.QtCore.QMetaObject object\textgreater{}}}
\end{fulllineitems}


\end{fulllineitems}

\index{globalLoggingHandler() (in module src.CANalyzat0r)}

\begin{fulllineitems}
\phantomsection\label{\detokenize{src:src.CANalyzat0r.globalLoggingHandler}}\pysiglinewithargsret{\sphinxcode{src.CANalyzat0r.}\sphinxbfcode{globalLoggingHandler}}{\emph{type}, \emph{value}, \emph{tb}}{}
This is the handler method for the global exception hook which parses the message in the traceback (tb).
Using this it is possible to log the exception and the last executed line of code.
\begin{quote}\begin{description}
\item[{Parameters}] \leavevmode\begin{itemize}
\item {} 
\sphinxstyleliteralstrong{type} \textendash{} Exception class

\item {} 
\sphinxstyleliteralstrong{value} \textendash{} Exception value (the object)

\item {} 
\sphinxstyleliteralstrong{tb} \textendash{} Traceback object containing the previously exectuted lines of code

\end{itemize}

\end{description}\end{quote}

\end{fulllineitems}

\index{uncaughtExceptionLogger (in module src.CANalyzat0r)}

\begin{fulllineitems}
\phantomsection\label{\detokenize{src:src.CANalyzat0r.uncaughtExceptionLogger}}\pysigline{\sphinxcode{src.CANalyzat0r.}\sphinxbfcode{uncaughtExceptionLogger}\sphinxstrong{ = \textless{}logging.Logger object\textgreater{}}}
Logger instance to log uncaught exceptions using {\hyperref[\detokenize{src:src.CANalyzat0r.globalLoggingHandler}]{\sphinxcrossref{\sphinxcode{globalLoggingHandler()}}}}

\end{fulllineitems}



\section{CANalyzat0r.ComparerTab module}
\label{\detokenize{src:canalyzat0r-comparertab-module}}\label{\detokenize{src:module-src.ComparerTab}}\index{src.ComparerTab (module)}
Created on Jun 30, 2017

@author: pschmied
\index{ComparerTab (class in src.ComparerTab)}

\begin{fulllineitems}
\phantomsection\label{\detokenize{src:src.ComparerTab.ComparerTab}}\pysiglinewithargsret{\sphinxstrong{class }\sphinxcode{src.ComparerTab.}\sphinxbfcode{ComparerTab}}{\emph{tabWidget}}{}
Bases: {\hyperref[\detokenize{src:src.AbstractTab.AbstractTab}]{\sphinxcrossref{\sphinxcode{src.AbstractTab.AbstractTab}}}}

This handles the logic of the comparer tab.
\index{\_\_init\_\_() (src.ComparerTab.ComparerTab method)}

\begin{fulllineitems}
\phantomsection\label{\detokenize{src:src.ComparerTab.ComparerTab.__init__}}\pysiglinewithargsret{\sphinxbfcode{\_\_init\_\_}}{\emph{tabWidget}}{}
This just sets data and adds click handlers.

\end{fulllineitems}

\index{compare() (src.ComparerTab.ComparerTab method)}

\begin{fulllineitems}
\phantomsection\label{\detokenize{src:src.ComparerTab.ComparerTab.compare}}\pysiglinewithargsret{\sphinxbfcode{compare}}{}{}
Compares \sphinxcode{rawPacketSet1} and \sphinxcode{rawPacketSet2} and display the packet they have in common on the GUI.

\end{fulllineitems}

\index{getPacketSet() (src.ComparerTab.ComparerTab method)}

\begin{fulllineitems}
\phantomsection\label{\detokenize{src:src.ComparerTab.ComparerTab.getPacketSet}}\pysiglinewithargsret{\sphinxbfcode{getPacketSet}}{\emph{rawPacketList}}{}
Opens a {\hyperref[\detokenize{src:src.PacketsDialog.PacketsDialog}]{\sphinxcrossref{\sphinxcode{PacketsDialog}}}} to load selected packets into the passed raw packet list.
\begin{quote}\begin{description}
\item[{Parameters}] \leavevmode
\sphinxstyleliteralstrong{rawPacketList} \textendash{} The raw packet list

\end{description}\end{quote}

\end{fulllineitems}

\index{setPacketSet1() (src.ComparerTab.ComparerTab method)}

\begin{fulllineitems}
\phantomsection\label{\detokenize{src:src.ComparerTab.ComparerTab.setPacketSet1}}\pysiglinewithargsret{\sphinxbfcode{setPacketSet1}}{}{}
Opens a {\hyperref[\detokenize{src:src.PacketsDialog.PacketsDialog}]{\sphinxcrossref{\sphinxcode{PacketsDialog}}}} to load packet set 1 into \sphinxcode{rawPacketSet1}.

\end{fulllineitems}

\index{setPacketSet2() (src.ComparerTab.ComparerTab method)}

\begin{fulllineitems}
\phantomsection\label{\detokenize{src:src.ComparerTab.ComparerTab.setPacketSet2}}\pysiglinewithargsret{\sphinxbfcode{setPacketSet2}}{}{}
Opens a {\hyperref[\detokenize{src:src.PacketsDialog.PacketsDialog}]{\sphinxcrossref{\sphinxcode{PacketsDialog}}}} to load packet set 2 into \sphinxcode{rawPacketSet2}.

\end{fulllineitems}


\end{fulllineitems}



\section{CANalyzat0r.Database module}
\label{\detokenize{src:canalyzat0r-database-module}}
The database model is as follows:

Note: The \sphinxstylestrong{bold} columns are \sphinxcode{NOT NULL}
\phantomsection\label{\detokenize{src:module-src.Database}}\index{src.Database (module)}
Created on May 17, 2017

@author: pschmied
\index{Database (class in src.Database)}

\begin{fulllineitems}
\phantomsection\label{\detokenize{src:src.Database.Database}}\pysigline{\sphinxstrong{class }\sphinxcode{src.Database.}\sphinxbfcode{Database}}
This class handles the database connection and is responsible for creating, deleting and updating values
\index{\_\_init\_\_() (src.Database.Database method)}

\begin{fulllineitems}
\phantomsection\label{\detokenize{src:src.Database.Database.__init__}}\pysiglinewithargsret{\sphinxbfcode{\_\_init\_\_}}{}{}~\begin{description}
\item[{This method does the following things:}] \leavevmode\begin{enumerate}
\item {} 
Setup logging

\item {} 
Create a DB connection and check the integrity

\item {} 
Create tables if necessary

\end{enumerate}

\end{description}

\end{fulllineitems}

\index{checkDB() (src.Database.Database method)}

\begin{fulllineitems}
\phantomsection\label{\detokenize{src:src.Database.Database.checkDB}}\pysiglinewithargsret{\sphinxbfcode{checkDB}}{}{}
Checks if all the table count of the SQLite database matches the needed table count.
If the check does pass the user will be notified to create a project if no project is exisiting yet.
If the check does not pass the user will be prompted for an action:
\begin{itemize}
\item {} 
Truncate the database and create an empty one

\item {} 
Keep the database and exit

\end{itemize}
\begin{quote}\begin{description}
\item[{Returns}] \leavevmode
A boolean value indicating the database integrity status (True = good)

\end{description}\end{quote}

\end{fulllineitems}

\index{connect() (src.Database.Database method)}

\begin{fulllineitems}
\phantomsection\label{\detokenize{src:src.Database.Database.connect}}\pysiglinewithargsret{\sphinxbfcode{connect}}{}{}
Connect to the hard coded SQLite database path (see Settings).

Note: If a DatabaseError is encountered, the application will close with error code 1
\begin{quote}\begin{description}
\item[{Returns}] \leavevmode
A SQLite3 connection object

\end{description}\end{quote}

\end{fulllineitems}

\index{createTables() (src.Database.Database method)}

\begin{fulllineitems}
\phantomsection\label{\detokenize{src:src.Database.Database.createTables}}\pysiglinewithargsret{\sphinxbfcode{createTables}}{}{}
Creates all needed tables.
Check {\hyperref[\detokenize{src:src.Database.DatabaseStatements}]{\sphinxcrossref{\sphinxcode{DatabaseStatements}}}} for the SQL statements.

\end{fulllineitems}

\index{deleteFromTableByID() (src.Database.Database method)}

\begin{fulllineitems}
\phantomsection\label{\detokenize{src:src.Database.Database.deleteFromTableByID}}\pysiglinewithargsret{\sphinxbfcode{deleteFromTableByID}}{\emph{tableName}, \emph{id}}{}
Delete a row from a table with a specific ID.
\begin{quote}\begin{description}
\item[{Parameters}] \leavevmode\begin{itemize}
\item {} 
\sphinxstyleliteralstrong{tableName} \textendash{} The table to delete from

\item {} 
\sphinxstyleliteralstrong{id} \textendash{} ID of the record to delete

\end{itemize}

\end{description}\end{quote}

\end{fulllineitems}

\index{deleteFromTableByValue() (src.Database.Database method)}

\begin{fulllineitems}
\phantomsection\label{\detokenize{src:src.Database.Database.deleteFromTableByValue}}\pysiglinewithargsret{\sphinxbfcode{deleteFromTableByValue}}{\emph{tableName}, \emph{column}, \emph{value}}{}
Delete rows from a table with a specific value.
\begin{quote}\begin{description}
\item[{Parameters}] \leavevmode\begin{itemize}
\item {} 
\sphinxstyleliteralstrong{tableName} \textendash{} The table to delete from

\item {} 
\sphinxstyleliteralstrong{column} \textendash{} The column to check

\item {} 
\sphinxstyleliteralstrong{value} \textendash{} The value to search for

\end{itemize}

\end{description}\end{quote}

\end{fulllineitems}

\index{deleteKnownPacket() (src.Database.Database method)}

\begin{fulllineitems}
\phantomsection\label{\detokenize{src:src.Database.Database.deleteKnownPacket}}\pysiglinewithargsret{\sphinxbfcode{deleteKnownPacket}}{\emph{knownPacket}}{}
Delete a KnownPacket object
\begin{quote}\begin{description}
\item[{Parameters}] \leavevmode
\sphinxstyleliteralstrong{knownPacket} \textendash{} The KnownPacket object to delete

\item[{Returns}] \leavevmode


\end{description}\end{quote}

\end{fulllineitems}

\index{deletePacketSet() (src.Database.Database method)}

\begin{fulllineitems}
\phantomsection\label{\detokenize{src:src.Database.Database.deletePacketSet}}\pysiglinewithargsret{\sphinxbfcode{deletePacketSet}}{\emph{packetSet}}{}
Delete a PacketSet along with the associated packets
\begin{quote}\begin{description}
\item[{Parameters}] \leavevmode
\sphinxstyleliteralstrong{packetSet} \textendash{} The PacketSet object to delete

\item[{Returns}] \leavevmode


\end{description}\end{quote}

\end{fulllineitems}

\index{deleteProjectAndData() (src.Database.Database method)}

\begin{fulllineitems}
\phantomsection\label{\detokenize{src:src.Database.Database.deleteProjectAndData}}\pysiglinewithargsret{\sphinxbfcode{deleteProjectAndData}}{\emph{project}}{}
Delete a project and all associated data.
\begin{quote}\begin{description}
\item[{Parameters}] \leavevmode
\sphinxstyleliteralstrong{project} \textendash{} The Project object to delete

\end{description}\end{quote}

\end{fulllineitems}

\index{getKnownPackets() (src.Database.Database method)}

\begin{fulllineitems}
\phantomsection\label{\detokenize{src:src.Database.Database.getKnownPackets}}\pysiglinewithargsret{\sphinxbfcode{getKnownPackets}}{\emph{project=None}}{}
Get all known packets of a Project as objects.
Uses the global project if no project is given.
\begin{quote}\begin{description}
\item[{Parameters}] \leavevmode
\sphinxstyleliteralstrong{project} \textendash{} Optional parameter to specify the project to use

\item[{Returns}] \leavevmode
A list of all known packets as KnownPacket objects

\end{description}\end{quote}

\end{fulllineitems}

\index{getOverallTableCount() (src.Database.Database method)}

\begin{fulllineitems}
\phantomsection\label{\detokenize{src:src.Database.Database.getOverallTableCount}}\pysiglinewithargsret{\sphinxbfcode{getOverallTableCount}}{\emph{tableName}}{}
Returns the count(*) of a table.
\begin{quote}\begin{description}
\item[{Parameters}] \leavevmode
\sphinxstyleliteralstrong{tableName} \textendash{} The table to count the rows of

\item[{Returns}] \leavevmode
The number of rows as integer

\end{description}\end{quote}

\end{fulllineitems}

\index{getPacketSets() (src.Database.Database method)}

\begin{fulllineitems}
\phantomsection\label{\detokenize{src:src.Database.Database.getPacketSets}}\pysiglinewithargsret{\sphinxbfcode{getPacketSets}}{\emph{project=None}}{}
Get all packet sets of a Project as objects.
Uses the global project if no project is given.
\begin{quote}\begin{description}
\item[{Parameters}] \leavevmode
\sphinxstyleliteralstrong{project} \textendash{} Optional parameter to specify the project to use

\item[{Returns}] \leavevmode
A list of all known packets as PacketSet objects

\end{description}\end{quote}

\end{fulllineitems}

\index{getPacketsOfPacketSet() (src.Database.Database method)}

\begin{fulllineitems}
\phantomsection\label{\detokenize{src:src.Database.Database.getPacketsOfPacketSet}}\pysiglinewithargsret{\sphinxbfcode{getPacketsOfPacketSet}}{\emph{packetSet}, \emph{raw=False}}{}
Get all packets of a specific packet set.
Note: Use raw=True for better performance.
\begin{quote}\begin{description}
\item[{Parameters}] \leavevmode\begin{itemize}
\item {} 
\sphinxstyleliteralstrong{packetSet} \textendash{} All returned packets will belong to this packet set

\item {} 
\sphinxstyleliteralstrong{raw} \textendash{} Boolean value to indicate if the packets will be returned as raw data list (True)
or as list of objects (False)

\end{itemize}

\item[{Returns}] \leavevmode
Depending on the value of raw:
- True: List of value lists (raw data)
- False: List of Packet objects

\end{description}\end{quote}

\end{fulllineitems}

\index{getProjects() (src.Database.Database method)}

\begin{fulllineitems}
\phantomsection\label{\detokenize{src:src.Database.Database.getProjects}}\pysiglinewithargsret{\sphinxbfcode{getProjects}}{}{}
Get all available projects as Project objects.
\begin{quote}\begin{description}
\item[{Returns}] \leavevmode
A list of all projects as Project objects

\end{description}\end{quote}

\end{fulllineitems}

\index{saveKnownPacket() (src.Database.Database method)}

\begin{fulllineitems}
\phantomsection\label{\detokenize{src:src.Database.Database.saveKnownPacket}}\pysiglinewithargsret{\sphinxbfcode{saveKnownPacket}}{\emph{knownPacket}}{}
Save a known packt to the database.
\begin{quote}\begin{description}
\item[{Parameters}] \leavevmode
\sphinxstyleliteralstrong{knownPacket} \textendash{} The KnownPacket object to save

\item[{Returns}] \leavevmode
The database ID of the saved known packet

\end{description}\end{quote}

\end{fulllineitems}

\index{savePacket() (src.Database.Database method)}

\begin{fulllineitems}
\phantomsection\label{\detokenize{src:src.Database.Database.savePacket}}\pysiglinewithargsret{\sphinxbfcode{savePacket}}{\emph{packet=None}, \emph{packetSetID=None}, \emph{CANID=None}, \emph{data=None}, \emph{timestamp=”}, \emph{iface=”}, \emph{commit=True}}{}
Save a packet to the database: Either by object Oo by list-values \textendash{}\textgreater{} Faster for many values
If the value in packet is not None: The passed object will be used
Else: The seperate values will be used
\begin{quote}\begin{description}
\item[{Parameters}] \leavevmode\begin{itemize}
\item {} 
\sphinxstyleliteralstrong{packet} \textendash{} Optional parameter: Packet object to save

\item {} 
\sphinxstyleliteralstrong{packetSetID} \textendash{} PacketSet ID of the saved packet

\item {} 
\sphinxstyleliteralstrong{CANID} \textendash{} CAN ID

\item {} 
\sphinxstyleliteralstrong{data} \textendash{} Payload data

\item {} 
\sphinxstyleliteralstrong{timestamp} \textendash{} Timestamp

\item {} 
\sphinxstyleliteralstrong{iface} \textendash{} The interface the packet was captured from

\item {} 
\sphinxstyleliteralstrong{commit} \textendash{} If the operation will be commite to the database or not. Batch operations use commit=False

\end{itemize}

\item[{Returns}] \leavevmode
The database ID of the saved packet if commit is True. Else -1

\end{description}\end{quote}

\end{fulllineitems}

\index{savePacketSet() (src.Database.Database method)}

\begin{fulllineitems}
\phantomsection\label{\detokenize{src:src.Database.Database.savePacketSet}}\pysiglinewithargsret{\sphinxbfcode{savePacketSet}}{\emph{packetSet}}{}
Save a PacketSet to the database.
If there’s a name collision, the user will be prompted for a new and unique name.
\begin{quote}\begin{description}
\item[{Parameters}] \leavevmode
\sphinxstyleliteralstrong{packetSet} \textendash{} The PacketSet to save

\end{description}\end{quote}

\end{fulllineitems}

\index{savePacketSetWithData() (src.Database.Database method)}

\begin{fulllineitems}
\phantomsection\label{\detokenize{src:src.Database.Database.savePacketSetWithData}}\pysiglinewithargsret{\sphinxbfcode{savePacketSetWithData}}{\emph{packetSetName}, \emph{rawPackets=None}, \emph{project=None}, \emph{packets=None}}{}
Save a packet set in the database with the given data.
If no project is given, the global project will be used.
You must specifiy \sphinxcode{rawPackets} or \sphinxcode{packets}.
\begin{quote}\begin{description}
\item[{Parameters}] \leavevmode\begin{itemize}
\item {} 
\sphinxstyleliteralstrong{packetSetName} \textendash{} The desired name of the PacketSet

\item {} 
\sphinxstyleliteralstrong{rawPackets} \textendash{} Optional: Raw Packet data (List of lists). You can also use \sphinxcode{packets}.

\item {} 
\sphinxstyleliteralstrong{project} \textendash{} Optional parameter to specify a project. If this is not specified, the selected project
will be used

\item {} 
\sphinxstyleliteralstrong{packets} \textendash{} Optional; List of packet objects to save to the packet set.
If this is specified, \sphinxcode{rawPackets} will be ignored

\end{itemize}

\item[{Returns}] \leavevmode


\end{description}\end{quote}

\end{fulllineitems}

\index{savePacketsBatch() (src.Database.Database method)}

\begin{fulllineitems}
\phantomsection\label{\detokenize{src:src.Database.Database.savePacketsBatch}}\pysiglinewithargsret{\sphinxbfcode{savePacketsBatch}}{\emph{packetSetID}, \emph{rawPackets=None}, \emph{packets=None}}{}
Save many packets as a batch to the database.
Use this for improved speed: No objects, only 1 commit.
\begin{quote}\begin{description}
\item[{Parameters}] \leavevmode\begin{itemize}
\item {} 
\sphinxstyleliteralstrong{packetSetID} \textendash{} The PacketSet ID the packets belong to

\item {} 
\sphinxstyleliteralstrong{rawPackets} \textendash{} Optional: Packet data as raw data list (List of lists)

\item {} 
\sphinxstyleliteralstrong{packets} \textendash{} Optional: List of packet objects to save. If this is not None, this will be used instead of
\sphinxcode{rawPackets}

\end{itemize}

\end{description}\end{quote}

\end{fulllineitems}

\index{saveProject() (src.Database.Database method)}

\begin{fulllineitems}
\phantomsection\label{\detokenize{src:src.Database.Database.saveProject}}\pysiglinewithargsret{\sphinxbfcode{saveProject}}{\emph{project}}{}
Save a Project objet to the database
\begin{quote}\begin{description}
\item[{Parameters}] \leavevmode
\sphinxstyleliteralstrong{project} \textendash{} The project to save

\item[{Returns}] \leavevmode
The assigned database ID

\end{description}\end{quote}

\end{fulllineitems}

\index{updateKnownPacket() (src.Database.Database method)}

\begin{fulllineitems}
\phantomsection\label{\detokenize{src:src.Database.Database.updateKnownPacket}}\pysiglinewithargsret{\sphinxbfcode{updateKnownPacket}}{\emph{knownPacket}}{}
Update a known packet object in the database
\begin{quote}\begin{description}
\item[{Parameters}] \leavevmode
\sphinxstyleliteralstrong{knownPacket} \textendash{} The KnownPacket object which holds the updated values

\item[{Returns}] \leavevmode


\end{description}\end{quote}

\end{fulllineitems}

\index{updatePackets() (src.Database.Database method)}

\begin{fulllineitems}
\phantomsection\label{\detokenize{src:src.Database.Database.updatePackets}}\pysiglinewithargsret{\sphinxbfcode{updatePackets}}{\emph{rowList}, \emph{packetSet}, \emph{packetIDsToRemove}}{}
Update the packets of a specific packet set
\begin{quote}\begin{description}
\item[{Parameters}] \leavevmode\begin{itemize}
\item {} 
\sphinxstyleliteralstrong{rowList} \textendash{} A list containing raw packet data
e.g.: \sphinxcode{{[}{[}("CANID", "232"), ("DATA", "BEEF"), packetID{]}, {[}("DATA", "C0FFEE"), ("CANID", "137"), packetID{]}{]}}

\item {} 
\sphinxstyleliteralstrong{packetSet} \textendash{} The PacketSet object the data in rowList belongs to

\item {} 
\sphinxstyleliteralstrong{packetIDsToRemove} \textendash{} Database IDs of packets to remove

\end{itemize}

\end{description}\end{quote}

\end{fulllineitems}

\index{updateProject() (src.Database.Database method)}

\begin{fulllineitems}
\phantomsection\label{\detokenize{src:src.Database.Database.updateProject}}\pysiglinewithargsret{\sphinxbfcode{updateProject}}{\emph{project}}{}
Update a Project object in the database.
\begin{quote}\begin{description}
\item[{Parameters}] \leavevmode
\sphinxstyleliteralstrong{project} \textendash{} The Project object which holds the updated values

\end{description}\end{quote}

\end{fulllineitems}


\end{fulllineitems}

\index{DatabaseStatements (class in src.Database)}

\begin{fulllineitems}
\phantomsection\label{\detokenize{src:src.Database.DatabaseStatements}}\pysigline{\sphinxstrong{class }\sphinxcode{src.Database.}\sphinxbfcode{DatabaseStatements}}
This class is used to store and generate database statements.
\index{checkTablesPresentStatement (src.Database.DatabaseStatements attribute)}

\begin{fulllineitems}
\phantomsection\label{\detokenize{src:src.Database.DatabaseStatements.checkTablesPresentStatement}}\pysigline{\sphinxbfcode{checkTablesPresentStatement}\sphinxstrong{ = “SELECT name FROM sqlite\_master WHERE type=’table’”}}
Statement which gets the names of all currently available tables in the database
to check the integrity

\end{fulllineitems}

\index{createKnownPacketTableStatement (src.Database.DatabaseStatements attribute)}

\begin{fulllineitems}
\phantomsection\label{\detokenize{src:src.Database.DatabaseStatements.createKnownPacketTableStatement}}\pysigline{\sphinxbfcode{createKnownPacketTableStatement}\sphinxstrong{ = ‘CREATE TABLE {}`KnownPacket{}` (\textbackslash{}n\textbackslash{}t{}`ID{}`\textbackslash{}tINTEGER PRIMARY KEY,\textbackslash{}n\textbackslash{}t{}`ProjectID{}`\textbackslash{}tINTEGER NOT NULL,\textbackslash{}n\textbackslash{}t{}`CANID{}`\textbackslash{}tTEXT NOT NULL,\textbackslash{}n\textbackslash{}t{}`Data{}`\textbackslash{}tTEXT,\textbackslash{}n\textbackslash{}t{}`Description{}`\textbackslash{}tTEXT NOT NULL,\textbackslash{}n\textbackslash{}tFOREIGN KEY(ProjectID) REFERENCES Project(ID)\textbackslash{}n    );’}}
\end{fulllineitems}

\index{createPacketSetTableStatement (src.Database.DatabaseStatements attribute)}

\begin{fulllineitems}
\phantomsection\label{\detokenize{src:src.Database.DatabaseStatements.createPacketSetTableStatement}}\pysigline{\sphinxbfcode{createPacketSetTableStatement}\sphinxstrong{ = ‘CREATE TABLE {}`PacketSet{}` (\textbackslash{}n\textbackslash{}t{}`ID{}`\textbackslash{}tINTEGER PRIMARY KEY,\textbackslash{}n\textbackslash{}t{}`ProjectID{}`\textbackslash{}tINTEGER NOT NULL,\textbackslash{}n\textbackslash{}t{}`Name{}`\textbackslash{}tTEXT NOT NULL,\textbackslash{}n\textbackslash{}t{}`Date{}`\textbackslash{}tTEXT NOT NULL,\textbackslash{}n\textbackslash{}tUNIQUE(ProjectID, Name),\textbackslash{}n\textbackslash{}tFOREIGN KEY(ProjectID) REFERENCES Project(ID)\textbackslash{}n    );’}}
Note: Unique index for the combination of
Project ID and Name
\textendash{}\textgreater{} A PacketSets name is unique for a project

\end{fulllineitems}

\index{createPacketTableStatement (src.Database.DatabaseStatements attribute)}

\begin{fulllineitems}
\phantomsection\label{\detokenize{src:src.Database.DatabaseStatements.createPacketTableStatement}}\pysigline{\sphinxbfcode{createPacketTableStatement}\sphinxstrong{ = ‘CREATE TABLE {}`Packet{}` (\textbackslash{}n\textbackslash{}t{}`ID{}`\textbackslash{}tINTEGER PRIMARY KEY,\textbackslash{}n\textbackslash{}t{}`PacketSetID{}`\textbackslash{}tINTEGER NOT NULL,\textbackslash{}n\textbackslash{}t{}`CANID{}`\textbackslash{}tTEXT NOT NULL,\textbackslash{}n\textbackslash{}t{}`Data{}`\textbackslash{}tTEXT,\textbackslash{}n\textbackslash{}t{}`Timestamp{}`\textbackslash{}tTEXT,\textbackslash{}n\textbackslash{}t{}`Interface{}`\textbackslash{}tTEXT,\textbackslash{}n\textbackslash{}tFOREIGN KEY(PacketSetID) REFERENCES PacketSet(ID)\textbackslash{}n    );’}}
\end{fulllineitems}

\index{createProjectTableStatement (src.Database.DatabaseStatements attribute)}

\begin{fulllineitems}
\phantomsection\label{\detokenize{src:src.Database.DatabaseStatements.createProjectTableStatement}}\pysigline{\sphinxbfcode{createProjectTableStatement}\sphinxstrong{ = ‘CREATE TABLE {}`Project{}` (\textbackslash{}n\textbackslash{}t{}`ID{}`\textbackslash{}tINTEGER PRIMARY KEY,\textbackslash{}n\textbackslash{}t{}`Name{}`\textbackslash{}tTEXT NOT NULL UNIQUE,\textbackslash{}n\textbackslash{}t{}`Description{}`\textbackslash{}tTEXT,\textbackslash{}n\textbackslash{}t{}`Date{}`\textbackslash{}tTEXT NOT NULL\textbackslash{}n    );’}}
Project names are unique

\end{fulllineitems}

\index{createTableStatementsList (src.Database.DatabaseStatements attribute)}

\begin{fulllineitems}
\phantomsection\label{\detokenize{src:src.Database.DatabaseStatements.createTableStatementsList}}\pysigline{\sphinxbfcode{createTableStatementsList}\sphinxstrong{ = {[}‘CREATE TABLE {}`Project{}` (\textbackslash{}n\textbackslash{}t{}`ID{}`\textbackslash{}tINTEGER PRIMARY KEY,\textbackslash{}n\textbackslash{}t{}`Name{}`\textbackslash{}tTEXT NOT NULL UNIQUE,\textbackslash{}n\textbackslash{}t{}`Description{}`\textbackslash{}tTEXT,\textbackslash{}n\textbackslash{}t{}`Date{}`\textbackslash{}tTEXT NOT NULL\textbackslash{}n    );’, ‘CREATE TABLE {}`Packet{}` (\textbackslash{}n\textbackslash{}t{}`ID{}`\textbackslash{}tINTEGER PRIMARY KEY,\textbackslash{}n\textbackslash{}t{}`PacketSetID{}`\textbackslash{}tINTEGER NOT NULL,\textbackslash{}n\textbackslash{}t{}`CANID{}`\textbackslash{}tTEXT NOT NULL,\textbackslash{}n\textbackslash{}t{}`Data{}`\textbackslash{}tTEXT,\textbackslash{}n\textbackslash{}t{}`Timestamp{}`\textbackslash{}tTEXT,\textbackslash{}n\textbackslash{}t{}`Interface{}`\textbackslash{}tTEXT,\textbackslash{}n\textbackslash{}tFOREIGN KEY(PacketSetID) REFERENCES PacketSet(ID)\textbackslash{}n    );’, ‘CREATE TABLE {}`PacketSet{}` (\textbackslash{}n\textbackslash{}t{}`ID{}`\textbackslash{}tINTEGER PRIMARY KEY,\textbackslash{}n\textbackslash{}t{}`ProjectID{}`\textbackslash{}tINTEGER NOT NULL,\textbackslash{}n\textbackslash{}t{}`Name{}`\textbackslash{}tTEXT NOT NULL,\textbackslash{}n\textbackslash{}t{}`Date{}`\textbackslash{}tTEXT NOT NULL,\textbackslash{}n\textbackslash{}tUNIQUE(ProjectID, Name),\textbackslash{}n\textbackslash{}tFOREIGN KEY(ProjectID) REFERENCES Project(ID)\textbackslash{}n    );’, ‘CREATE TABLE {}`KnownPacket{}` (\textbackslash{}n\textbackslash{}t{}`ID{}`\textbackslash{}tINTEGER PRIMARY KEY,\textbackslash{}n\textbackslash{}t{}`ProjectID{}`\textbackslash{}tINTEGER NOT NULL,\textbackslash{}n\textbackslash{}t{}`CANID{}`\textbackslash{}tTEXT NOT NULL,\textbackslash{}n\textbackslash{}t{}`Data{}`\textbackslash{}tTEXT,\textbackslash{}n\textbackslash{}t{}`Description{}`\textbackslash{}tTEXT NOT NULL,\textbackslash{}n\textbackslash{}tFOREIGN KEY(ProjectID) REFERENCES Project(ID)\textbackslash{}n    );’{]}}}
Holds all needed create table statements

\end{fulllineitems}

\index{getDeleteByIDStatement() (src.Database.DatabaseStatements static method)}

\begin{fulllineitems}
\phantomsection\label{\detokenize{src:src.Database.DatabaseStatements.getDeleteByIDStatement}}\pysiglinewithargsret{\sphinxstrong{static }\sphinxbfcode{getDeleteByIDStatement}}{\emph{tableName}, \emph{id}}{}
Returns an SQL delete statement using an ID where clause using {\hyperref[\detokenize{src:src.Database.DatabaseStatements.getDeleteByValueStatement}]{\sphinxcrossref{\sphinxcode{getDeleteByValueStatement()}}}}
\begin{quote}\begin{description}
\item[{Parameters}] \leavevmode\begin{itemize}
\item {} 
\sphinxstyleliteralstrong{tableName} \textendash{} The table name to delete from

\item {} 
\sphinxstyleliteralstrong{id} \textendash{} Desired ID value for the where clause

\end{itemize}

\item[{Returns}] \leavevmode

The resulting SQL statement

e.g.: \sphinxcode{DELETE FROM TABLE1 WHERE ID = 1337}


\end{description}\end{quote}

\end{fulllineitems}

\index{getDeleteByValueStatement() (src.Database.DatabaseStatements static method)}

\begin{fulllineitems}
\phantomsection\label{\detokenize{src:src.Database.DatabaseStatements.getDeleteByValueStatement}}\pysiglinewithargsret{\sphinxstrong{static }\sphinxbfcode{getDeleteByValueStatement}}{\emph{tableName}, \emph{column}, \emph{value}}{}
Returns an SQL delete statement using a where clause using {\hyperref[\detokenize{src:src.Database.DatabaseStatements.getDeleteByValueStatement}]{\sphinxcrossref{\sphinxcode{getDeleteByValueStatement()}}}}
\begin{quote}\begin{description}
\item[{Parameters}] \leavevmode\begin{itemize}
\item {} 
\sphinxstyleliteralstrong{tableName} \textendash{} The table name to delete from

\item {} 
\sphinxstyleliteralstrong{column} \textendash{} Desired column for the where clause

\item {} 
\sphinxstyleliteralstrong{value} \textendash{} Desired column value for the where clause

\end{itemize}

\item[{Returns}] \leavevmode

The resulting SQL statement

e.g.: \sphinxcode{DELETE FROM TABLE1 WHERE NAME = 'BANANA'}


\end{description}\end{quote}

\end{fulllineitems}

\index{getInsertKnownPacketStatement() (src.Database.DatabaseStatements static method)}

\begin{fulllineitems}
\phantomsection\label{\detokenize{src:src.Database.DatabaseStatements.getInsertKnownPacketStatement}}\pysiglinewithargsret{\sphinxstrong{static }\sphinxbfcode{getInsertKnownPacketStatement}}{\emph{projectID}, \emph{CANID}, \emph{data}, \emph{description}}{}
Returns an SQL insert statement for a KnownPacket using {\hyperref[\detokenize{src:src.Database.DatabaseStatements.getInsertStatement}]{\sphinxcrossref{\sphinxcode{getInsertStatement()}}}}.
\begin{quote}\begin{description}
\item[{Parameters}] \leavevmode\begin{itemize}
\item {} 
\sphinxstyleliteralstrong{projectID} \textendash{} The project this KnownPacket belongs to

\item {} 
\sphinxstyleliteralstrong{CANID} \textendash{} CAN ID

\item {} 
\sphinxstyleliteralstrong{data} \textendash{} Payload data

\item {} 
\sphinxstyleliteralstrong{description} \textendash{} What this specific Packet does

\end{itemize}

\item[{Returns}] \leavevmode
The SQL insert statement with all KnownPacket specific values set

\end{description}\end{quote}

\end{fulllineitems}

\index{getInsertPacketSetStatement() (src.Database.DatabaseStatements static method)}

\begin{fulllineitems}
\phantomsection\label{\detokenize{src:src.Database.DatabaseStatements.getInsertPacketSetStatement}}\pysiglinewithargsret{\sphinxstrong{static }\sphinxbfcode{getInsertPacketSetStatement}}{\emph{projectID}, \emph{name}, \emph{date}}{}
Returns an SQL insert statement for a PacketSet using {\hyperref[\detokenize{src:src.Database.DatabaseStatements.getInsertStatement}]{\sphinxcrossref{\sphinxcode{getInsertStatement()}}}}.
\begin{quote}\begin{description}
\item[{Parameters}] \leavevmode\begin{itemize}
\item {} 
\sphinxstyleliteralstrong{projectID} \textendash{} The project ID the record belongs to

\item {} 
\sphinxstyleliteralstrong{name} \textendash{} The name of the PacketSet

\item {} 
\sphinxstyleliteralstrong{date} \textendash{} Date

\end{itemize}

\item[{Returns}] \leavevmode
The SQL insert statement with all PacketSet specific values set

\end{description}\end{quote}

\end{fulllineitems}

\index{getInsertPacketStatement() (src.Database.DatabaseStatements static method)}

\begin{fulllineitems}
\phantomsection\label{\detokenize{src:src.Database.DatabaseStatements.getInsertPacketStatement}}\pysiglinewithargsret{\sphinxstrong{static }\sphinxbfcode{getInsertPacketStatement}}{\emph{packetSetID}, \emph{CANID}, \emph{data}, \emph{timestamp}, \emph{iface}}{}
Returns an SQL insert statement for a Packet using {\hyperref[\detokenize{src:src.Database.DatabaseStatements.getInsertStatement}]{\sphinxcrossref{\sphinxcode{getInsertStatement()}}}}.
\begin{quote}\begin{description}
\item[{Parameters}] \leavevmode\begin{itemize}
\item {} 
\sphinxstyleliteralstrong{packetSetID} \textendash{} The PacketSet this Packet belongs to

\item {} 
\sphinxstyleliteralstrong{CANID} \textendash{} CAN ID

\item {} 
\sphinxstyleliteralstrong{data} \textendash{} Payload data

\item {} 
\sphinxstyleliteralstrong{timestamp} \textendash{} Timestamp of the packet

\item {} 
\sphinxstyleliteralstrong{iface} \textendash{} Interface name from which this packet was captured from

\end{itemize}

\item[{Returns}] \leavevmode
The SQL insert statement with all Packet specific values set

\end{description}\end{quote}

\end{fulllineitems}

\index{getInsertProjectStatement() (src.Database.DatabaseStatements static method)}

\begin{fulllineitems}
\phantomsection\label{\detokenize{src:src.Database.DatabaseStatements.getInsertProjectStatement}}\pysiglinewithargsret{\sphinxstrong{static }\sphinxbfcode{getInsertProjectStatement}}{\emph{name}, \emph{desription}, \emph{date}}{}
Returns an SQL insert statement for a project using {\hyperref[\detokenize{src:src.Database.DatabaseStatements.getInsertStatement}]{\sphinxcrossref{\sphinxcode{getInsertStatement()}}}}.
\begin{quote}\begin{description}
\item[{Parameters}] \leavevmode\begin{itemize}
\item {} 
\sphinxstyleliteralstrong{name} \textendash{} Projectname

\item {} 
\sphinxstyleliteralstrong{desription} \textendash{} Project description

\item {} 
\sphinxstyleliteralstrong{date} \textendash{} Project date

\end{itemize}

\item[{Returns}] \leavevmode
The SQL insert statement with all project specific values set

\end{description}\end{quote}

\end{fulllineitems}

\index{getInsertStatement() (src.Database.DatabaseStatements static method)}

\begin{fulllineitems}
\phantomsection\label{\detokenize{src:src.Database.DatabaseStatements.getInsertStatement}}\pysiglinewithargsret{\sphinxstrong{static }\sphinxbfcode{getInsertStatement}}{\emph{tableName}, \emph{columnList}, \emph{valuesList}}{}
Builds a SQL insert statement using the passed parameters
\begin{quote}\begin{description}
\item[{Parameters}] \leavevmode\begin{itemize}
\item {} 
\sphinxstyleliteralstrong{tableName} \textendash{} The table name to insert into

\item {} 
\sphinxstyleliteralstrong{columnList} \textendash{} List of column names that will be affected

\item {} 
\sphinxstyleliteralstrong{valuesList} \textendash{} List of values to put into the columns

\end{itemize}

\item[{Returns}] \leavevmode

An SQL insert statement with the desired values mapped to the columns

e.g. \sphinxcode{INSERT INTO TABLE1 (col1, col2) VALUES (1, 2)}


\end{description}\end{quote}

\end{fulllineitems}

\index{getOverallCountStatement() (src.Database.DatabaseStatements static method)}

\begin{fulllineitems}
\phantomsection\label{\detokenize{src:src.Database.DatabaseStatements.getOverallCountStatement}}\pysiglinewithargsret{\sphinxstrong{static }\sphinxbfcode{getOverallCountStatement}}{\emph{tableName}}{}
Returns an SQL select count statement for the desired table using all rows
\begin{quote}\begin{description}
\item[{Parameters}] \leavevmode
\sphinxstyleliteralstrong{tableName} \textendash{} The table name to get the rowcount from

\item[{Returns}] \leavevmode

The resulting SQL statement

e.g.: \sphinxcode{SELECT COUNT(*) FROM TABLE1}


\end{description}\end{quote}

\end{fulllineitems}

\index{getSelectAllStatement() (src.Database.DatabaseStatements static method)}

\begin{fulllineitems}
\phantomsection\label{\detokenize{src:src.Database.DatabaseStatements.getSelectAllStatement}}\pysiglinewithargsret{\sphinxstrong{static }\sphinxbfcode{getSelectAllStatement}}{\emph{tableName}}{}
Returns an SQL select statement to get all data from a table.
\begin{quote}\begin{description}
\item[{Parameters}] \leavevmode
\sphinxstyleliteralstrong{tableName} \textendash{} The table name from which data will be selected

\item[{Returns}] \leavevmode

The resulting SQL select statement

e.g.: \sphinxcode{SELECT * FROM TABLE1}


\end{description}\end{quote}

\end{fulllineitems}

\index{getSelectAllStatementWhereEquals() (src.Database.DatabaseStatements static method)}

\begin{fulllineitems}
\phantomsection\label{\detokenize{src:src.Database.DatabaseStatements.getSelectAllStatementWhereEquals}}\pysiglinewithargsret{\sphinxstrong{static }\sphinxbfcode{getSelectAllStatementWhereEquals}}{\emph{tableName}, \emph{column}, \emph{value}}{}
Returns an SQL select statement to gather all data from a table using a where clause
\begin{quote}\begin{description}
\item[{Parameters}] \leavevmode\begin{itemize}
\item {} 
\sphinxstyleliteralstrong{tableName} \textendash{} The table name from which data will be selected

\item {} 
\sphinxstyleliteralstrong{column} \textendash{} The column which the where clause affects

\item {} 
\sphinxstyleliteralstrong{value} \textendash{} The desired value of the column

\end{itemize}

\item[{Returns}] \leavevmode

The resulting SQL statement with where clause

e.g.: \sphinxcode{SELECT * FROM TABLE1 WHERE ID = 1337}


\end{description}\end{quote}

\end{fulllineitems}

\index{getUpdateByIDStatement() (src.Database.DatabaseStatements static method)}

\begin{fulllineitems}
\phantomsection\label{\detokenize{src:src.Database.DatabaseStatements.getUpdateByIDStatement}}\pysiglinewithargsret{\sphinxstrong{static }\sphinxbfcode{getUpdateByIDStatement}}{\emph{tableName}, \emph{colValuePairs}, \emph{ID}}{}
Builds a SQL update statement using the passed parameters \sphinxstylestrong{and an ID}
\begin{quote}\begin{description}
\item[{Parameters}] \leavevmode\begin{itemize}
\item {} 
\sphinxstyleliteralstrong{tableName} \textendash{} The table name to update

\item {} 
\sphinxstyleliteralstrong{colValuePairs} \textendash{} List of tuples: (column, desired value)

\item {} 
\sphinxstyleliteralstrong{ID} \textendash{} The ID of the record to update

\end{itemize}

\item[{Returns}] \leavevmode

An SQL update statement with the desired values mapped to the columns using the ID

e.g. \sphinxcode{UPDATE TABLE1 SET col1 = 1, col2 = 2 WHERE ID = 1337}


\end{description}\end{quote}

\end{fulllineitems}

\index{knownPacketTableCANIDColName (src.Database.DatabaseStatements attribute)}

\begin{fulllineitems}
\phantomsection\label{\detokenize{src:src.Database.DatabaseStatements.knownPacketTableCANIDColName}}\pysigline{\sphinxbfcode{knownPacketTableCANIDColName}\sphinxstrong{ = ‘CANID’}}
\end{fulllineitems}

\index{knownPacketTableDataColName (src.Database.DatabaseStatements attribute)}

\begin{fulllineitems}
\phantomsection\label{\detokenize{src:src.Database.DatabaseStatements.knownPacketTableDataColName}}\pysigline{\sphinxbfcode{knownPacketTableDataColName}\sphinxstrong{ = ‘Data’}}
\end{fulllineitems}

\index{knownPacketTableDescriptionColName (src.Database.DatabaseStatements attribute)}

\begin{fulllineitems}
\phantomsection\label{\detokenize{src:src.Database.DatabaseStatements.knownPacketTableDescriptionColName}}\pysigline{\sphinxbfcode{knownPacketTableDescriptionColName}\sphinxstrong{ = ‘Description’}}
\end{fulllineitems}

\index{knownPacketTableName (src.Database.DatabaseStatements attribute)}

\begin{fulllineitems}
\phantomsection\label{\detokenize{src:src.Database.DatabaseStatements.knownPacketTableName}}\pysigline{\sphinxbfcode{knownPacketTableName}\sphinxstrong{ = ‘KnownPacket’}}
\end{fulllineitems}

\index{packetSetTableName (src.Database.DatabaseStatements attribute)}

\begin{fulllineitems}
\phantomsection\label{\detokenize{src:src.Database.DatabaseStatements.packetSetTableName}}\pysigline{\sphinxbfcode{packetSetTableName}\sphinxstrong{ = ‘PacketSet’}}
\end{fulllineitems}

\index{packetTableCANIDColName (src.Database.DatabaseStatements attribute)}

\begin{fulllineitems}
\phantomsection\label{\detokenize{src:src.Database.DatabaseStatements.packetTableCANIDColName}}\pysigline{\sphinxbfcode{packetTableCANIDColName}\sphinxstrong{ = ‘CANID’}}
\end{fulllineitems}

\index{packetTableDataColName (src.Database.DatabaseStatements attribute)}

\begin{fulllineitems}
\phantomsection\label{\detokenize{src:src.Database.DatabaseStatements.packetTableDataColName}}\pysigline{\sphinxbfcode{packetTableDataColName}\sphinxstrong{ = ‘Data’}}
\end{fulllineitems}

\index{packetTableName (src.Database.DatabaseStatements attribute)}

\begin{fulllineitems}
\phantomsection\label{\detokenize{src:src.Database.DatabaseStatements.packetTableName}}\pysigline{\sphinxbfcode{packetTableName}\sphinxstrong{ = ‘Packet’}}
\end{fulllineitems}

\index{projectTableDescriptionColName (src.Database.DatabaseStatements attribute)}

\begin{fulllineitems}
\phantomsection\label{\detokenize{src:src.Database.DatabaseStatements.projectTableDescriptionColName}}\pysigline{\sphinxbfcode{projectTableDescriptionColName}\sphinxstrong{ = ‘Description’}}
\end{fulllineitems}

\index{projectTableName (src.Database.DatabaseStatements attribute)}

\begin{fulllineitems}
\phantomsection\label{\detokenize{src:src.Database.DatabaseStatements.projectTableName}}\pysigline{\sphinxbfcode{projectTableName}\sphinxstrong{ = ‘Project’}}
\end{fulllineitems}

\index{projectTableNameColName (src.Database.DatabaseStatements attribute)}

\begin{fulllineitems}
\phantomsection\label{\detokenize{src:src.Database.DatabaseStatements.projectTableNameColName}}\pysigline{\sphinxbfcode{projectTableNameColName}\sphinxstrong{ = ‘Name’}}
\end{fulllineitems}

\index{tableCount (src.Database.DatabaseStatements attribute)}

\begin{fulllineitems}
\phantomsection\label{\detokenize{src:src.Database.DatabaseStatements.tableCount}}\pysigline{\sphinxbfcode{tableCount}\sphinxstrong{ = 4}}
The Amount of tables that must be present

\end{fulllineitems}


\end{fulllineitems}



\section{CANalyzat0r.FilterTab module}
\label{\detokenize{src:canalyzat0r-filtertab-module}}\label{\detokenize{src:module-src.FilterTab}}\index{src.FilterTab (module)}
Created on Jun 02, 2017

@author: pschmied
\index{DataAdderThread (class in src.FilterTab)}

\begin{fulllineitems}
\phantomsection\label{\detokenize{src:src.FilterTab.DataAdderThread}}\pysiglinewithargsret{\sphinxstrong{class }\sphinxcode{src.FilterTab.}\sphinxbfcode{DataAdderThread}}{\emph{snifferReceivePipe}, \emph{sharedEnabledFlag}, \emph{curSampleIndex}}{}
Bases: \sphinxcode{PySide.QtCore.QThread}

This thread receives data from the sniffer process and
emits a signal which causes the main thread to add the packets.
\index{\_\_init\_\_() (src.FilterTab.DataAdderThread method)}

\begin{fulllineitems}
\phantomsection\label{\detokenize{src:src.FilterTab.DataAdderThread.__init__}}\pysiglinewithargsret{\sphinxbfcode{\_\_init\_\_}}{\emph{snifferReceivePipe}, \emph{sharedEnabledFlag}, \emph{curSampleIndex}}{}
\end{fulllineitems}

\index{frameToList() (src.FilterTab.DataAdderThread method)}

\begin{fulllineitems}
\phantomsection\label{\detokenize{src:src.FilterTab.DataAdderThread.frameToList}}\pysiglinewithargsret{\sphinxbfcode{frameToList}}{\emph{frame}}{}
Converts a received pyvit frame to raw list data.
After that, the data is emitted using \sphinxcode{signalSniffedPacket}
\begin{quote}\begin{description}
\item[{Parameters}] \leavevmode
\sphinxstyleliteralstrong{frame} \textendash{} pyvit CAN frame

\end{description}\end{quote}

\end{fulllineitems}

\index{run() (src.FilterTab.DataAdderThread method)}

\begin{fulllineitems}
\phantomsection\label{\detokenize{src:src.FilterTab.DataAdderThread.run}}\pysiglinewithargsret{\sphinxbfcode{run}}{}{}
As long as \sphinxcode{sharedEnabledFlag} is not set to \sphinxcode{0} data will be
received using the pipe and processed using {\hyperref[\detokenize{src:src.FilterTab.DataAdderThread.frameToList}]{\sphinxcrossref{\sphinxcode{frameToList()}}}}.

\end{fulllineitems}

\index{signalSniffedPacket (src.FilterTab.DataAdderThread attribute)}

\begin{fulllineitems}
\phantomsection\label{\detokenize{src:src.FilterTab.DataAdderThread.signalSniffedPacket}}\pysigline{\sphinxbfcode{signalSniffedPacket}\sphinxstrong{ = \textless{}PySide.QtCore.Signal object\textgreater{}}}
Emit a signal to the main thread when items are ready to be added
This emits the packet and the current sample index

\end{fulllineitems}

\index{staticMetaObject (src.FilterTab.DataAdderThread attribute)}

\begin{fulllineitems}
\phantomsection\label{\detokenize{src:src.FilterTab.DataAdderThread.staticMetaObject}}\pysigline{\sphinxbfcode{staticMetaObject}\sphinxstrong{ = \textless{}PySide.QtCore.QMetaObject object\textgreater{}}}
\end{fulllineitems}


\end{fulllineitems}

\index{FilterTab (class in src.FilterTab)}

\begin{fulllineitems}
\phantomsection\label{\detokenize{src:src.FilterTab.FilterTab}}\pysiglinewithargsret{\sphinxstrong{class }\sphinxcode{src.FilterTab.}\sphinxbfcode{FilterTab}}{\emph{tabWidget}}{}
Bases: {\hyperref[\detokenize{src:src.AbstractTab.AbstractTab}]{\sphinxcrossref{\sphinxcode{src.AbstractTab.AbstractTab}}}}

This class handles the logic of the filter tab
\index{\_\_init\_\_() (src.FilterTab.FilterTab method)}

\begin{fulllineitems}
\phantomsection\label{\detokenize{src:src.FilterTab.FilterTab.__init__}}\pysiglinewithargsret{\sphinxbfcode{\_\_init\_\_}}{\emph{tabWidget}}{}
\end{fulllineitems}

\index{addSniffedNoise() (src.FilterTab.FilterTab method)}

\begin{fulllineitems}
\phantomsection\label{\detokenize{src:src.FilterTab.FilterTab.addSniffedNoise}}\pysiglinewithargsret{\sphinxbfcode{addSniffedNoise}}{\emph{dummyIndex}, \emph{packet}}{}
Adds the passed packet data to \sphinxcode{noiseData}.
This method gets called by a DataAdderThread.
\begin{quote}\begin{description}
\item[{Parameters}] \leavevmode\begin{itemize}
\item {} 
\sphinxstyleliteralstrong{dummyIndex} \textendash{} Not used, only exists to match the signature defined in the signal of the DataAdderThread

\item {} 
\sphinxstyleliteralstrong{packet} \textendash{} The packet object to extract and add data from

\end{itemize}

\end{description}\end{quote}

\end{fulllineitems}

\index{addSniffedPacketToSample() (src.FilterTab.FilterTab method)}

\begin{fulllineitems}
\phantomsection\label{\detokenize{src:src.FilterTab.FilterTab.addSniffedPacketToSample}}\pysiglinewithargsret{\sphinxbfcode{addSniffedPacketToSample}}{\emph{curSampleIndex}, \emph{packet}}{}
Adds a sniffed packet to the sample defined by \sphinxcode{curSampleIndex}.
Gets called by a DataAdderThread.
\begin{quote}\begin{description}
\item[{Parameters}] \leavevmode\begin{itemize}
\item {} 
\sphinxstyleliteralstrong{curSampleIndex} \textendash{} The sample index to get a list from \sphinxcode{rawData{[}curSampleIndex{]}}

\item {} 
\sphinxstyleliteralstrong{packet} \textendash{} Packet data to add

\end{itemize}

\item[{Returns}] \leavevmode


\end{description}\end{quote}

\end{fulllineitems}

\index{analyze() (src.FilterTab.FilterTab method)}

\begin{fulllineitems}
\phantomsection\label{\detokenize{src:src.FilterTab.FilterTab.analyze}}\pysiglinewithargsret{\sphinxbfcode{analyze}}{\emph{removeNoiseWithIDAndData=True}}{}~\begin{description}
\item[{Analyze captured data:}] \leavevmode\begin{enumerate}
\item {} 
Remove sorted noise data (if collected):
For each sample:
\begin{itemize}
\item {} 
Sort the sample to increase filtering speed

\item {} 
Remove noise

\end{itemize}

\item {} \begin{description}
\item[{Find relevant packets:}] \leavevmode\begin{itemize}
\item {} 
Sort each sample to increase filtering speed

\item {} 
Assume that all packets of the first sample occurred in every other sample

\item {} 
Take every other sample and compare

\end{itemize}

\end{description}

\end{enumerate}

\end{description}

Depending on \sphinxcode{removeNoiseWithIDAndData} noise will be filtered by
ID and data (default) or ID only
\begin{quote}\begin{description}
\item[{Parameters}] \leavevmode
\sphinxstyleliteralstrong{removeNoiseWithIDAndData} \textendash{} Optional value: Filter data by ID and Data or ID only

\end{description}\end{quote}

\end{fulllineitems}

\index{clear() (src.FilterTab.FilterTab method)}

\begin{fulllineitems}
\phantomsection\label{\detokenize{src:src.FilterTab.FilterTab.clear}}\pysiglinewithargsret{\sphinxbfcode{clear}}{\emph{returnOldPackets=False}}{}
Clear the currently displayed data on the GUI and in the lists.

\end{fulllineitems}

\index{collectNoise() (src.FilterTab.FilterTab method)}

\begin{fulllineitems}
\phantomsection\label{\detokenize{src:src.FilterTab.FilterTab.collectNoise}}\pysiglinewithargsret{\sphinxbfcode{collectNoise}}{\emph{seconds}}{}
Collect noise data and update \sphinxcode{noiseData}.
Uses the processes/threads started in {\hyperref[\detokenize{src:src.FilterTab.FilterTab.startSnifferAndAdder}]{\sphinxcrossref{\sphinxcode{startSnifferAndAdder()}}}}.
\begin{quote}\begin{description}
\item[{Parameters}] \leavevmode
\sphinxstyleliteralstrong{seconds} \textendash{} Amount of seconds to capture noise

\item[{Returns}] \leavevmode
True if noise was captured. False if the user pressed “Cancel”

\end{description}\end{quote}

\end{fulllineitems}

\index{getSampleData() (src.FilterTab.FilterTab method)}

\begin{fulllineitems}
\phantomsection\label{\detokenize{src:src.FilterTab.FilterTab.getSampleData}}\pysiglinewithargsret{\sphinxbfcode{getSampleData}}{\emph{sampleAmount}, \emph{curSampleIndex}}{}
Collect sample data and add the sniffed data to \sphinxcode{rawData{[}curSampleIndex{]}}.
Uses the processes/threads started using {\hyperref[\detokenize{src:src.FilterTab.FilterTab.startSnifferAndAdder}]{\sphinxcrossref{\sphinxcode{startSnifferAndAdder()}}}}.
\begin{quote}\begin{description}
\item[{Parameters}] \leavevmode\begin{itemize}
\item {} 
\sphinxstyleliteralstrong{sampleAmount} \textendash{} Amount of samples to collect

\item {} 
\sphinxstyleliteralstrong{curSampleIndex} \textendash{} Index of the currently captured sample in \sphinxcode{rawData}

\end{itemize}

\end{description}\end{quote}

\end{fulllineitems}

\index{noiseData (src.FilterTab.FilterTab attribute)}

\begin{fulllineitems}
\phantomsection\label{\detokenize{src:src.FilterTab.FilterTab.noiseData}}\pysigline{\sphinxbfcode{noiseData}\sphinxstrong{ = None}}
Noise that will be substracted from the collected data

\end{fulllineitems}

\index{outputRemainingPackets() (src.FilterTab.FilterTab method)}

\begin{fulllineitems}
\phantomsection\label{\detokenize{src:src.FilterTab.FilterTab.outputRemainingPackets}}\pysiglinewithargsret{\sphinxbfcode{outputRemainingPackets}}{\emph{remainingPackets}}{}
Output all remaining packets after filtering to the table view.
Note: This also clears previous data
\begin{quote}\begin{description}
\item[{Parameters}] \leavevmode
\sphinxstyleliteralstrong{remainingPackets} \textendash{} Raw packet list to display

\end{description}\end{quote}

\end{fulllineitems}

\index{sharedDataAdderEnabledFlag (src.FilterTab.FilterTab attribute)}

\begin{fulllineitems}
\phantomsection\label{\detokenize{src:src.FilterTab.FilterTab.sharedDataAdderEnabledFlag}}\pysigline{\sphinxbfcode{sharedDataAdderEnabledFlag}\sphinxstrong{ = None}}
Shared process independent flag to terminate the data adder

\end{fulllineitems}

\index{sharedSnifferEnabledFlag (src.FilterTab.FilterTab attribute)}

\begin{fulllineitems}
\phantomsection\label{\detokenize{src:src.FilterTab.FilterTab.sharedSnifferEnabledFlag}}\pysigline{\sphinxbfcode{sharedSnifferEnabledFlag}\sphinxstrong{ = None}}
Shared process independent flag to terminate the sniffer process

\end{fulllineitems}

\index{startFilter() (src.FilterTab.FilterTab method)}

\begin{fulllineitems}
\phantomsection\label{\detokenize{src:src.FilterTab.FilterTab.startFilter}}\pysiglinewithargsret{\sphinxbfcode{startFilter}}{}{}~\begin{description}
\item[{Handle the filtering process:}] \leavevmode\begin{enumerate}
\item {} 
Collect noise

\item {} 
Record sample data

\item {} 
Analyze captured data

\end{enumerate}

\end{description}

\end{fulllineitems}

\index{startSnifferAndAdder() (src.FilterTab.FilterTab method)}

\begin{fulllineitems}
\phantomsection\label{\detokenize{src:src.FilterTab.FilterTab.startSnifferAndAdder}}\pysiglinewithargsret{\sphinxbfcode{startSnifferAndAdder}}{\emph{adderMethod}, \emph{curSampleIndex=-1}}{}
Start a DataAdderThread and a SnifferProcess to collect data. They will communicate using a
multiprocess pipe to collect data without interrupting the GUI thread.
\begin{quote}\begin{description}
\item[{Parameters}] \leavevmode\begin{itemize}
\item {} 
\sphinxstyleliteralstrong{adderMethod} \textendash{} The DataAdderThread will call this method to handle the received data

\item {} 
\sphinxstyleliteralstrong{curSampleIndex} \textendash{} The index of the currently captured sample (-1 as default)

\end{itemize}

\end{description}\end{quote}

\end{fulllineitems}

\index{stopSnifferAndAdder() (src.FilterTab.FilterTab method)}

\begin{fulllineitems}
\phantomsection\label{\detokenize{src:src.FilterTab.FilterTab.stopSnifferAndAdder}}\pysiglinewithargsret{\sphinxbfcode{stopSnifferAndAdder}}{}{}
Stop the DataAdderThread and SnifferProcess using the shared integer variable.

\end{fulllineitems}

\index{toggleGUIElements() (src.FilterTab.FilterTab method)}

\begin{fulllineitems}
\phantomsection\label{\detokenize{src:src.FilterTab.FilterTab.toggleGUIElements}}\pysiglinewithargsret{\sphinxbfcode{toggleGUIElements}}{\emph{state}}{}
\{En, Dis\}able all GUI elements that are used to change filter settings
\begin{quote}\begin{description}
\item[{Parameters}] \leavevmode
\sphinxstyleliteralstrong{state} \textendash{} Boolean value to indicate whether to enable or disable elements

\end{description}\end{quote}

\end{fulllineitems}

\index{updateNoiseCollectProgress() (src.FilterTab.FilterTab method)}

\begin{fulllineitems}
\phantomsection\label{\detokenize{src:src.FilterTab.FilterTab.updateNoiseCollectProgress}}\pysiglinewithargsret{\sphinxbfcode{updateNoiseCollectProgress}}{\emph{progressDialog}, \emph{value}}{}
Update the text and progressbar value displayed while collecting noise.
\begin{quote}\begin{description}
\item[{Parameters}] \leavevmode\begin{itemize}
\item {} 
\sphinxstyleliteralstrong{progressDialog} \textendash{} The QProgressDialog to update

\item {} 
\sphinxstyleliteralstrong{value} \textendash{} The value to set the progressbar to

\end{itemize}

\end{description}\end{quote}

\end{fulllineitems}


\end{fulllineitems}



\section{CANalyzat0r.FuzzerTab module}
\label{\detokenize{src:module-src.FuzzerTab}}\label{\detokenize{src:canalyzat0r-fuzzertab-module}}\index{src.FuzzerTab (module)}
Created on May 31, 2017

@author: pschmied
\index{FuzzerTab (class in src.FuzzerTab)}

\begin{fulllineitems}
\phantomsection\label{\detokenize{src:src.FuzzerTab.FuzzerTab}}\pysiglinewithargsret{\sphinxstrong{class }\sphinxcode{src.FuzzerTab.}\sphinxbfcode{FuzzerTab}}{\emph{tabWidget}}{}
Bases: {\hyperref[\detokenize{src:src.AbstractTab.AbstractTab}]{\sphinxcrossref{\sphinxcode{src.AbstractTab.AbstractTab}}}}

This class handles the logic of the fuzzer tab
\index{IDMask (src.FuzzerTab.FuzzerTab attribute)}

\begin{fulllineitems}
\phantomsection\label{\detokenize{src:src.FuzzerTab.FuzzerTab.IDMask}}\pysigline{\sphinxbfcode{IDMask}\sphinxstrong{ = None}}
The ID is 8 chars max. - initialize it with only X chars

\end{fulllineitems}

\index{IDMaskChanged() (src.FuzzerTab.FuzzerTab method)}

\begin{fulllineitems}
\phantomsection\label{\detokenize{src:src.FuzzerTab.FuzzerTab.IDMaskChanged}}\pysiglinewithargsret{\sphinxbfcode{IDMaskChanged}}{}{}
This allows changing the ID mask values on the fly because a new value
will only be set if the new value is valid.

\end{fulllineitems}

\index{IDMaxValue (src.FuzzerTab.FuzzerTab attribute)}

\begin{fulllineitems}
\phantomsection\label{\detokenize{src:src.FuzzerTab.FuzzerTab.IDMaxValue}}\pysigline{\sphinxbfcode{IDMaxValue}\sphinxstrong{ = None}}
Default: allow the max value of extended frames

\end{fulllineitems}

\index{\_\_init\_\_() (src.FuzzerTab.FuzzerTab method)}

\begin{fulllineitems}
\phantomsection\label{\detokenize{src:src.FuzzerTab.FuzzerTab.__init__}}\pysiglinewithargsret{\sphinxbfcode{\_\_init\_\_}}{\emph{tabWidget}}{}
\end{fulllineitems}

\index{addPacket() (src.FuzzerTab.FuzzerTab method)}

\begin{fulllineitems}
\phantomsection\label{\detokenize{src:src.FuzzerTab.FuzzerTab.addPacket}}\pysiglinewithargsret{\sphinxbfcode{addPacket}}{\emph{valueList}, \emph{addAtFront=True}, \emph{append=True}, \emph{emit=True}, \emph{addToRawDataOnly=False}}{}
Override the parents class method to add packets at front and to update the counter label

\end{fulllineitems}

\index{clear() (src.FuzzerTab.FuzzerTab method)}

\begin{fulllineitems}
\phantomsection\label{\detokenize{src:src.FuzzerTab.FuzzerTab.clear}}\pysiglinewithargsret{\sphinxbfcode{clear}}{\emph{returnOldPackets=False}}{}
Clear the currently displayed data on the GUI and in the rawData list
\begin{quote}\begin{description}
\item[{Parameters}] \leavevmode
\sphinxstyleliteralstrong{returnOldPackets} \textendash{} If this is true, then the previously displayed packets will
be returned as raw data list

\item[{Returns}] \leavevmode
Previously displayed packets as raw data list (if returnOldPackets is True), else an empty list

\end{description}\end{quote}

\end{fulllineitems}

\index{dataMask (src.FuzzerTab.FuzzerTab attribute)}

\begin{fulllineitems}
\phantomsection\label{\detokenize{src:src.FuzzerTab.FuzzerTab.dataMask}}\pysigline{\sphinxbfcode{dataMask}\sphinxstrong{ = None}}
The data is 16 chars max.

\end{fulllineitems}

\index{dataMaskChanged() (src.FuzzerTab.FuzzerTab method)}

\begin{fulllineitems}
\phantomsection\label{\detokenize{src:src.FuzzerTab.FuzzerTab.dataMaskChanged}}\pysiglinewithargsret{\sphinxbfcode{dataMaskChanged}}{}{}
This allows changing the data mask values on the fly because a new value
will only be set if the new value is valid.

\end{fulllineitems}

\index{dataMaxLength (src.FuzzerTab.FuzzerTab attribute)}

\begin{fulllineitems}
\phantomsection\label{\detokenize{src:src.FuzzerTab.FuzzerTab.dataMaxLength}}\pysigline{\sphinxbfcode{dataMaxLength}\sphinxstrong{ = None}}
This length corresponds the length when interpreted as bytes

\end{fulllineitems}

\index{fuzzSenderThread (src.FuzzerTab.FuzzerTab attribute)}

\begin{fulllineitems}
\phantomsection\label{\detokenize{src:src.FuzzerTab.FuzzerTab.fuzzSenderThread}}\pysigline{\sphinxbfcode{fuzzSenderThread}\sphinxstrong{ = None}}
Sending takes place in a loop in a separate thread

\end{fulllineitems}

\index{fuzzingModeChanged() (src.FuzzerTab.FuzzerTab method)}

\begin{fulllineitems}
\phantomsection\label{\detokenize{src:src.FuzzerTab.FuzzerTab.fuzzingModeChanged}}\pysiglinewithargsret{\sphinxbfcode{fuzzingModeChanged}}{}{}
This gets called if the ComboBox gets changed to update the active fuzzing mode.
The other GUI elements will be set and enabled depending on the selected mode.

\end{fulllineitems}

\index{fuzzingModeComboBoxValuePairs (src.FuzzerTab.FuzzerTab attribute)}

\begin{fulllineitems}
\phantomsection\label{\detokenize{src:src.FuzzerTab.FuzzerTab.fuzzingModeComboBoxValuePairs}}\pysigline{\sphinxbfcode{fuzzingModeComboBoxValuePairs}\sphinxstrong{ = None}}
These values will be available in the fuzzing mode ComboBox

\end{fulllineitems}

\index{generateRandomPacket() (src.FuzzerTab.FuzzerTab method)}

\begin{fulllineitems}
\phantomsection\label{\detokenize{src:src.FuzzerTab.FuzzerTab.generateRandomPacket}}\pysiglinewithargsret{\sphinxbfcode{generateRandomPacket}}{}{}
This generates a random pyvit Frame using \sphinxcode{tryBuildPacket()}
\begin{quote}\begin{description}
\item[{Returns}] \leavevmode
Pyvit frame with random data (random ID, data length and data)

\end{description}\end{quote}

\end{fulllineitems}

\index{itemAdderThread (src.FuzzerTab.FuzzerTab attribute)}

\begin{fulllineitems}
\phantomsection\label{\detokenize{src:src.FuzzerTab.FuzzerTab.itemAdderThread}}\pysigline{\sphinxbfcode{itemAdderThread}\sphinxstrong{ = None}}
Adding items also takes place in a separate thread to avoid blocking the GUI thread

\end{fulllineitems}

\index{packetBuildErrorCount (src.FuzzerTab.FuzzerTab attribute)}

\begin{fulllineitems}
\phantomsection\label{\detokenize{src:src.FuzzerTab.FuzzerTab.packetBuildErrorCount}}\pysigline{\sphinxbfcode{packetBuildErrorCount}\sphinxstrong{ = None}}
Used to avoid spamming the log box when the user specified wrong parameters while sending

\end{fulllineitems}

\index{prepareUI() (src.FuzzerTab.FuzzerTab method)}

\begin{fulllineitems}
\phantomsection\label{\detokenize{src:src.FuzzerTab.FuzzerTab.prepareUI}}\pysiglinewithargsret{\sphinxbfcode{prepareUI}}{}{}
\end{fulllineitems}

\index{sliderChanged() (src.FuzzerTab.FuzzerTab method)}

\begin{fulllineitems}
\phantomsection\label{\detokenize{src:src.FuzzerTab.FuzzerTab.sliderChanged}}\pysiglinewithargsret{\sphinxbfcode{sliderChanged}}{}{}
This method gets called if one of the two length sliders (min. and max. value) are changed.
\sphinxcode{dataMinLength} and \sphinxcode{dataMaxLength} will be directly updated and available
to a running FuzzerThread.

\end{fulllineitems}

\index{toggleFuzzing() (src.FuzzerTab.FuzzerTab method)}

\begin{fulllineitems}
\phantomsection\label{\detokenize{src:src.FuzzerTab.FuzzerTab.toggleFuzzing}}\pysiglinewithargsret{\sphinxbfcode{toggleFuzzing}}{}{}~\begin{description}
\item[{This starts and stops fuzzing.}] \leavevmode\begin{itemize}
\item {} 
Starting:
- Input values are read and validated
- ItemAdderThread and FuzzSenderThread (see {\hyperref[\detokenize{src:src.SenderThread.FuzzSenderThread}]{\sphinxcrossref{\sphinxcode{FuzzSenderThread}}}}) are started
- Some GUI elements will be disabled

\item {} 
Stopping:
- The threads will be disabled
- Disabled GUI elements will be enabled again

\end{itemize}

\end{description}

\end{fulllineitems}

\index{toggleGUIElements() (src.FuzzerTab.FuzzerTab method)}

\begin{fulllineitems}
\phantomsection\label{\detokenize{src:src.FuzzerTab.FuzzerTab.toggleGUIElements}}\pysiglinewithargsret{\sphinxbfcode{toggleGUIElements}}{\emph{state}}{}
\{En, Dis\}able all GUI elements that are used to change fuzzer settings
\begin{quote}\begin{description}
\item[{Parameters}] \leavevmode
\sphinxstyleliteralstrong{state} \textendash{} Boolean value to indicate whether to enable or disable elements

\end{description}\end{quote}

\end{fulllineitems}

\index{toggleLoopActive() (src.FuzzerTab.FuzzerTab method)}

\begin{fulllineitems}
\phantomsection\label{\detokenize{src:src.FuzzerTab.FuzzerTab.toggleLoopActive}}\pysiglinewithargsret{\sphinxbfcode{toggleLoopActive}}{}{}
If there is a FuzzerThread sending then the tab title will be red.

\end{fulllineitems}

\index{validateDataMaskInput() (src.FuzzerTab.FuzzerTab method)}

\begin{fulllineitems}
\phantomsection\label{\detokenize{src:src.FuzzerTab.FuzzerTab.validateDataMaskInput}}\pysiglinewithargsret{\sphinxbfcode{validateDataMaskInput}}{}{}~\begin{description}
\item[{Validates the user specified data mask:}] \leavevmode\begin{itemize}
\item {} 
The length must be \textless{}= 16

\item {} 
It must be a valid hex string

\item {} 
Value will be padded to 16 chars (8 bytes)

\end{itemize}

\end{description}
\begin{quote}\begin{description}
\item[{Returns}] \leavevmode
A validated data mask or None if it’s not possible to validate the input

\end{description}\end{quote}

\end{fulllineitems}

\index{validateIDMaskInput() (src.FuzzerTab.FuzzerTab method)}

\begin{fulllineitems}
\phantomsection\label{\detokenize{src:src.FuzzerTab.FuzzerTab.validateIDMaskInput}}\pysiglinewithargsret{\sphinxbfcode{validateIDMaskInput}}{}{}~\begin{description}
\item[{Validates the user specified ID mask:}] \leavevmode\begin{itemize}
\item {} 
The length must be either 3 or 8

\item {} 
It must be a valid hex string

\item {} 
Has to be \textless{} 0x1FFFFFFF which is the max. value for extended frames

\end{itemize}

\end{description}
\begin{quote}\begin{description}
\item[{Returns}] \leavevmode
A validated ID mask or None if it’s not possible to validate the input

\end{description}\end{quote}

\end{fulllineitems}


\end{fulllineitems}



\section{CANalyzat0r.Globals module}
\label{\detokenize{src:canalyzat0r-globals-module}}\label{\detokenize{src:module-src.Globals}}\index{src.Globals (module)}
Created on May 17, 2017

@author: pschmied
\index{CANData (in module src.Globals)}

\begin{fulllineitems}
\phantomsection\label{\detokenize{src:src.Globals.CANData}}\pysigline{\sphinxcode{src.Globals.}\sphinxbfcode{CANData}\sphinxstrong{ = None}}
Instance to interact with the bus

\end{fulllineitems}

\index{db (in module src.Globals)}

\begin{fulllineitems}
\phantomsection\label{\detokenize{src:src.Globals.db}}\pysigline{\sphinxcode{src.Globals.}\sphinxbfcode{db}\sphinxstrong{ = None}}
Object to handle db connections

\end{fulllineitems}

\index{knownPackets (in module src.Globals)}

\begin{fulllineitems}
\phantomsection\label{\detokenize{src:src.Globals.knownPackets}}\pysigline{\sphinxcode{src.Globals.}\sphinxbfcode{knownPackets}\sphinxstrong{ = \{\}}}
Stores all known packets for the current project
Key: CAN ID and data concatenated and separated with a “\#”
Value: Description

\end{fulllineitems}

\index{project (in module src.Globals)}

\begin{fulllineitems}
\phantomsection\label{\detokenize{src:src.Globals.project}}\pysigline{\sphinxcode{src.Globals.}\sphinxbfcode{project}\sphinxstrong{ = None}}
Manage the currently selected project

\end{fulllineitems}

\index{textBrowserLogs (in module src.Globals)}

\begin{fulllineitems}
\phantomsection\label{\detokenize{src:src.Globals.textBrowserLogs}}\pysigline{\sphinxcode{src.Globals.}\sphinxbfcode{textBrowserLogs}\sphinxstrong{ = None}}
Display logs in the GUI

\end{fulllineitems}

\index{ui (in module src.Globals)}

\begin{fulllineitems}
\phantomsection\label{\detokenize{src:src.Globals.ui}}\pysigline{\sphinxcode{src.Globals.}\sphinxbfcode{ui}\sphinxstrong{ = None}}
The general UI

\end{fulllineitems}



\section{CANalyzat0r.ItemAdderThread module}
\label{\detokenize{src:canalyzat0r-itemadderthread-module}}\label{\detokenize{src:module-src.ItemAdderThread}}\index{src.ItemAdderThread (module)}
Created on May 18, 2017

@author: pschmied
\index{ItemAdderThread (class in src.ItemAdderThread)}

\begin{fulllineitems}
\phantomsection\label{\detokenize{src:src.ItemAdderThread.ItemAdderThread}}\pysiglinewithargsret{\sphinxstrong{class }\sphinxcode{src.ItemAdderThread.}\sphinxbfcode{ItemAdderThread}}{\emph{receivePipe}, \emph{tableModel}, \emph{rawData}, \emph{useTimestamp=True}}{}
Bases: \sphinxcode{PySide.QtCore.QThread}

This thread receives data from a process and
emits a signal which causes the main thread to add the packets
to the table.
\index{\_\_init\_\_() (src.ItemAdderThread.ItemAdderThread method)}

\begin{fulllineitems}
\phantomsection\label{\detokenize{src:src.ItemAdderThread.ItemAdderThread.__init__}}\pysiglinewithargsret{\sphinxbfcode{\_\_init\_\_}}{\emph{receivePipe}, \emph{tableModel}, \emph{rawData}, \emph{useTimestamp=True}}{}
\end{fulllineitems}

\index{appendRow (src.ItemAdderThread.ItemAdderThread attribute)}

\begin{fulllineitems}
\phantomsection\label{\detokenize{src:src.ItemAdderThread.ItemAdderThread.appendRow}}\pysigline{\sphinxbfcode{appendRow}\sphinxstrong{ = \textless{}PySide.QtCore.Signal object\textgreater{}}}
Emit a signal to the main thread when items are ready to be added
Parameters: valueList

\end{fulllineitems}

\index{disable() (src.ItemAdderThread.ItemAdderThread method)}

\begin{fulllineitems}
\phantomsection\label{\detokenize{src:src.ItemAdderThread.ItemAdderThread.disable}}\pysiglinewithargsret{\sphinxbfcode{disable}}{}{}
This sets the enabled flag to False which causes the infinite loop in {\hyperref[\detokenize{src:src.ItemAdderThread.ItemAdderThread.run}]{\sphinxcrossref{\sphinxcode{run()}}}} to exit.

\end{fulllineitems}

\index{frameToRow() (src.ItemAdderThread.ItemAdderThread method)}

\begin{fulllineitems}
\phantomsection\label{\detokenize{src:src.ItemAdderThread.ItemAdderThread.frameToRow}}\pysiglinewithargsret{\sphinxbfcode{frameToRow}}{\emph{frame}}{}
Converts a pyvit frame to a raw value list. This list will be emitted using the signal
\sphinxcode{appendRow} along with the table data and rawData list.
\begin{quote}\begin{description}
\item[{Parameters}] \leavevmode
\sphinxstyleliteralstrong{frame} \textendash{} Pyvit CAN frame

\end{description}\end{quote}

\end{fulllineitems}

\index{run() (src.ItemAdderThread.ItemAdderThread method)}

\begin{fulllineitems}
\phantomsection\label{\detokenize{src:src.ItemAdderThread.ItemAdderThread.run}}\pysiglinewithargsret{\sphinxbfcode{run}}{}{}
As long as the thread is enabled: Receive a frame from the pipe and pass it to {\hyperref[\detokenize{src:src.ItemAdderThread.ItemAdderThread.frameToRow}]{\sphinxcrossref{\sphinxcode{frameToRow()}}}}.

\end{fulllineitems}

\index{staticMetaObject (src.ItemAdderThread.ItemAdderThread attribute)}

\begin{fulllineitems}
\phantomsection\label{\detokenize{src:src.ItemAdderThread.ItemAdderThread.staticMetaObject}}\pysigline{\sphinxbfcode{staticMetaObject}\sphinxstrong{ = \textless{}PySide.QtCore.QMetaObject object\textgreater{}}}
\end{fulllineitems}


\end{fulllineitems}



\section{CANalyzat0r.KnownPacket module}
\label{\detokenize{src:canalyzat0r-knownpacket-module}}\label{\detokenize{src:module-src.KnownPacket}}\index{src.KnownPacket (module)}
Created on May 22, 2017

@author: pschmied
\index{KnownPacket (class in src.KnownPacket)}

\begin{fulllineitems}
\phantomsection\label{\detokenize{src:src.KnownPacket.KnownPacket}}\pysiglinewithargsret{\sphinxstrong{class }\sphinxcode{src.KnownPacket.}\sphinxbfcode{KnownPacket}}{\emph{id}, \emph{projectID}, \emph{CANID}, \emph{data}, \emph{description}}{}
This class is being used to handle known packet data.
It’s more comfortable to use a object to pass data
than to use lists and list indexes. Please note that
this costs much more performance, so please use lists
if you have to deal with much data.
\index{\_\_init\_\_() (src.KnownPacket.KnownPacket method)}

\begin{fulllineitems}
\phantomsection\label{\detokenize{src:src.KnownPacket.KnownPacket.__init__}}\pysiglinewithargsret{\sphinxbfcode{\_\_init\_\_}}{\emph{id}, \emph{projectID}, \emph{CANID}, \emph{data}, \emph{description}}{}
\end{fulllineitems}

\index{fromJSON() (src.KnownPacket.KnownPacket static method)}

\begin{fulllineitems}
\phantomsection\label{\detokenize{src:src.KnownPacket.KnownPacket.fromJSON}}\pysiglinewithargsret{\sphinxstrong{static }\sphinxbfcode{fromJSON}}{\emph{importJSON}}{}
This class method creates a KnownPacket object using a JSON string.
\begin{quote}\begin{description}
\item[{Parameters}] \leavevmode
\sphinxstyleliteralstrong{importJSON} \textendash{} The JSON string containing the object data

\item[{Returns}] \leavevmode
A KnownPacket object with the values set accordingly

\end{description}\end{quote}

\end{fulllineitems}

\index{toComboBoxString() (src.KnownPacket.KnownPacket method)}

\begin{fulllineitems}
\phantomsection\label{\detokenize{src:src.KnownPacket.KnownPacket.toComboBoxString}}\pysiglinewithargsret{\sphinxbfcode{toComboBoxString}}{}{}
Calculate a string that will be displayed in a ComboBox
\begin{quote}\begin{description}
\item[{Returns}] \leavevmode
String representation of a KnownPacket object

\end{description}\end{quote}

\end{fulllineitems}

\index{toJSON() (src.KnownPacket.KnownPacket method)}

\begin{fulllineitems}
\phantomsection\label{\detokenize{src:src.KnownPacket.KnownPacket.toJSON}}\pysiglinewithargsret{\sphinxbfcode{toJSON}}{}{}
To export a KnownPacket, all data is being formatted as JSON.
The internal attribute \_\_dict\_\_ is being used to gather the data.
This data will then be formatted.
\begin{quote}\begin{description}
\item[{Returns}] \leavevmode
The formatted JSON data of the object

\end{description}\end{quote}

\end{fulllineitems}


\end{fulllineitems}



\section{CANalyzat0r.Logger module}
\label{\detokenize{src:module-src.Logger}}\label{\detokenize{src:canalyzat0r-logger-module}}\index{src.Logger (module)}
Created on May 17, 2017

@author: pschmied
\index{LogHandler (class in src.Logger)}

\begin{fulllineitems}
\phantomsection\label{\detokenize{src:src.Logger.LogHandler}}\pysigline{\sphinxstrong{class }\sphinxcode{src.Logger.}\sphinxbfcode{LogHandler}}
Bases: \sphinxcode{logging.Handler}

To manage different log levels and custom logging to the log box/text browser, this class is needed.
\index{\_\_init\_\_() (src.Logger.LogHandler method)}

\begin{fulllineitems}
\phantomsection\label{\detokenize{src:src.Logger.LogHandler.__init__}}\pysiglinewithargsret{\sphinxbfcode{\_\_init\_\_}}{}{}
\end{fulllineitems}

\index{emit() (src.Logger.LogHandler method)}

\begin{fulllineitems}
\phantomsection\label{\detokenize{src:src.Logger.LogHandler.emit}}\pysiglinewithargsret{\sphinxbfcode{emit}}{\emph{record}}{}
This checks to loglevel and also logs to log box/text browser.
\begin{quote}\begin{description}
\item[{Parameters}] \leavevmode
\sphinxstyleliteralstrong{record} \textendash{} The record to log

\end{description}\end{quote}

\end{fulllineitems}


\end{fulllineitems}

\index{Logger (class in src.Logger)}

\begin{fulllineitems}
\phantomsection\label{\detokenize{src:src.Logger.Logger}}\pysiglinewithargsret{\sphinxstrong{class }\sphinxcode{src.Logger.}\sphinxbfcode{Logger}}{\emph{className}}{}
Bases: \sphinxcode{object}

This class implements a simple logger with a formatter.
\index{\_\_init\_\_() (src.Logger.Logger method)}

\begin{fulllineitems}
\phantomsection\label{\detokenize{src:src.Logger.Logger.__init__}}\pysiglinewithargsret{\sphinxbfcode{\_\_init\_\_}}{\emph{className}}{}
Along with set statements, a formatter is being applied here.
\begin{quote}\begin{description}
\item[{Parameters}] \leavevmode
\sphinxstyleliteralstrong{className} \textendash{} The tag for the logger

\end{description}\end{quote}

\end{fulllineitems}

\index{getLogger() (src.Logger.Logger method)}

\begin{fulllineitems}
\phantomsection\label{\detokenize{src:src.Logger.Logger.getLogger}}\pysiglinewithargsret{\sphinxbfcode{getLogger}}{}{}
\end{fulllineitems}

\index{minLogLevel (src.Logger.Logger attribute)}

\begin{fulllineitems}
\phantomsection\label{\detokenize{src:src.Logger.Logger.minLogLevel}}\pysigline{\sphinxbfcode{minLogLevel}\sphinxstrong{ = 20}}
\end{fulllineitems}


\end{fulllineitems}



\section{CANalyzat0r.MainTab module}
\label{\detokenize{src:canalyzat0r-maintab-module}}\label{\detokenize{src:module-src.MainTab}}\index{src.MainTab (module)}
Created on May 22, 2017

@author: pschmied
\index{MainTab (class in src.MainTab)}

\begin{fulllineitems}
\phantomsection\label{\detokenize{src:src.MainTab.MainTab}}\pysigline{\sphinxstrong{class }\sphinxcode{src.MainTab.}\sphinxbfcode{MainTab}}
This class handles the logic of the main tab
\index{VCANCheckboxChanged() (src.MainTab.MainTab static method)}

\begin{fulllineitems}
\phantomsection\label{\detokenize{src:src.MainTab.MainTab.VCANCheckboxChanged}}\pysiglinewithargsret{\sphinxstrong{static }\sphinxbfcode{VCANCheckboxChanged}}{}{}
Clickhandler for the VCAN CheckBox which causes the SpinBox to be toggled.

\end{fulllineitems}

\index{addApplicationStatus() (src.MainTab.MainTab static method)}

\begin{fulllineitems}
\phantomsection\label{\detokenize{src:src.MainTab.MainTab.addApplicationStatus}}\pysiglinewithargsret{\sphinxstrong{static }\sphinxbfcode{addApplicationStatus}}{\emph{status}}{}
Add a new status to the status bar by name (ordered).
If no status is present, it will display \sphinxcode{Strings.statusBarReady}
\begin{quote}\begin{description}
\item[{Parameters}] \leavevmode
\sphinxstyleliteralstrong{status} \textendash{} The new status to add

\end{description}\end{quote}

\end{fulllineitems}

\index{addVCANInterface() (src.MainTab.MainTab static method)}

\begin{fulllineitems}
\phantomsection\label{\detokenize{src:src.MainTab.MainTab.addVCANInterface}}\pysiglinewithargsret{\sphinxstrong{static }\sphinxbfcode{addVCANInterface}}{}{}
Manually add a virtual CAN interface. This uses a syscall to \sphinxcode{ip link}.
If this call succeeds, a new CANDataInstance will be created using \sphinxcode{createCANDataInstance()}.
The detected CAN interfaces will be refreshed, too.

\end{fulllineitems}

\index{applyGlobalInterfaceSettings() (src.MainTab.MainTab static method)}

\begin{fulllineitems}
\phantomsection\label{\detokenize{src:src.MainTab.MainTab.applyGlobalInterfaceSettings}}\pysiglinewithargsret{\sphinxstrong{static }\sphinxbfcode{applyGlobalInterfaceSettings}}{}{}
Set the currently selected interface as the global interface.
Also, the bitrate will be updated and GUI elements will be toggled.
The CANData instances of all \sphinxstylestrong{inactive} tabs will also be set to the global interface.

\end{fulllineitems}

\index{applyLogLevelSetting() (src.MainTab.MainTab static method)}

\begin{fulllineitems}
\phantomsection\label{\detokenize{src:src.MainTab.MainTab.applyLogLevelSetting}}\pysiglinewithargsret{\sphinxstrong{static }\sphinxbfcode{applyLogLevelSetting}}{}{}
Set the minimum logging level to display messages for.

\end{fulllineitems}

\index{detectCANInterfaces() (src.MainTab.MainTab static method)}

\begin{fulllineitems}
\phantomsection\label{\detokenize{src:src.MainTab.MainTab.detectCANInterfaces}}\pysiglinewithargsret{\sphinxstrong{static }\sphinxbfcode{detectCANInterfaces}}{\emph{updateLabels=True}}{}
Detect CAN and VCAN interfaces available in the system. A syscall to \sphinxcode{/sys/class/net} is being used for this.
For every detected interface a new CANData instance will be
created using \sphinxcode{createCANDataInstance()}.

Also, interface labels and the global interface ComboBox will be updated.
\begin{quote}\begin{description}
\item[{Parameters}] \leavevmode
\sphinxstyleliteralstrong{updateLabels} \textendash{} Whether to update the interface labels or not

\end{description}\end{quote}

\end{fulllineitems}

\index{easterEgg() (src.MainTab.MainTab static method)}

\begin{fulllineitems}
\phantomsection\label{\detokenize{src:src.MainTab.MainTab.easterEgg}}\pysiglinewithargsret{\sphinxstrong{static }\sphinxbfcode{easterEgg}}{\emph{event}}{}
Nothing to see here
:return: fun

\end{fulllineitems}

\index{loadKernelModules() (src.MainTab.MainTab static method)}

\begin{fulllineitems}
\phantomsection\label{\detokenize{src:src.MainTab.MainTab.loadKernelModules}}\pysiglinewithargsret{\sphinxstrong{static }\sphinxbfcode{loadKernelModules}}{}{}
Load kernel modules to interact with CAN networks (\sphinxcode{can} and \sphinxcode{vcan}).

\end{fulllineitems}

\index{logger (src.MainTab.MainTab attribute)}

\begin{fulllineitems}
\phantomsection\label{\detokenize{src:src.MainTab.MainTab.logger}}\pysigline{\sphinxbfcode{logger}\sphinxstrong{ = \textless{}logging.Logger object\textgreater{}}}
The tab specific logger

\end{fulllineitems}

\index{populateProjects() (src.MainTab.MainTab static method)}

\begin{fulllineitems}
\phantomsection\label{\detokenize{src:src.MainTab.MainTab.populateProjects}}\pysiglinewithargsret{\sphinxstrong{static }\sphinxbfcode{populateProjects}}{\emph{keepCurrentIndex=False}}{}
This populates the project ComboBox in the main tab.
\begin{quote}\begin{description}
\item[{Parameters}] \leavevmode
\sphinxstyleliteralstrong{keepCurrentIndex} \textendash{} If this is set to True, the previously selected index will be re-selected in the end

\end{description}\end{quote}

\end{fulllineitems}

\index{prepareUI() (src.MainTab.MainTab static method)}

\begin{fulllineitems}
\phantomsection\label{\detokenize{src:src.MainTab.MainTab.prepareUI}}\pysiglinewithargsret{\sphinxstrong{static }\sphinxbfcode{prepareUI}}{}{}~\begin{enumerate}
\item {} 
Setup the status bar

\item {} 
Detect CAN interfaces and preselect the VCAN CheckBox

\item {} 
Populate project ComboBoxes

\item {} 
Add the logo

\end{enumerate}

\end{fulllineitems}

\index{preselectUseBitrateCheckBox() (src.MainTab.MainTab static method)}

\begin{fulllineitems}
\phantomsection\label{\detokenize{src:src.MainTab.MainTab.preselectUseBitrateCheckBox}}\pysiglinewithargsret{\sphinxstrong{static }\sphinxbfcode{preselectUseBitrateCheckBox}}{}{}
Preselect the VCAN CheckBox state because we can’t use the bitrate along with VCAN interfaces.

\end{fulllineitems}

\index{removeApplicationStatus() (src.MainTab.MainTab static method)}

\begin{fulllineitems}
\phantomsection\label{\detokenize{src:src.MainTab.MainTab.removeApplicationStatus}}\pysiglinewithargsret{\sphinxstrong{static }\sphinxbfcode{removeApplicationStatus}}{\emph{status}}{}
Remove a status from the status bar.
For statuses with mutiple possible values (e.g. \sphinxcode{Sending (X Threads)}
the search will be done using a substring search
\begin{quote}\begin{description}
\item[{Parameters}] \leavevmode
\sphinxstyleliteralstrong{status} \textendash{} The status to remove

\item[{Returns}] \leavevmode


\end{description}\end{quote}

\end{fulllineitems}

\index{removeVCANInterface() (src.MainTab.MainTab static method)}

\begin{fulllineitems}
\phantomsection\label{\detokenize{src:src.MainTab.MainTab.removeVCANInterface}}\pysiglinewithargsret{\sphinxstrong{static }\sphinxbfcode{removeVCANInterface}}{}{}
This removes the currently selected VCAN interface. This uses a syscall to \sphinxcode{ip link}.
If the removed interface was the current global interface, the global interface will become None.
:return:

\end{fulllineitems}

\index{setGlobalInterfaceStatus() (src.MainTab.MainTab static method)}

\begin{fulllineitems}
\phantomsection\label{\detokenize{src:src.MainTab.MainTab.setGlobalInterfaceStatus}}\pysiglinewithargsret{\sphinxstrong{static }\sphinxbfcode{setGlobalInterfaceStatus}}{\emph{red=False}}{}
Sets the text of the global interface status in the status bar.
If the global CANData instance is None then the text will read “None”.
\begin{quote}\begin{description}
\item[{Parameters}] \leavevmode
\sphinxstyleliteralstrong{red} \textendash{} Optional; If this is set to True, the text will appear red. Else black.

\end{description}\end{quote}

\end{fulllineitems}

\index{setProject() (src.MainTab.MainTab static method)}

\begin{fulllineitems}
\phantomsection\label{\detokenize{src:src.MainTab.MainTab.setProject}}\pysiglinewithargsret{\sphinxstrong{static }\sphinxbfcode{setProject}}{\emph{wasDeleted=False}, \emph{setNone=False}}{}
This sets the current project to the currently selected project in the corresponding ComboBox.
Also, the status bar and project specific ComboBoxes and GUI Elements will be updated.
\begin{quote}\begin{description}
\item[{Parameters}] \leavevmode\begin{itemize}
\item {} 
\sphinxstyleliteralstrong{wasDeleted} \textendash{} This is set to True if the current selected project was deleted. This causes
\sphinxcode{Globals.project} to become None, too.

\item {} 
\sphinxstyleliteralstrong{wasNull} \textendash{} This is set to True, if the project has to be set to None. Default: False

\end{itemize}

\end{description}\end{quote}

\end{fulllineitems}

\index{setProjectStatus() (src.MainTab.MainTab static method)}

\begin{fulllineitems}
\phantomsection\label{\detokenize{src:src.MainTab.MainTab.setProjectStatus}}\pysiglinewithargsret{\sphinxstrong{static }\sphinxbfcode{setProjectStatus}}{\emph{projectName}, \emph{red=False}}{}
Sets the text of the project status in the status bar.
\begin{quote}\begin{description}
\item[{Parameters}] \leavevmode\begin{itemize}
\item {} 
\sphinxstyleliteralstrong{projectName} \textendash{} The text to put as the new project name

\item {} 
\sphinxstyleliteralstrong{red} \textendash{} Optional; If this is set to True, the text will appear red. Else black.

\end{itemize}

\end{description}\end{quote}

\end{fulllineitems}

\index{setupStatusBar() (src.MainTab.MainTab static method)}

\begin{fulllineitems}
\phantomsection\label{\detokenize{src:src.MainTab.MainTab.setupStatusBar}}\pysiglinewithargsret{\sphinxstrong{static }\sphinxbfcode{setupStatusBar}}{}{}
Add labels to the status bar and prepare it.

\end{fulllineitems}

\index{statusBarActiveStatuses (src.MainTab.MainTab attribute)}

\begin{fulllineitems}
\phantomsection\label{\detokenize{src:src.MainTab.MainTab.statusBarActiveStatuses}}\pysigline{\sphinxbfcode{statusBarActiveStatuses}\sphinxstrong{ = {[}{]}}}
These text will appear in the status bar

\end{fulllineitems}

\index{statusBarApplicationStatus (src.MainTab.MainTab attribute)}

\begin{fulllineitems}
\phantomsection\label{\detokenize{src:src.MainTab.MainTab.statusBarApplicationStatus}}\pysigline{\sphinxbfcode{statusBarApplicationStatus}\sphinxstrong{ = None}}
Statusbar labels

\end{fulllineitems}

\index{statusBarInterface (src.MainTab.MainTab attribute)}

\begin{fulllineitems}
\phantomsection\label{\detokenize{src:src.MainTab.MainTab.statusBarInterface}}\pysigline{\sphinxbfcode{statusBarInterface}\sphinxstrong{ = None}}
\end{fulllineitems}

\index{statusBarProject (src.MainTab.MainTab attribute)}

\begin{fulllineitems}
\phantomsection\label{\detokenize{src:src.MainTab.MainTab.statusBarProject}}\pysigline{\sphinxbfcode{statusBarProject}\sphinxstrong{ = None}}
\end{fulllineitems}

\index{updateVCANButtons() (src.MainTab.MainTab static method)}

\begin{fulllineitems}
\phantomsection\label{\detokenize{src:src.MainTab.MainTab.updateVCANButtons}}\pysiglinewithargsret{\sphinxstrong{static }\sphinxbfcode{updateVCANButtons}}{}{}
Update the text of the buttons to add and remove VCAN interfaces.

\end{fulllineitems}


\end{fulllineitems}



\section{CANalyzat0r.ManagerTab module}
\label{\detokenize{src:canalyzat0r-managertab-module}}\label{\detokenize{src:module-src.ManagerTab}}\index{src.ManagerTab (module)}
Created on May 22, 2017

@author: pschmied
\index{ManagerTab (class in src.ManagerTab)}

\begin{fulllineitems}
\phantomsection\label{\detokenize{src:src.ManagerTab.ManagerTab}}\pysiglinewithargsret{\sphinxstrong{class }\sphinxcode{src.ManagerTab.}\sphinxbfcode{ManagerTab}}{\emph{tabWidget}}{}
Bases: {\hyperref[\detokenize{src:src.AbstractTab.AbstractTab}]{\sphinxcrossref{\sphinxcode{src.AbstractTab.AbstractTab}}}}

This class handles the logic of the manager tab
\index{\_\_init\_\_() (src.ManagerTab.ManagerTab method)}

\begin{fulllineitems}
\phantomsection\label{\detokenize{src:src.ManagerTab.ManagerTab.__init__}}\pysiglinewithargsret{\sphinxbfcode{\_\_init\_\_}}{\emph{tabWidget}}{}
\end{fulllineitems}

\index{addKnownPacket() (src.ManagerTab.ManagerTab method)}

\begin{fulllineitems}
\phantomsection\label{\detokenize{src:src.ManagerTab.ManagerTab.addKnownPacket}}\pysiglinewithargsret{\sphinxbfcode{addKnownPacket}}{\emph{CANID=None}, \emph{data=None}, \emph{description=None}}{}
Save a known packet to the current project (and database)
Default: Get values from the GUI elements. But you can also specify the values
using the optional parameters.
\begin{quote}\begin{description}
\item[{Parameters}] \leavevmode\begin{itemize}
\item {} 
\sphinxstyleliteralstrong{CANID} \textendash{} Optional: CAN ID

\item {} 
\sphinxstyleliteralstrong{data} \textendash{} Optional: Payload data

\item {} 
\sphinxstyleliteralstrong{description} \textendash{} Optional: Description of the known packet

\end{itemize}

\item[{Returns}] \leavevmode


\end{description}\end{quote}

\end{fulllineitems}

\index{clear() (src.ManagerTab.ManagerTab method)}

\begin{fulllineitems}
\phantomsection\label{\detokenize{src:src.ManagerTab.ManagerTab.clear}}\pysiglinewithargsret{\sphinxbfcode{clear}}{\emph{returnOldPackets=False}}{}
Clear the GUI table displaying PacketSets along with data lists.

\end{fulllineitems}

\index{createDump() (src.ManagerTab.ManagerTab method)}

\begin{fulllineitems}
\phantomsection\label{\detokenize{src:src.ManagerTab.ManagerTab.createDump}}\pysiglinewithargsret{\sphinxbfcode{createDump}}{\emph{rawPackets=None}}{}
Save a new dump to the database. This creates a new PacketSet along with associated Packet objects.
If only one packet is saved, the user will be asked if he wants to create a known packet entry for the
just saved packet.
\begin{quote}\begin{description}
\item[{Parameters}] \leavevmode
\sphinxstyleliteralstrong{rawPackets} \textendash{} Optional: If this is not None, the values from \sphinxcode{rawPackets} will be used instead of
the data that is currently being displayed in the GUI table.

\end{description}\end{quote}

\end{fulllineitems}

\index{createProject() (src.ManagerTab.ManagerTab method)}

\begin{fulllineitems}
\phantomsection\label{\detokenize{src:src.ManagerTab.ManagerTab.createProject}}\pysiglinewithargsret{\sphinxbfcode{createProject}}{}{}
Create a new project and save it to the database. This also updates the project ComboBoxes.

\end{fulllineitems}

\index{deleteDump() (src.ManagerTab.ManagerTab method)}

\begin{fulllineitems}
\phantomsection\label{\detokenize{src:src.ManagerTab.ManagerTab.deleteDump}}\pysiglinewithargsret{\sphinxbfcode{deleteDump}}{}{}
Delete the currently selected PacketSet from the database. This also re-populates the table with the data
of another dump (if existing)

\end{fulllineitems}

\index{deleteProject() (src.ManagerTab.ManagerTab method)}

\begin{fulllineitems}
\phantomsection\label{\detokenize{src:src.ManagerTab.ManagerTab.deleteProject}}\pysiglinewithargsret{\sphinxbfcode{deleteProject}}{}{}
Delete a project along with associated data. This also updates the project ComboBoxes.

\end{fulllineitems}

\index{dumpsRowIDs (src.ManagerTab.ManagerTab attribute)}

\begin{fulllineitems}
\phantomsection\label{\detokenize{src:src.ManagerTab.ManagerTab.dumpsRowIDs}}\pysigline{\sphinxbfcode{dumpsRowIDs}\sphinxstrong{ = None}}
Kepps track between the association of
table row \textless{}-\textgreater{} database id of the packet
e.g. row 2 - database ID 5

\end{fulllineitems}

\index{editKnownPacket() (src.ManagerTab.ManagerTab method)}

\begin{fulllineitems}
\phantomsection\label{\detokenize{src:src.ManagerTab.ManagerTab.editKnownPacket}}\pysiglinewithargsret{\sphinxbfcode{editKnownPacket}}{}{}
Update a known packet with new specified values.

\end{fulllineitems}

\index{editProject() (src.ManagerTab.ManagerTab method)}

\begin{fulllineitems}
\phantomsection\label{\detokenize{src:src.ManagerTab.ManagerTab.editProject}}\pysiglinewithargsret{\sphinxbfcode{editProject}}{}{}
Update a project with new specified values.

\end{fulllineitems}

\index{exportProject() (src.ManagerTab.ManagerTab method)}

\begin{fulllineitems}
\phantomsection\label{\detokenize{src:src.ManagerTab.ManagerTab.exportProject}}\pysiglinewithargsret{\sphinxbfcode{exportProject}}{}{}
Export a project as JSON string to a textfile.
The \sphinxcode{toJSON()} method is called for every object to be exported.

\end{fulllineitems}

\index{getDump() (src.ManagerTab.ManagerTab method)}

\begin{fulllineitems}
\phantomsection\label{\detokenize{src:src.ManagerTab.ManagerTab.getDump}}\pysiglinewithargsret{\sphinxbfcode{getDump}}{}{}
Display the data of the selected PacketSet in the GUI table. This also updates \sphinxcode{rawData}

\end{fulllineitems}

\index{getKnownPacketsForCurrentProject() (src.ManagerTab.ManagerTab method)}

\begin{fulllineitems}
\phantomsection\label{\detokenize{src:src.ManagerTab.ManagerTab.getKnownPacketsForCurrentProject}}\pysiglinewithargsret{\sphinxbfcode{getKnownPacketsForCurrentProject}}{}{}
(Re-)Populate the dictionary \sphinxcode{Globals.knownPackets} with up-to-date data.
If no project is set, the dictionary will be cleared only.

\end{fulllineitems}

\index{handleCopy() (src.ManagerTab.ManagerTab method)}

\begin{fulllineitems}
\phantomsection\label{\detokenize{src:src.ManagerTab.ManagerTab.handleCopy}}\pysiglinewithargsret{\sphinxbfcode{handleCopy}}{}{}
Pass the copy event to the Toolbox, but only if no data is being loaded

\end{fulllineitems}

\index{handlePaste() (src.ManagerTab.ManagerTab method)}

\begin{fulllineitems}
\phantomsection\label{\detokenize{src:src.ManagerTab.ManagerTab.handlePaste}}\pysiglinewithargsret{\sphinxbfcode{handlePaste}}{}{}
Pass the paste event to the Toolbox, but only if no data is being loaded

\end{fulllineitems}

\index{importProject() (src.ManagerTab.ManagerTab method)}

\begin{fulllineitems}
\phantomsection\label{\detokenize{src:src.ManagerTab.ManagerTab.importProject}}\pysiglinewithargsret{\sphinxbfcode{importProject}}{}{}
Import a project from a JSON file.
The \sphinxcode{fromJSON()} method of every class is called to re-create objects.

\end{fulllineitems}

\index{loadingData (src.ManagerTab.ManagerTab attribute)}

\begin{fulllineitems}
\phantomsection\label{\detokenize{src:src.ManagerTab.ManagerTab.loadingData}}\pysigline{\sphinxbfcode{loadingData}\sphinxstrong{ = None}}
Disallow copying while loading data

\end{fulllineitems}

\index{populateKnownPacketEditLineEdits() (src.ManagerTab.ManagerTab method)}

\begin{fulllineitems}
\phantomsection\label{\detokenize{src:src.ManagerTab.ManagerTab.populateKnownPacketEditLineEdits}}\pysiglinewithargsret{\sphinxbfcode{populateKnownPacketEditLineEdits}}{}{}
Sets the GUI elements concerning editing known packets to the current selected known packet data.

\end{fulllineitems}

\index{populateKnownPackets() (src.ManagerTab.ManagerTab method)}

\begin{fulllineitems}
\phantomsection\label{\detokenize{src:src.ManagerTab.ManagerTab.populateKnownPackets}}\pysiglinewithargsret{\sphinxbfcode{populateKnownPackets}}{\emph{keepCurrentIndex=False}}{}
Populate the known packet ComboBoxex (delete and edit).
\begin{quote}\begin{description}
\item[{Parameters}] \leavevmode
\sphinxstyleliteralstrong{keepCurrentIndex} \textendash{} Optional: Reselect a specific KnownPacket in the ComboBox

\end{description}\end{quote}

\end{fulllineitems}

\index{populatePacketSets() (src.ManagerTab.ManagerTab method)}

\begin{fulllineitems}
\phantomsection\label{\detokenize{src:src.ManagerTab.ManagerTab.populatePacketSets}}\pysiglinewithargsret{\sphinxbfcode{populatePacketSets}}{\emph{IDtoChoose=None}}{}
Populate the dumps ComboBox in the dumps tab.
Reloading dump data is being handled by the triggered event
\begin{quote}\begin{description}
\item[{Parameters}] \leavevmode
\sphinxstyleliteralstrong{IDtoChoose} \textendash{} Optional: Preselect a specific PacketSet in the ComboBox

\end{description}\end{quote}

\end{fulllineitems}

\index{populateProjectEditLineEdits() (src.ManagerTab.ManagerTab method)}

\begin{fulllineitems}
\phantomsection\label{\detokenize{src:src.ManagerTab.ManagerTab.populateProjectEditLineEdits}}\pysiglinewithargsret{\sphinxbfcode{populateProjectEditLineEdits}}{}{}
Sets the GUI elements concerning editing projects to the current selected project data.

\end{fulllineitems}

\index{populateProjects() (src.ManagerTab.ManagerTab method)}

\begin{fulllineitems}
\phantomsection\label{\detokenize{src:src.ManagerTab.ManagerTab.populateProjects}}\pysiglinewithargsret{\sphinxbfcode{populateProjects}}{\emph{keepCurrentIndex=False}}{}
Populate the project ComboBoxes (delete, Edit, export project).
\begin{quote}\begin{description}
\item[{Parameters}] \leavevmode
\sphinxstyleliteralstrong{keepCurrentIndex} \textendash{} If this is set to True, the previously selected index will be reselected

\end{description}\end{quote}

\end{fulllineitems}

\index{prepareUI() (src.ManagerTab.ManagerTab method)}

\begin{fulllineitems}
\phantomsection\label{\detokenize{src:src.ManagerTab.ManagerTab.prepareUI}}\pysiglinewithargsret{\sphinxbfcode{prepareUI}}{}{}
Prepare the tab specific GUI elements, add keyboard shortcuts and set a CANData instance

\end{fulllineitems}

\index{removeKnownPacket() (src.ManagerTab.ManagerTab method)}

\begin{fulllineitems}
\phantomsection\label{\detokenize{src:src.ManagerTab.ManagerTab.removeKnownPacket}}\pysiglinewithargsret{\sphinxbfcode{removeKnownPacket}}{}{}
Remove a known packet from the current project (and the database)

\end{fulllineitems}

\index{removeSelectedPackets() (src.ManagerTab.ManagerTab method)}

\begin{fulllineitems}
\phantomsection\label{\detokenize{src:src.ManagerTab.ManagerTab.removeSelectedPackets}}\pysiglinewithargsret{\sphinxbfcode{removeSelectedPackets}}{}{}
Pass the remove requested event to the super class.
After that, add the \sphinxstylestrong{database} IDs of the deleted packets to \sphinxcode{dumpsDeletedPacketIDs}.
The deleted rows will be removed from \sphinxcode{dumpsRowIDs}, too.

\end{fulllineitems}

\index{saveToFile() (src.ManagerTab.ManagerTab method)}

\begin{fulllineitems}
\phantomsection\label{\detokenize{src:src.ManagerTab.ManagerTab.saveToFile}}\pysiglinewithargsret{\sphinxbfcode{saveToFile}}{}{}
Save the packets in the GUI table to a file in SocketCAN format.

\end{fulllineitems}

\index{updateDump() (src.ManagerTab.ManagerTab method)}

\begin{fulllineitems}
\phantomsection\label{\detokenize{src:src.ManagerTab.ManagerTab.updateDump}}\pysiglinewithargsret{\sphinxbfcode{updateDump}}{}{}
Users can change the data displayed in the GUI table. This method allows the changed data
to be saved to the database.

\end{fulllineitems}


\end{fulllineitems}



\section{CANalyzat0r.Packet module}
\label{\detokenize{src:module-src.Packet}}\label{\detokenize{src:canalyzat0r-packet-module}}\index{src.Packet (module)}
Created on May 19, 2017

@author: pschmied
\index{Packet (class in src.Packet)}

\begin{fulllineitems}
\phantomsection\label{\detokenize{src:src.Packet.Packet}}\pysiglinewithargsret{\sphinxstrong{class }\sphinxcode{src.Packet.}\sphinxbfcode{Packet}}{\emph{packetSetID}, \emph{CANID}, \emph{data}, \emph{timestamp=”}, \emph{iface=”}, \emph{length=None}, \emph{id=None}}{}
This class is being used to handle packet data.
It’s more comfortable to use a object to pass data
than to use lists and list indexes. Please note that
this costs much more performance, so please use lists
if you have to deal with much data.
\index{\_\_init\_\_() (src.Packet.Packet method)}

\begin{fulllineitems}
\phantomsection\label{\detokenize{src:src.Packet.Packet.__init__}}\pysiglinewithargsret{\sphinxbfcode{\_\_init\_\_}}{\emph{packetSetID}, \emph{CANID}, \emph{data}, \emph{timestamp=”}, \emph{iface=”}, \emph{length=None}, \emph{id=None}}{}
The parameters \sphinxcode{CANID} and \sphinxcode{data} must be valid hex strings.
If length is not specified, it will be calculated automatically.

\end{fulllineitems}

\index{fromJSON() (src.Packet.Packet static method)}

\begin{fulllineitems}
\phantomsection\label{\detokenize{src:src.Packet.Packet.fromJSON}}\pysiglinewithargsret{\sphinxstrong{static }\sphinxbfcode{fromJSON}}{\emph{importJSON}}{}
This class method creates a Packet object using a JSON string.
\begin{quote}\begin{description}
\item[{Parameters}] \leavevmode
\sphinxstyleliteralstrong{importJSON} \textendash{} The JSON string containing the object data

\item[{Returns}] \leavevmode
A packet object with the values set accordingly

\end{description}\end{quote}

\end{fulllineitems}

\index{getDisplayDataLength() (src.Packet.Packet static method)}

\begin{fulllineitems}
\phantomsection\label{\detokenize{src:src.Packet.Packet.getDisplayDataLength}}\pysiglinewithargsret{\sphinxstrong{static }\sphinxbfcode{getDisplayDataLength}}{\emph{CANID}, \emph{hexData}}{}
This makes sure that the displayed length is correct.
If CANID is empty, the length will also be empty.
If the length if hexData is odd, the length will read “INVALID” which prevents saving to the database.
Else the length will be the amount of chars / 2
\begin{quote}\begin{description}
\item[{Parameters}] \leavevmode\begin{itemize}
\item {} 
\sphinxstyleliteralstrong{CANID} \textendash{} CAN ID

\item {} 
\sphinxstyleliteralstrong{hexData} \textendash{} Payload data as hex string

\end{itemize}

\item[{Returns}] \leavevmode
The correct length as string

\end{description}\end{quote}

\end{fulllineitems}

\index{lengthStringToInt() (src.Packet.Packet method)}

\begin{fulllineitems}
\phantomsection\label{\detokenize{src:src.Packet.Packet.lengthStringToInt}}\pysiglinewithargsret{\sphinxbfcode{lengthStringToInt}}{\emph{string}}{}
This makes sure that the specified length is an int
:return: The length as integer (if possible) - else None by exception)
:raises ValueError if the length isn’t an integer

\end{fulllineitems}

\index{toJSON() (src.Packet.Packet method)}

\begin{fulllineitems}
\phantomsection\label{\detokenize{src:src.Packet.Packet.toJSON}}\pysiglinewithargsret{\sphinxbfcode{toJSON}}{}{}
To export a Packet, all data is being formatted as JSON.
The internal attribute \_\_dict\_\_ is being used to gather the data.
This data will then be formatted.
\begin{quote}\begin{description}
\item[{Returns}] \leavevmode
The formatted JSON data of the object

\end{description}\end{quote}

\end{fulllineitems}


\end{fulllineitems}



\section{CANalyzat0r.PacketsDialog module}
\label{\detokenize{src:module-src.PacketsDialog}}\label{\detokenize{src:canalyzat0r-packetsdialog-module}}\index{src.PacketsDialog (module)}
Created on Jun 23, 2017

@author: pschmied
\index{PacketsDialog (class in src.PacketsDialog)}

\begin{fulllineitems}
\phantomsection\label{\detokenize{src:src.PacketsDialog.PacketsDialog}}\pysiglinewithargsret{\sphinxstrong{class }\sphinxcode{src.PacketsDialog.}\sphinxbfcode{PacketsDialog}}{\emph{packets=None}, \emph{rawPacketList=None}, \emph{returnPacketsAsRawList=True}}{}
Bases: {\hyperref[\detokenize{src:src.AbstractTab.AbstractTab}]{\sphinxcrossref{\sphinxcode{src.AbstractTab.AbstractTab}}}}

This class handles the logic of the “manage packets” dialog. For example, this can be found in the
{\hyperref[\detokenize{src:src.SnifferTabElement.SnifferTabElement}]{\sphinxcrossref{\sphinxcode{SnifferTabElement}}}}.
\index{\_\_init\_\_() (src.PacketsDialog.PacketsDialog method)}

\begin{fulllineitems}
\phantomsection\label{\detokenize{src:src.PacketsDialog.PacketsDialog.__init__}}\pysiglinewithargsret{\sphinxbfcode{\_\_init\_\_}}{\emph{packets=None}, \emph{rawPacketList=None}, \emph{returnPacketsAsRawList=True}}{}
This basically just sets data and reads the widget from the \sphinxcode{.ui} file.
\begin{quote}\begin{description}
\item[{Parameters}] \leavevmode\begin{itemize}
\item {} 
\sphinxstyleliteralstrong{packets} \textendash{} Optional: List that contains the elements that will be pre loaded into the GUI table
in the following format: \sphinxcode{\textless{}CAN ID\textgreater{}\#\textless{}Data\textgreater{}}.
This is used for the {\hyperref[\detokenize{src:src.SnifferTabElement.SnifferTabElement}]{\sphinxcrossref{\sphinxcode{SnifferTabElement}}}}

\item {} 
\sphinxstyleliteralstrong{rawPacketList} \textendash{} Optional: Raw packet list that contains the elements that will be pre loaded into the GUI table.
If this is specified, \sphinxcode{packets} will be ignored.

\item {} 
\sphinxstyleliteralstrong{returnPacketsAsRawList} \textendash{} Boolean value indicating whether the displayed packets will be returned
as raw packet list. If this is False, the values will be returned as list
in the following format: \sphinxcode{\textless{}CAN ID\textgreater{}\#\textless{}Data\textgreater{}}.

\end{itemize}

\end{description}\end{quote}

\end{fulllineitems}

\index{displayUniquePackets() (src.PacketsDialog.PacketsDialog method)}

\begin{fulllineitems}
\phantomsection\label{\detokenize{src:src.PacketsDialog.PacketsDialog.displayUniquePackets}}\pysiglinewithargsret{\sphinxbfcode{displayUniquePackets}}{\emph{IDOnly=False}}{}
Filter the currently displayed data for unique packets and display them on the table.
:param IDOnly: If this is True, only the ID will be matched to compare data. This allows wildcard ignores.

\end{fulllineitems}

\index{getUniqueIDs() (src.PacketsDialog.PacketsDialog method)}

\begin{fulllineitems}
\phantomsection\label{\detokenize{src:src.PacketsDialog.PacketsDialog.getUniqueIDs}}\pysiglinewithargsret{\sphinxbfcode{getUniqueIDs}}{}{}
Filters all unique IDs out of \sphinxcode{rawData} and displays them on the GUI table.
Unique ID means that the data column will be ignored and left blank for wildcard ignores.
This uses {\hyperref[\detokenize{src:src.PacketsDialog.PacketsDialog.displayUniquePackets}]{\sphinxcrossref{\sphinxcode{displayUniquePackets()}}}}.

\end{fulllineitems}

\index{getUniquePackets() (src.PacketsDialog.PacketsDialog method)}

\begin{fulllineitems}
\phantomsection\label{\detokenize{src:src.PacketsDialog.PacketsDialog.getUniquePackets}}\pysiglinewithargsret{\sphinxbfcode{getUniquePackets}}{}{}
Filters all unique packets out of \sphinxcode{rawData} and displays them on the GUI table.
This uses {\hyperref[\detokenize{src:src.PacketsDialog.PacketsDialog.displayUniquePackets}]{\sphinxcrossref{\sphinxcode{displayUniquePackets()}}}}.

\end{fulllineitems}

\index{getUniqueRawPackets() (src.PacketsDialog.PacketsDialog static method)}

\begin{fulllineitems}
\phantomsection\label{\detokenize{src:src.PacketsDialog.PacketsDialog.getUniqueRawPackets}}\pysiglinewithargsret{\sphinxstrong{static }\sphinxbfcode{getUniqueRawPackets}}{\emph{rawPacketList}}{}
Helper method to extract unique raw packets out of a given raw packet list
which has been cleaned before (Only necessary data fields are present, all others are
set to an empty string)
\begin{quote}\begin{description}
\item[{Parameters}] \leavevmode
\sphinxstyleliteralstrong{rawPacketList} \textendash{} The cleaned list of raw packets

\item[{Returns}] \leavevmode
A list of unique raw packet lists

\end{description}\end{quote}

\end{fulllineitems}

\index{open() (src.PacketsDialog.PacketsDialog method)}

\begin{fulllineitems}
\phantomsection\label{\detokenize{src:src.PacketsDialog.PacketsDialog.open}}\pysiglinewithargsret{\sphinxbfcode{open}}{}{}
Show the widget, extract data and return it
\begin{quote}\begin{description}
\item[{Returns}] \leavevmode
Raw packet list if the user didn’t press Cancel and if \sphinxcode{returnPacketsAsRawList} is True.
Else: list of values of the following form: \sphinxcode{\textless{}CAN ID\textgreater{}\#\textless{}Data\textgreater{}} if the user didn’t press Cancel.
Else None.

\end{description}\end{quote}

\end{fulllineitems}

\index{prepareUI() (src.PacketsDialog.PacketsDialog method)}

\begin{fulllineitems}
\phantomsection\label{\detokenize{src:src.PacketsDialog.PacketsDialog.prepareUI}}\pysiglinewithargsret{\sphinxbfcode{prepareUI}}{}{}
Prepare the GUI elements and add keyboard shortcuts. Also, pre populate the table

\end{fulllineitems}


\end{fulllineitems}



\section{CANalyzat0r.PacketSet module}
\label{\detokenize{src:canalyzat0r-packetset-module}}\label{\detokenize{src:module-src.PacketSet}}\index{src.PacketSet (module)}
Created on May 19, 2017

@author: pschmied
\index{PacketSet (class in src.PacketSet)}

\begin{fulllineitems}
\phantomsection\label{\detokenize{src:src.PacketSet.PacketSet}}\pysiglinewithargsret{\sphinxstrong{class }\sphinxcode{src.PacketSet.}\sphinxbfcode{PacketSet}}{\emph{id}, \emph{projectID}, \emph{name}, \emph{date=None}}{}
This class is being used to handle packet set data.
It’s more comfortable to use a object to pass data
than to use lists and list indexes. Please note that
this costs much more performance, so please use lists
if you have to deal with much data.
\index{\_\_init\_\_() (src.PacketSet.PacketSet method)}

\begin{fulllineitems}
\phantomsection\label{\detokenize{src:src.PacketSet.PacketSet.__init__}}\pysiglinewithargsret{\sphinxbfcode{\_\_init\_\_}}{\emph{id}, \emph{projectID}, \emph{name}, \emph{date=None}}{}
The date of a PacketSet will be automatically set to the current date as string

\end{fulllineitems}

\index{fromJSON() (src.PacketSet.PacketSet static method)}

\begin{fulllineitems}
\phantomsection\label{\detokenize{src:src.PacketSet.PacketSet.fromJSON}}\pysiglinewithargsret{\sphinxstrong{static }\sphinxbfcode{fromJSON}}{\emph{importJSON}}{}
This class method creates a PacketSet object using a JSON string.
\begin{quote}\begin{description}
\item[{Parameters}] \leavevmode
\sphinxstyleliteralstrong{importJSON} \textendash{} The JSON string containing the object data

\item[{Returns}] \leavevmode
A PacketSet object with the values set accordingly

\end{description}\end{quote}

\end{fulllineitems}

\index{toComboBoxString() (src.PacketSet.PacketSet method)}

\begin{fulllineitems}
\phantomsection\label{\detokenize{src:src.PacketSet.PacketSet.toComboBoxString}}\pysiglinewithargsret{\sphinxbfcode{toComboBoxString}}{}{}
Calculate a string that will be displayed in a ComboBox
\begin{quote}\begin{description}
\item[{Returns}] \leavevmode
String representation of a PacketSet object

\end{description}\end{quote}

\end{fulllineitems}

\index{toJSON() (src.PacketSet.PacketSet method)}

\begin{fulllineitems}
\phantomsection\label{\detokenize{src:src.PacketSet.PacketSet.toJSON}}\pysiglinewithargsret{\sphinxbfcode{toJSON}}{}{}
To export a PacketSet, all data is being formatted as JSON.
The internal attribute \_\_dict\_\_ is being used to gather the data.
This data will then be formatted.
\begin{quote}\begin{description}
\item[{Returns}] \leavevmode
The formatted JSON data of the object

\end{description}\end{quote}

\end{fulllineitems}


\end{fulllineitems}



\section{CANalyzat0r.PacketTableModel module}
\label{\detokenize{src:module-src.PacketTableModel}}\label{\detokenize{src:canalyzat0r-packettablemodel-module}}\index{src.PacketTableModel (module)}
Created on May 31, 2017

@author: pschmied
\index{PacketTableModel (class in src.PacketTableModel)}

\begin{fulllineitems}
\phantomsection\label{\detokenize{src:src.PacketTableModel.PacketTableModel}}\pysiglinewithargsret{\sphinxstrong{class }\sphinxcode{src.PacketTableModel.}\sphinxbfcode{PacketTableModel}}{\emph{parent}, \emph{dataList}, \emph{header}, \emph{readOnlyCols}, \emph{IDColIndex=1}, \emph{dataColIndex=2}, \emph{lengthColIndex=3}, \emph{timestampColIndex=4}, \emph{descriptionColIndex=5}, \emph{*args}}{}
Bases: \sphinxcode{PySide.QtCore.QAbstractTableModel}, \sphinxcode{PySide.QtCore.QObject}

A custom TableModel is needed to allow efficient handling of \sphinxstylestrong{many} values.
\index{\_\_init\_\_() (src.PacketTableModel.PacketTableModel method)}

\begin{fulllineitems}
\phantomsection\label{\detokenize{src:src.PacketTableModel.PacketTableModel.__init__}}\pysiglinewithargsret{\sphinxbfcode{\_\_init\_\_}}{\emph{parent}, \emph{dataList}, \emph{header}, \emph{readOnlyCols}, \emph{IDColIndex=1}, \emph{dataColIndex=2}, \emph{lengthColIndex=3}, \emph{timestampColIndex=4}, \emph{descriptionColIndex=5}, \emph{*args}}{}
\end{fulllineitems}

\index{appendRow() (src.PacketTableModel.PacketTableModel method)}

\begin{fulllineitems}
\phantomsection\label{\detokenize{src:src.PacketTableModel.PacketTableModel.appendRow}}\pysiglinewithargsret{\sphinxbfcode{appendRow}}{\emph{dataList={[}{]}}, \emph{addAtFront=False}, \emph{emit=True}, \emph{resolveDescription=False}}{}
Inserts the \sphinxcode{dataList} list into \sphinxcode{self.dataList} to add a whole row with values at once.
\begin{quote}\begin{description}
\item[{Parameters}] \leavevmode\begin{itemize}
\item {} 
\sphinxstyleliteralstrong{dataList} \textendash{} The list containing data. The length must be equal to {\hyperref[\detokenize{src:src.PacketTableModel.PacketTableModel.rowCount}]{\sphinxcrossref{\sphinxcode{rowCount()}}}}.

\item {} 
\sphinxstyleliteralstrong{addAtFront} \textendash{} Values will be added to the front of \sphinxcode{self.dataList} if this is True.
Else: They will be appended at the end

\item {} 
\sphinxstyleliteralstrong{emit} \textendash{} Optional: If the GUI will be notified of the data change or not. This is used for batch
imports where the GUI isn’t notified after each row to increase speed
Default: True (Emit everytime)

\item {} 
\sphinxstyleliteralstrong{resolveDescription} \textendash{} If this is set to True, the description of the potential known packet will
be resolved. Default: False

\end{itemize}

\item[{Returns}] \leavevmode
The description of the known packet. If \sphinxcode{resolveDescription} is False, an empty string is returned.
Else None.

\end{description}\end{quote}

This also emits the \sphinxcode{dataChanged} and \sphinxcode{layoutChanged} signals to let the GUI know that
the data/layout has been changed.

\end{fulllineitems}

\index{appendRows() (src.PacketTableModel.PacketTableModel method)}

\begin{fulllineitems}
\phantomsection\label{\detokenize{src:src.PacketTableModel.PacketTableModel.appendRows}}\pysiglinewithargsret{\sphinxbfcode{appendRows}}{\emph{rowList}, \emph{addAtFront=False}, \emph{resolveDescriptions=True}}{}
This allows appending a whole set of rows at once using the best possible speed
\begin{quote}\begin{description}
\item[{Parameters}] \leavevmode\begin{itemize}
\item {} 
\sphinxstyleliteralstrong{rowList} \textendash{} List of raw data lists to append

\item {} 
\sphinxstyleliteralstrong{addAtFront} \textendash{} Values will be added to the front of \sphinxcode{self.dataList} if this is True.
Else: They will be appended at the end

\item {} 
\sphinxstyleliteralstrong{resolveDescriptions} \textendash{} If this is set to true, the description for every packet will be resolved.
Default: True

\end{itemize}

\item[{Returns}] \leavevmode
If \sphinxcode{resolveDescriptions} is True, a list of known packet descriptions will be returned. If no
description for a particular packet can be resolved, an empty string will be inserted in the list
to keep indexes. Else None will be returned

\end{description}\end{quote}

\end{fulllineitems}

\index{cellChanged (src.PacketTableModel.PacketTableModel attribute)}

\begin{fulllineitems}
\phantomsection\label{\detokenize{src:src.PacketTableModel.PacketTableModel.cellChanged}}\pysigline{\sphinxbfcode{cellChanged}\sphinxstrong{ = \textless{}PySide.QtCore.Signal object\textgreater{}}}
Emits rowIndex and columnIndex of the changed cell

\end{fulllineitems}

\index{clear() (src.PacketTableModel.PacketTableModel method)}

\begin{fulllineitems}
\phantomsection\label{\detokenize{src:src.PacketTableModel.PacketTableModel.clear}}\pysiglinewithargsret{\sphinxbfcode{clear}}{}{}
Clears all managed data from the \sphinxcode{dataList}.
This is a shortcut to {\hyperref[\detokenize{src:src.PacketTableModel.PacketTableModel.setRowCount}]{\sphinxcrossref{\sphinxcode{setRowCount()}}}} with parameter \sphinxcode{0}.

\end{fulllineitems}

\index{columnCount() (src.PacketTableModel.PacketTableModel method)}

\begin{fulllineitems}
\phantomsection\label{\detokenize{src:src.PacketTableModel.PacketTableModel.columnCount}}\pysiglinewithargsret{\sphinxbfcode{columnCount}}{\emph{parent=None}}{}
Returns the current column count by returning the length of the header list.
\begin{quote}\begin{description}
\item[{Parameters}] \leavevmode
\sphinxstyleliteralstrong{parent} \textendash{} Dummy parameter to keep the needed signature

\item[{Returns}] \leavevmode
The column count as integer

\end{description}\end{quote}

\end{fulllineitems}

\index{data() (src.PacketTableModel.PacketTableModel method)}

\begin{fulllineitems}
\phantomsection\label{\detokenize{src:src.PacketTableModel.PacketTableModel.data}}\pysiglinewithargsret{\sphinxbfcode{data}}{\emph{index}, \emph{role}}{}
Return managed data depending on the \sphinxcode{role} parameter.
\begin{quote}\begin{description}
\item[{Parameters}] \leavevmode\begin{itemize}
\item {} 
\sphinxstyleliteralstrong{index} \textendash{} Index object containing row and column index

\item {} 
\sphinxstyleliteralstrong{role} \textendash{} The display role that requests data

\end{itemize}

\item[{Returns}] \leavevmode
\begin{itemize}
\item {} 
If the index is invalid: None

\item {} 
\sphinxcode{AlignCenter} if \sphinxcode{role = TextAlignmentRole}

\item {} 
Column data if \sphinxcode{role = DisplayRole} or \sphinxcode{EditRole}

\end{itemize}


\end{description}\end{quote}

\end{fulllineitems}

\index{flags() (src.PacketTableModel.PacketTableModel method)}

\begin{fulllineitems}
\phantomsection\label{\detokenize{src:src.PacketTableModel.PacketTableModel.flags}}\pysiglinewithargsret{\sphinxbfcode{flags}}{\emph{index}}{}
Return the flags for cell at a given index.
\begin{quote}\begin{description}
\item[{Parameters}] \leavevmode
\sphinxstyleliteralstrong{index} \textendash{} Index object containing row and column index

\item[{Returns}] \leavevmode
A flags object containing whether an object is editable, selectable or enabled

\end{description}\end{quote}

\end{fulllineitems}

\index{getValue() (src.PacketTableModel.PacketTableModel method)}

\begin{fulllineitems}
\phantomsection\label{\detokenize{src:src.PacketTableModel.PacketTableModel.getValue}}\pysiglinewithargsret{\sphinxbfcode{getValue}}{\emph{rowIndex}, \emph{colIndex}}{}
Get the data from the table at the given indexes.
\begin{quote}\begin{description}
\item[{Parameters}] \leavevmode\begin{itemize}
\item {} 
\sphinxstyleliteralstrong{rowIndex} \textendash{} Row index

\item {} 
\sphinxstyleliteralstrong{colIndex} \textendash{} Column index

\end{itemize}

\item[{Returns}] \leavevmode
The data at the specified index (if possible); Else None

\end{description}\end{quote}

\end{fulllineitems}

\index{headerData() (src.PacketTableModel.PacketTableModel method)}

\begin{fulllineitems}
\phantomsection\label{\detokenize{src:src.PacketTableModel.PacketTableModel.headerData}}\pysiglinewithargsret{\sphinxbfcode{headerData}}{\emph{headerIndex}, \emph{orientation}, \emph{role}}{}
Returns the header data to properly display the managed data on the GUI.
\begin{quote}\begin{description}
\item[{Parameters}] \leavevmode\begin{itemize}
\item {} 
\sphinxstyleliteralstrong{headerIndex} \textendash{} Which column of the data is requested

\item {} 
\sphinxstyleliteralstrong{orientation} \textendash{} 
This can be either \sphinxcode{Horizontal} or \sphinxcode{Vertical}:
\begin{itemize}
\item {} 
\sphinxcode{Horizontal}: Return a value from \sphinxcode{self.header}

\item {} 
\sphinxcode{Vertical}: Return the \sphinxcode{headerIndex}

\end{itemize}


\item {} 
\sphinxstyleliteralstrong{role} \textendash{} This is always expected to be \sphinxcode{DisplayRole}

\end{itemize}

\item[{Returns}] \leavevmode
See \sphinxcode{orientation}. None is returned if \sphinxcode{orientation} or \sphinxcode{role} do not match

\end{description}\end{quote}

\end{fulllineitems}

\index{insertRow() (src.PacketTableModel.PacketTableModel method)}

\begin{fulllineitems}
\phantomsection\label{\detokenize{src:src.PacketTableModel.PacketTableModel.insertRow}}\pysiglinewithargsret{\sphinxbfcode{insertRow}}{\emph{dataList={[}{]}}}{}
This is just an alias to {\hyperref[\detokenize{src:src.PacketTableModel.PacketTableModel.appendRow}]{\sphinxcrossref{\sphinxcode{appendRow()}}}} for compatibility.
\begin{quote}\begin{description}
\item[{Parameters}] \leavevmode
\sphinxstyleliteralstrong{dataList} \textendash{} A list that stores the data that will be added

\end{description}\end{quote}

\end{fulllineitems}

\index{removeRow() (src.PacketTableModel.PacketTableModel method)}

\begin{fulllineitems}
\phantomsection\label{\detokenize{src:src.PacketTableModel.PacketTableModel.removeRow}}\pysiglinewithargsret{\sphinxbfcode{removeRow}}{\emph{rowIndex}}{}
Removes the specified row from the table model.
This also emits the \sphinxcode{layoutChanged} signal to let the GUI know that the layout has been changed.
\begin{quote}\begin{description}
\item[{Parameters}] \leavevmode
\sphinxstyleliteralstrong{rowIndex} \textendash{} The row index to delete

\end{description}\end{quote}

\end{fulllineitems}

\index{removeRows() (src.PacketTableModel.PacketTableModel method)}

\begin{fulllineitems}
\phantomsection\label{\detokenize{src:src.PacketTableModel.PacketTableModel.removeRows}}\pysiglinewithargsret{\sphinxbfcode{removeRows}}{\emph{rowIndexes}}{}
Remove multiple rows at once.
\begin{quote}\begin{description}
\item[{Parameters}] \leavevmode
\sphinxstyleliteralstrong{rowIndexes} \textendash{} The rows that will be deleted

\end{description}\end{quote}

\end{fulllineitems}

\index{rowCount() (src.PacketTableModel.PacketTableModel method)}

\begin{fulllineitems}
\phantomsection\label{\detokenize{src:src.PacketTableModel.PacketTableModel.rowCount}}\pysiglinewithargsret{\sphinxbfcode{rowCount}}{\emph{parent=None}}{}
Returns the current row count by returning the length of the data list.
\begin{quote}\begin{description}
\item[{Parameters}] \leavevmode
\sphinxstyleliteralstrong{parent} \textendash{} Dummy parameter to keep the needed signature

\item[{Returns}] \leavevmode
The row count as integer

\end{description}\end{quote}

\end{fulllineitems}

\index{setData() (src.PacketTableModel.PacketTableModel method)}

\begin{fulllineitems}
\phantomsection\label{\detokenize{src:src.PacketTableModel.PacketTableModel.setData}}\pysiglinewithargsret{\sphinxbfcode{setData}}{\emph{index}, \emph{value}, \emph{role=PySide.QtCore.Qt.ItemDataRole.EditRole}}{}
This gets called to change the element on the GUI at the given indexes
This also emits the \sphinxcode{layoutChanged} signal to let the GUI know that the layout has been changed.
\begin{quote}\begin{description}
\item[{Parameters}] \leavevmode\begin{itemize}
\item {} 
\sphinxstyleliteralstrong{index} \textendash{} Index object containing row and column index

\item {} 
\sphinxstyleliteralstrong{value} \textendash{} The new value

\item {} 
\sphinxstyleliteralstrong{role} \textendash{} Optional: The role calling this method. Default: EditRole

\end{itemize}

\item[{Returns}] \leavevmode
True if the operation succeeded

\end{description}\end{quote}

\end{fulllineitems}

\index{setRowCount() (src.PacketTableModel.PacketTableModel method)}

\begin{fulllineitems}
\phantomsection\label{\detokenize{src:src.PacketTableModel.PacketTableModel.setRowCount}}\pysiglinewithargsret{\sphinxbfcode{setRowCount}}{\emph{count}}{}
Sets the row count by removing lines / adding empty lines.
This also emits the \sphinxcode{layoutChanged} signal to let the GUI know that the layout has been changed.
\begin{quote}\begin{description}
\item[{Parameters}] \leavevmode
\sphinxstyleliteralstrong{count} \textendash{} The desired amount of rows

\end{description}\end{quote}

\end{fulllineitems}

\index{setText() (src.PacketTableModel.PacketTableModel method)}

\begin{fulllineitems}
\phantomsection\label{\detokenize{src:src.PacketTableModel.PacketTableModel.setText}}\pysiglinewithargsret{\sphinxbfcode{setText}}{\emph{rowIndex}, \emph{colIndex}, \emph{data}}{}
Sets the text of at the given indexes.
If \sphinxcode{data} is None the text will be an empty string.
This method also emits the \sphinxcode{dataChanged} and \sphinxcode{layoutChanged} signals to let the GUI know that
the data/layout has been changed.
\begin{quote}\begin{description}
\item[{Parameters}] \leavevmode\begin{itemize}
\item {} 
\sphinxstyleliteralstrong{rowIndex} \textendash{} Row index

\item {} 
\sphinxstyleliteralstrong{colIndex} \textendash{} Column index

\item {} 
\sphinxstyleliteralstrong{data} \textendash{} New data for the column

\end{itemize}

\end{description}\end{quote}

\end{fulllineitems}

\index{sort() (src.PacketTableModel.PacketTableModel method)}

\begin{fulllineitems}
\phantomsection\label{\detokenize{src:src.PacketTableModel.PacketTableModel.sort}}\pysiglinewithargsret{\sphinxbfcode{sort}}{\emph{colIndex}, \emph{order}}{}
Sort the data by given column number.
This also emits the \sphinxcode{layoutChanged} signal to let the GUI know that the layout has been changed.
\begin{quote}\begin{description}
\item[{Parameters}] \leavevmode\begin{itemize}
\item {} 
\sphinxstyleliteralstrong{colIndex} \textendash{} The column index to sort for

\item {} 
\sphinxstyleliteralstrong{order} \textendash{} Either \sphinxcode{DescendingOrder} or \sphinxcode{AscendingOrder}

\end{itemize}

\end{description}\end{quote}

\end{fulllineitems}

\index{staticMetaObject (src.PacketTableModel.PacketTableModel attribute)}

\begin{fulllineitems}
\phantomsection\label{\detokenize{src:src.PacketTableModel.PacketTableModel.staticMetaObject}}\pysigline{\sphinxbfcode{staticMetaObject}\sphinxstrong{ = \textless{}PySide.QtCore.QMetaObject object\textgreater{}}}
\end{fulllineitems}


\end{fulllineitems}



\section{CANalyzat0r.Project module}
\label{\detokenize{src:canalyzat0r-project-module}}\label{\detokenize{src:module-src.Project}}\index{src.Project (module)}
Created on May 19, 2017

@author: pschmied
\index{Project (class in src.Project)}

\begin{fulllineitems}
\phantomsection\label{\detokenize{src:src.Project.Project}}\pysiglinewithargsret{\sphinxstrong{class }\sphinxcode{src.Project.}\sphinxbfcode{Project}}{\emph{id}, \emph{name}, \emph{description}}{}
This class is being used to handle project data.
It’s more comfortable to use a object to pass data
than to use lists and list indexes. Please note that
this costs much more performance, so please use lists
if you have to deal with much data.
\index{\_\_init\_\_() (src.Project.Project method)}

\begin{fulllineitems}
\phantomsection\label{\detokenize{src:src.Project.Project.__init__}}\pysiglinewithargsret{\sphinxbfcode{\_\_init\_\_}}{\emph{id}, \emph{name}, \emph{description}}{}
This just sets the attributes to the passed parameters.

\end{fulllineitems}

\index{fromJSON() (src.Project.Project static method)}

\begin{fulllineitems}
\phantomsection\label{\detokenize{src:src.Project.Project.fromJSON}}\pysiglinewithargsret{\sphinxstrong{static }\sphinxbfcode{fromJSON}}{\emph{importJSON}}{}
This class method creates a project object using a JSON string.
\begin{quote}\begin{description}
\item[{Parameters}] \leavevmode
\sphinxstyleliteralstrong{importJSON} \textendash{} The JSON string containing the object data

\item[{Returns}] \leavevmode
A project object with the values set accordingly

\end{description}\end{quote}

\end{fulllineitems}

\index{toComboBoxString() (src.Project.Project method)}

\begin{fulllineitems}
\phantomsection\label{\detokenize{src:src.Project.Project.toComboBoxString}}\pysiglinewithargsret{\sphinxbfcode{toComboBoxString}}{}{}
Calculate a string which will be displayed in the ComboBoxes.
\begin{quote}\begin{description}
\item[{Returns}] \leavevmode
String representation of the object

\end{description}\end{quote}

\end{fulllineitems}

\index{toJSON() (src.Project.Project method)}

\begin{fulllineitems}
\phantomsection\label{\detokenize{src:src.Project.Project.toJSON}}\pysiglinewithargsret{\sphinxbfcode{toJSON}}{}{}
To export a Project, all data is being formatted as JSON.
The internal attribute \_\_dict\_\_ is being used to gather the data.
This data will then be formatted.
\begin{quote}\begin{description}
\item[{Returns}] \leavevmode
The formatted JSON data of the object

\end{description}\end{quote}

\end{fulllineitems}


\end{fulllineitems}



\section{CANalyzat0r.SearcherTab module}
\label{\detokenize{src:canalyzat0r-searchertab-module}}\label{\detokenize{src:module-src.SearcherTab}}\index{src.SearcherTab (module)}
Created on May 17, 2017

@author: pschmied
\index{SearcherTab (class in src.SearcherTab)}

\begin{fulllineitems}
\phantomsection\label{\detokenize{src:src.SearcherTab.SearcherTab}}\pysiglinewithargsret{\sphinxstrong{class }\sphinxcode{src.SearcherTab.}\sphinxbfcode{SearcherTab}}{\emph{tabWidget}}{}
Bases: {\hyperref[\detokenize{src:src.AbstractTab.AbstractTab}]{\sphinxcrossref{\sphinxcode{src.AbstractTab.AbstractTab}}}}

This class handles the logic of the filter tab
\index{\_\_init\_\_() (src.SearcherTab.SearcherTab method)}

\begin{fulllineitems}
\phantomsection\label{\detokenize{src:src.SearcherTab.SearcherTab.__init__}}\pysiglinewithargsret{\sphinxbfcode{\_\_init\_\_}}{\emph{tabWidget}}{}
\end{fulllineitems}

\index{askActionPerformed() (src.SearcherTab.SearcherTab method)}

\begin{fulllineitems}
\phantomsection\label{\detokenize{src:src.SearcherTab.SearcherTab.askActionPerformed}}\pysiglinewithargsret{\sphinxbfcode{askActionPerformed}}{}{}
Ask the user if the action has been performed using a MessageBox
\begin{quote}\begin{description}
\item[{Returns}] \leavevmode
True if the user pressed yes, else False

\end{description}\end{quote}

\end{fulllineitems}

\index{askWhichAction() (src.SearcherTab.SearcherTab method)}

\begin{fulllineitems}
\phantomsection\label{\detokenize{src:src.SearcherTab.SearcherTab.askWhichAction}}\pysiglinewithargsret{\sphinxbfcode{askWhichAction}}{}{}~\begin{description}
\item[{Ask the user what to do if no chunk worked:}] \leavevmode\begin{itemize}
\item {} 
Try again

\item {} 
Re-test the current last working chunk

\item {} 
Cancel

\end{itemize}

\end{description}
\begin{quote}\begin{description}
\item[{Returns}] \leavevmode
An integer value indicating the pressed button:
- 0 if the user wants to try again
- 1 if the user wants to re-test
- 2 if the user wants to cancel

\end{description}\end{quote}

\end{fulllineitems}

\index{beep() (src.SearcherTab.SearcherTab method)}

\begin{fulllineitems}
\phantomsection\label{\detokenize{src:src.SearcherTab.SearcherTab.beep}}\pysiglinewithargsret{\sphinxbfcode{beep}}{}{}
To play a sound after sending has been finished.

\end{fulllineitems}

\index{clear() (src.SearcherTab.SearcherTab method)}

\begin{fulllineitems}
\phantomsection\label{\detokenize{src:src.SearcherTab.SearcherTab.clear}}\pysiglinewithargsret{\sphinxbfcode{clear}}{\emph{returnOldPackets=False}}{}
Clear the GUI table and all associated data lists

\end{fulllineitems}

\index{downwardsSearch (src.SearcherTab.SearcherTab attribute)}

\begin{fulllineitems}
\phantomsection\label{\detokenize{src:src.SearcherTab.SearcherTab.downwardsSearch}}\pysigline{\sphinxbfcode{downwardsSearch}\sphinxstrong{ = None}}
We first search downwards in the binary search tree. If this doesn’t
succeed, we use randomization and begint to search upwards.

\end{fulllineitems}

\index{enterWhenReady() (src.SearcherTab.SearcherTab method)}

\begin{fulllineitems}
\phantomsection\label{\detokenize{src:src.SearcherTab.SearcherTab.enterWhenReady}}\pysiglinewithargsret{\sphinxbfcode{enterWhenReady}}{}{}
Block the GUI thread until the user pressed the button on the MessageBox.

\end{fulllineitems}

\index{lastWorkingChunk (src.SearcherTab.SearcherTab attribute)}

\begin{fulllineitems}
\phantomsection\label{\detokenize{src:src.SearcherTab.SearcherTab.lastWorkingChunk}}\pysigline{\sphinxbfcode{lastWorkingChunk}\sphinxstrong{ = None}}
The currently smallest known set of packets that cause a specific action.
Values are lists with raw data

\end{fulllineitems}

\index{outputRemainingPacket() (src.SearcherTab.SearcherTab method)}

\begin{fulllineitems}
\phantomsection\label{\detokenize{src:src.SearcherTab.SearcherTab.outputRemainingPacket}}\pysiglinewithargsret{\sphinxbfcode{outputRemainingPacket}}{\emph{packet}}{}
Show the passed packet on the GUI table.
\begin{quote}\begin{description}
\item[{Parameters}] \leavevmode
\sphinxstyleliteralstrong{packet} \textendash{} The raw packet to display

\end{description}\end{quote}

\end{fulllineitems}

\index{outputRemainingPackets() (src.SearcherTab.SearcherTab method)}

\begin{fulllineitems}
\phantomsection\label{\detokenize{src:src.SearcherTab.SearcherTab.outputRemainingPackets}}\pysiglinewithargsret{\sphinxbfcode{outputRemainingPackets}}{\emph{rawPackets}}{}
Show all passed packets on the GUI table.
Note: This also sets \sphinxcode{rawData} to the passed set of packets.
\begin{quote}\begin{description}
\item[{Parameters}] \leavevmode
\sphinxstyleliteralstrong{packet} \textendash{} List of raw packets to display

\end{description}\end{quote}

\end{fulllineitems}

\index{searchPackets() (src.SearcherTab.SearcherTab method)}

\begin{fulllineitems}
\phantomsection\label{\detokenize{src:src.SearcherTab.SearcherTab.searchPackets}}\pysiglinewithargsret{\sphinxbfcode{searchPackets}}{}{}~\begin{description}
\item[{This starts the whole searching routine and sets up things first:}] \leavevmode\begin{enumerate}
\item {} 
Set a CANData instance

\item {} 
Get user input values

\item {} 
Walk down the binary search tree: Try to find a specific packet for an action

\item {} 
If 1 packet has been found: output the packet

\item {} 
If not: Get the last working chunk of packets that worked.
Use shuffling and new values for the chunk amount to find a minimal set of packets

\end{enumerate}

\end{description}

\end{fulllineitems}

\index{sendAndSearch() (src.SearcherTab.SearcherTab method)}

\begin{fulllineitems}
\phantomsection\label{\detokenize{src:src.SearcherTab.SearcherTab.sendAndSearch}}\pysiglinewithargsret{\sphinxbfcode{sendAndSearch}}{\emph{chunkAmount=2}}{}~\begin{description}
\item[{Use the remaining data to search for relevant packets:}] \leavevmode\begin{enumerate}
\item {} 
Setup a progress bar

\item {} 
Split the raw packet list in the desired amount of chunks

\item {} 
Test each chunk (Newest packets first) and ask the user if it worked

\item {} 
If it worked: Set \sphinxcode{lastWorkingChunk} to the last tested chunk and return True

\item {} 
Else: Return False if all other chunks failed too.

\end{enumerate}

\end{description}
\begin{quote}\begin{description}
\item[{Parameters}] \leavevmode
\sphinxstyleliteralstrong{chunkAmount} \textendash{} The amount of chunks to generate from the given data list

\item[{Returns}] \leavevmode
True if a specific chunk worked, else False

\end{description}\end{quote}

\end{fulllineitems}

\index{splitLists() (src.SearcherTab.SearcherTab method)}

\begin{fulllineitems}
\phantomsection\label{\detokenize{src:src.SearcherTab.SearcherTab.splitLists}}\pysiglinewithargsret{\sphinxbfcode{splitLists}}{\emph{lst}, \emph{chunkAmount=2}}{}
Split a list into a specific amount of chunks
\begin{quote}\begin{description}
\item[{Parameters}] \leavevmode\begin{itemize}
\item {} 
\sphinxstyleliteralstrong{lst} \textendash{} The list to split

\item {} 
\sphinxstyleliteralstrong{chunkAmount} \textendash{} Desired amount of chunks

\end{itemize}

\item[{Returns}] \leavevmode
List of chunks (List of lists in this case)

\end{description}\end{quote}

\end{fulllineitems}

\index{toggleGUIElements() (src.SearcherTab.SearcherTab method)}

\begin{fulllineitems}
\phantomsection\label{\detokenize{src:src.SearcherTab.SearcherTab.toggleGUIElements}}\pysiglinewithargsret{\sphinxbfcode{toggleGUIElements}}{\emph{state}}{}
\{En, Dis\}able all GUI elements that are used to change searcher settings
\begin{quote}\begin{description}
\item[{Parameters}] \leavevmode
\sphinxstyleliteralstrong{state} \textendash{} Boolean value to indicate whether to enable or disable elements

\end{description}\end{quote}

\end{fulllineitems}


\end{fulllineitems}



\section{CANalyzat0r.SenderTab module}
\label{\detokenize{src:canalyzat0r-sendertab-module}}\label{\detokenize{src:module-src.SenderTab}}\index{src.SenderTab (module)}
Created on May 22, 2017

@author: pschmied
\index{SenderTab (class in src.SenderTab)}

\begin{fulllineitems}
\phantomsection\label{\detokenize{src:src.SenderTab.SenderTab}}\pysigline{\sphinxstrong{class }\sphinxcode{src.SenderTab.}\sphinxbfcode{SenderTab}}
This class handles the logic of the sender tab.
Subtabs are being handled in {\hyperref[\detokenize{src:src.SenderTabElement.SenderTabElement}]{\sphinxcrossref{\sphinxcode{SenderTabElement}}}}.
\index{CANData (src.SenderTab.SenderTab attribute)}

\begin{fulllineitems}
\phantomsection\label{\detokenize{src:src.SenderTab.SenderTab.CANData}}\pysigline{\sphinxbfcode{CANData}\sphinxstrong{ = None}}
The tab specific CANData instance

\end{fulllineitems}

\index{active (src.SenderTab.SenderTab attribute)}

\begin{fulllineitems}
\phantomsection\label{\detokenize{src:src.SenderTab.SenderTab.active}}\pysigline{\sphinxbfcode{active}\sphinxstrong{ = False}}
\end{fulllineitems}

\index{addSender() (src.SenderTab.SenderTab static method)}

\begin{fulllineitems}
\phantomsection\label{\detokenize{src:src.SenderTab.SenderTab.addSender}}\pysiglinewithargsret{\sphinxstrong{static }\sphinxbfcode{addSender}}{\emph{senderTabName=None}}{}
Appends a new sender tab to the sub tab bar.
\begin{quote}\begin{description}
\item[{Parameters}] \leavevmode
\sphinxstyleliteralstrong{senderTabName} \textendash{} Optional; The displayed name of the tab.
If this is None, the user is requested to enter a name

\end{description}\end{quote}

\end{fulllineitems}

\index{addSenderWithData() (src.SenderTab.SenderTab static method)}

\begin{fulllineitems}
\phantomsection\label{\detokenize{src:src.SenderTab.SenderTab.addSenderWithData}}\pysiglinewithargsret{\sphinxstrong{static }\sphinxbfcode{addSenderWithData}}{\emph{listOfRawPackets=None}, \emph{listOfPackets=None}}{}
Uses {\hyperref[\detokenize{src:src.SenderTab.SenderTab.addSender}]{\sphinxcrossref{\sphinxcode{addSender()}}}} to add a new sender tab with data already filled in into the GUI table.
You must specify \sphinxcode{listOfRawPackets} \sphinxstylestrong{or} \sphinxcode{listOfPackets}. If both are specified,
\sphinxcode{listOfRawPackets} will be used.
\begin{quote}\begin{description}
\item[{Parameters}] \leavevmode\begin{itemize}
\item {} 
\sphinxstyleliteralstrong{listOfRawPackets} \textendash{} Optional; List of raw packets to add to the table.

\item {} 
\sphinxstyleliteralstrong{listOfPackets} \textendash{} Optional; List of packet objects to add to the table.

\end{itemize}

\item[{Returns}] \leavevmode


\end{description}\end{quote}

\end{fulllineitems}

\index{currentlySendingTabs (src.SenderTab.SenderTab attribute)}

\begin{fulllineitems}
\phantomsection\label{\detokenize{src:src.SenderTab.SenderTab.currentlySendingTabs}}\pysigline{\sphinxbfcode{currentlySendingTabs}\sphinxstrong{ = 0}}
Used to handle the font color of the sender tab

\end{fulllineitems}

\index{handleInterfaceSettingsDialog() (src.SenderTab.SenderTab class method)}

\begin{fulllineitems}
\phantomsection\label{\detokenize{src:src.SenderTab.SenderTab.handleInterfaceSettingsDialog}}\pysiglinewithargsret{\sphinxstrong{classmethod }\sphinxbfcode{handleInterfaceSettingsDialog}}{}{}
This invokes \sphinxcode{handleInterfaceSettingsDialog()} for the class

\end{fulllineitems}

\index{indexInMainTabBar (src.SenderTab.SenderTab attribute)}

\begin{fulllineitems}
\phantomsection\label{\detokenize{src:src.SenderTab.SenderTab.indexInMainTabBar}}\pysigline{\sphinxbfcode{indexInMainTabBar}\sphinxstrong{ = 2}}
The index of the sender tab in the main tab bar

\end{fulllineitems}

\index{labelInterfaceValue (src.SenderTab.SenderTab attribute)}

\begin{fulllineitems}
\phantomsection\label{\detokenize{src:src.SenderTab.SenderTab.labelInterfaceValue}}\pysigline{\sphinxbfcode{labelInterfaceValue}\sphinxstrong{ = None}}
\end{fulllineitems}

\index{logger (src.SenderTab.SenderTab attribute)}

\begin{fulllineitems}
\phantomsection\label{\detokenize{src:src.SenderTab.SenderTab.logger}}\pysigline{\sphinxbfcode{logger}\sphinxstrong{ = \textless{}logging.Logger object\textgreater{}}}
The tab specific logger

\end{fulllineitems}

\index{prepareUI() (src.SenderTab.SenderTab static method)}

\begin{fulllineitems}
\phantomsection\label{\detokenize{src:src.SenderTab.SenderTab.prepareUI}}\pysiglinewithargsret{\sphinxstrong{static }\sphinxbfcode{prepareUI}}{}{}
Prepare the tab specific GUI elements, add sender tab and keyboard shortcuts. Also set a CANData instance.

\end{fulllineitems}

\index{removeSender() (src.SenderTab.SenderTab static method)}

\begin{fulllineitems}
\phantomsection\label{\detokenize{src:src.SenderTab.SenderTab.removeSender}}\pysiglinewithargsret{\sphinxstrong{static }\sphinxbfcode{removeSender}}{\emph{senderTabElement}}{}
Remove a sender from the sub tab bar. This method gets called from an instance of
{\hyperref[\detokenize{src:src.SenderTabElement.SenderTabElement}]{\sphinxcrossref{\sphinxcode{SenderTabElement}}}} by {\hyperref[\detokenize{src:src.SenderTabElement.SenderTabElement.removeSender}]{\sphinxcrossref{\sphinxcode{removeSender()}}}}.
\begin{quote}\begin{description}
\item[{Parameters}] \leavevmode
\sphinxstyleliteralstrong{senderTabElement} \textendash{} The {\hyperref[\detokenize{src:src.SenderTabElement.SenderTabElement}]{\sphinxcrossref{\sphinxcode{SenderTabElement}}}} instance to remove

\end{description}\end{quote}

\end{fulllineitems}

\index{sendSinglePacket() (src.SenderTab.SenderTab static method)}

\begin{fulllineitems}
\phantomsection\label{\detokenize{src:src.SenderTab.SenderTab.sendSinglePacket}}\pysiglinewithargsret{\sphinxstrong{static }\sphinxbfcode{sendSinglePacket}}{}{}
Sends a single packet using the specified interface.
All packet values are read from the GUI elements.

\end{fulllineitems}

\index{senderTabs (src.SenderTab.SenderTab attribute)}

\begin{fulllineitems}
\phantomsection\label{\detokenize{src:src.SenderTab.SenderTab.senderTabs}}\pysigline{\sphinxbfcode{senderTabs}\sphinxstrong{ = {[}{]}}}
Consinsts of all SenderTabElements

\end{fulllineitems}

\index{setInitialCANData() (src.SenderTab.SenderTab class method)}

\begin{fulllineitems}
\phantomsection\label{\detokenize{src:src.SenderTab.SenderTab.setInitialCANData}}\pysiglinewithargsret{\sphinxstrong{classmethod }\sphinxbfcode{setInitialCANData}}{}{}
This invokes \sphinxcode{setInitialCANData()} for the class

\end{fulllineitems}

\index{toggleActive() (src.SenderTab.SenderTab static method)}

\begin{fulllineitems}
\phantomsection\label{\detokenize{src:src.SenderTab.SenderTab.toggleActive}}\pysiglinewithargsret{\sphinxstrong{static }\sphinxbfcode{toggleActive}}{}{}
If there is at least one tab sending then the tab bar title will be red.

\end{fulllineitems}

\index{toggleGUIElements() (src.SenderTab.SenderTab static method)}

\begin{fulllineitems}
\phantomsection\label{\detokenize{src:src.SenderTab.SenderTab.toggleGUIElements}}\pysiglinewithargsret{\sphinxstrong{static }\sphinxbfcode{toggleGUIElements}}{\emph{state}}{}
\{En, Dis\}able all GUI elements that are used to change filter settings
\begin{quote}\begin{description}
\item[{Parameters}] \leavevmode
\sphinxstyleliteralstrong{state} \textendash{} Boolean value to indicate whether to enable or disable elements

\end{description}\end{quote}

\end{fulllineitems}

\index{updateCANDataInstance() (src.SenderTab.SenderTab class method)}

\begin{fulllineitems}
\phantomsection\label{\detokenize{src:src.SenderTab.SenderTab.updateCANDataInstance}}\pysiglinewithargsret{\sphinxstrong{classmethod }\sphinxbfcode{updateCANDataInstance}}{\emph{CANDataInstance}, \emph{delegate=False}}{}
This invokes \sphinxcode{updateCANDataInstance()} for the class
\begin{quote}\begin{description}
\item[{Parameters}] \leavevmode\begin{itemize}
\item {} 
\sphinxstyleliteralstrong{CANDataInstance} \textendash{} The new CANData instance

\item {} 
\sphinxstyleliteralstrong{delegate} \textendash{} Boolean indicating if all sender sub tabs will be updated too. Default: False

\end{itemize}

\end{description}\end{quote}

\end{fulllineitems}

\index{updateInterfaceLabel() (src.SenderTab.SenderTab class method)}

\begin{fulllineitems}
\phantomsection\label{\detokenize{src:src.SenderTab.SenderTab.updateInterfaceLabel}}\pysiglinewithargsret{\sphinxstrong{classmethod }\sphinxbfcode{updateInterfaceLabel}}{}{}
This invokes \sphinxcode{updateInterfaceLabel()} for the class

\end{fulllineitems}


\end{fulllineitems}



\section{CANalyzat0r.SenderTabElement module}
\label{\detokenize{src:module-src.SenderTabElement}}\label{\detokenize{src:canalyzat0r-sendertabelement-module}}\index{src.SenderTabElement (module)}
Created on May 23, 2017

@author: pschmied
\index{SenderTabElement (class in src.SenderTabElement)}

\begin{fulllineitems}
\phantomsection\label{\detokenize{src:src.SenderTabElement.SenderTabElement}}\pysiglinewithargsret{\sphinxstrong{class }\sphinxcode{src.SenderTabElement.}\sphinxbfcode{SenderTabElement}}{\emph{tabWidget}, \emph{tabName}}{}
Bases: {\hyperref[\detokenize{src:src.AbstractTab.AbstractTab}]{\sphinxcrossref{\sphinxcode{src.AbstractTab.AbstractTab}}}}

This class handles the logic of the sender sub tab.
The main tab is being handled in {\hyperref[\detokenize{src:src.SenderTab.SenderTab}]{\sphinxcrossref{\sphinxcode{SenderTab}}}}.
\index{\_\_init\_\_() (src.SenderTabElement.SenderTabElement method)}

\begin{fulllineitems}
\phantomsection\label{\detokenize{src:src.SenderTabElement.SenderTabElement.__init__}}\pysiglinewithargsret{\sphinxbfcode{\_\_init\_\_}}{\emph{tabWidget}, \emph{tabName}}{}
Set all passed data. Also, add the own send button to \sphinxcode{SenderTab.sendButtonList} to
allow managing it globally.
\begin{quote}\begin{description}
\item[{Parameters}] \leavevmode
\sphinxstyleliteralstrong{tabWidget} \textendash{} The element in the tab bar. \sphinxstylestrong{Not} the table widget.

\end{description}\end{quote}

\end{fulllineitems}

\index{amountThreadsRunning (src.SenderTabElement.SenderTabElement attribute)}

\begin{fulllineitems}
\phantomsection\label{\detokenize{src:src.SenderTabElement.SenderTabElement.amountThreadsRunning}}\pysigline{\sphinxbfcode{amountThreadsRunning}\sphinxstrong{ = 0}}
Amount of sending threads running to display in the status bar

\end{fulllineitems}

\index{getTabIndex() (src.SenderTabElement.SenderTabElement method)}

\begin{fulllineitems}
\phantomsection\label{\detokenize{src:src.SenderTabElement.SenderTabElement.getTabIndex}}\pysiglinewithargsret{\sphinxbfcode{getTabIndex}}{}{}
Get the \sphinxstylestrong{current} tab index of the sub tab element
\begin{quote}\begin{description}
\item[{Returns}] \leavevmode
The tab index of the sender tab

\end{description}\end{quote}

\end{fulllineitems}

\index{loopSenderThread (src.SenderTabElement.SenderTabElement attribute)}

\begin{fulllineitems}
\phantomsection\label{\detokenize{src:src.SenderTabElement.SenderTabElement.loopSenderThread}}\pysigline{\sphinxbfcode{loopSenderThread}\sphinxstrong{ = None}}
The thread that runs when sending takes place in a loop

\end{fulllineitems}

\index{prepareUI() (src.SenderTabElement.SenderTabElement method)}

\begin{fulllineitems}
\phantomsection\label{\detokenize{src:src.SenderTabElement.SenderTabElement.prepareUI}}\pysiglinewithargsret{\sphinxbfcode{prepareUI}}{}{}
Prepare the tab specific GUI elements, add keyboard shortcuts and set a CANData instance

\end{fulllineitems}

\index{removeSender() (src.SenderTabElement.SenderTabElement method)}

\begin{fulllineitems}
\phantomsection\label{\detokenize{src:src.SenderTabElement.SenderTabElement.removeSender}}\pysiglinewithargsret{\sphinxbfcode{removeSender}}{}{}
This gets called when the remove sender button is pressed on the sub tab.
This stops the sender thread and calls the parents method ({\hyperref[\detokenize{src:src.SenderTab.SenderTab.removeSender}]{\sphinxcrossref{\sphinxcode{removeSender()}}}})
to remove the sender form the tab bar.

\end{fulllineitems}

\index{sendAll() (src.SenderTabElement.SenderTabElement method)}

\begin{fulllineitems}
\phantomsection\label{\detokenize{src:src.SenderTabElement.SenderTabElement.sendAll}}\pysiglinewithargsret{\sphinxbfcode{sendAll}}{}{}
Send all packets in the GUI table. By default, this just sends the packet once using simple calls.
If the user requests to send the packets in a loop, an instance of {\hyperref[\detokenize{src:src.SenderThread.LoopSenderThread}]{\sphinxcrossref{\sphinxcode{LoopSenderThread}}}}
is being used to send the packets.
Also, GUI elements like the status bar are being updated.

\end{fulllineitems}

\index{sendButtonList (src.SenderTabElement.SenderTabElement attribute)}

\begin{fulllineitems}
\phantomsection\label{\detokenize{src:src.SenderTabElement.SenderTabElement.sendButtonList}}\pysigline{\sphinxbfcode{sendButtonList}\sphinxstrong{ = {[}{]}}}
Static attribute of send buttons to manage \{en, dis\}abled states

\end{fulllineitems}

\index{setSendButtonState() (src.SenderTabElement.SenderTabElement method)}

\begin{fulllineitems}
\phantomsection\label{\detokenize{src:src.SenderTabElement.SenderTabElement.setSendButtonState}}\pysiglinewithargsret{\sphinxbfcode{setSendButtonState}}{\emph{state}}{}
This sets the enabled state of the send button.
\begin{quote}\begin{description}
\item[{Parameters}] \leavevmode
\sphinxstyleliteralstrong{state} \textendash{} The desired enabled state as boolean value

\end{description}\end{quote}

\end{fulllineitems}

\index{stopSending() (src.SenderTabElement.SenderTabElement method)}

\begin{fulllineitems}
\phantomsection\label{\detokenize{src:src.SenderTabElement.SenderTabElement.stopSending}}\pysiglinewithargsret{\sphinxbfcode{stopSending}}{}{}
This stops the currently running instance of {\hyperref[\detokenize{src:src.SenderThread.LoopSenderThread}]{\sphinxcrossref{\sphinxcode{LoopSenderThread}}}} from sending.
Also, GUI elements like the status bar are being updated.

\end{fulllineitems}

\index{toggleGUIElements() (src.SenderTabElement.SenderTabElement method)}

\begin{fulllineitems}
\phantomsection\label{\detokenize{src:src.SenderTabElement.SenderTabElement.toggleGUIElements}}\pysiglinewithargsret{\sphinxbfcode{toggleGUIElements}}{\emph{state}}{}
\{En, Dis\}able all GUI elements that are used to change filter settings
\begin{quote}\begin{description}
\item[{Parameters}] \leavevmode
\sphinxstyleliteralstrong{state} \textendash{} Boolean value to indicate whether to enable or disable elements

\end{description}\end{quote}

\end{fulllineitems}

\index{toggleLoopActive() (src.SenderTabElement.SenderTabElement method)}

\begin{fulllineitems}
\phantomsection\label{\detokenize{src:src.SenderTabElement.SenderTabElement.toggleLoopActive}}\pysiglinewithargsret{\sphinxbfcode{toggleLoopActive}}{}{}
Toggles the current sub tab to (in)active. This also calls {\hyperref[\detokenize{src:src.SenderTab.SenderTab.toggleActive}]{\sphinxcrossref{\sphinxcode{toggleActive()}}}} to manage
the color of the main tab (parent tab).

\end{fulllineitems}

\index{updateStatusBar() (src.SenderTabElement.SenderTabElement method)}

\begin{fulllineitems}
\phantomsection\label{\detokenize{src:src.SenderTabElement.SenderTabElement.updateStatusBar}}\pysiglinewithargsret{\sphinxbfcode{updateStatusBar}}{}{}
Updates the status bar label to display the correct amount of sending tabs (if any)

\end{fulllineitems}


\end{fulllineitems}



\section{CANalyzat0r.SenderThread module}
\label{\detokenize{src:module-src.SenderThread}}\label{\detokenize{src:canalyzat0r-senderthread-module}}\index{src.SenderThread (module)}
Created on Jun 08, 2017

@author: pschmied
\index{FuzzSenderThread (class in src.SenderThread)}

\begin{fulllineitems}
\phantomsection\label{\detokenize{src:src.SenderThread.FuzzSenderThread}}\pysiglinewithargsret{\sphinxstrong{class }\sphinxcode{src.SenderThread.}\sphinxbfcode{FuzzSenderThread}}{\emph{sleepTime}, \emph{fuzzerSendPipe}, \emph{CANData}, \emph{threadName}}{}
Bases: \sphinxcode{PySide.QtCore.QThread}

Spawns a new thread that will send random data in a loop.
\index{\_\_init\_\_() (src.SenderThread.FuzzSenderThread method)}

\begin{fulllineitems}
\phantomsection\label{\detokenize{src:src.SenderThread.FuzzSenderThread.__init__}}\pysiglinewithargsret{\sphinxbfcode{\_\_init\_\_}}{\emph{sleepTime}, \emph{fuzzerSendPipe}, \emph{CANData}, \emph{threadName}}{}
\end{fulllineitems}

\index{disable() (src.SenderThread.FuzzSenderThread method)}

\begin{fulllineitems}
\phantomsection\label{\detokenize{src:src.SenderThread.FuzzSenderThread.disable}}\pysiglinewithargsret{\sphinxbfcode{disable}}{}{}
This sets the \sphinxcode{enabled} flag to False which causes the main loop to terminate.

\end{fulllineitems}

\index{run() (src.SenderThread.FuzzSenderThread method)}

\begin{fulllineitems}
\phantomsection\label{\detokenize{src:src.SenderThread.FuzzSenderThread.run}}\pysiglinewithargsret{\sphinxbfcode{run}}{}{}
Send the packets in a loop and wait accordingly until the thread is disabled.

\end{fulllineitems}

\index{staticMetaObject (src.SenderThread.FuzzSenderThread attribute)}

\begin{fulllineitems}
\phantomsection\label{\detokenize{src:src.SenderThread.FuzzSenderThread.staticMetaObject}}\pysigline{\sphinxbfcode{staticMetaObject}\sphinxstrong{ = \textless{}PySide.QtCore.QMetaObject object\textgreater{}}}
\end{fulllineitems}


\end{fulllineitems}

\index{LoopSenderThread (class in src.SenderThread)}

\begin{fulllineitems}
\phantomsection\label{\detokenize{src:src.SenderThread.LoopSenderThread}}\pysiglinewithargsret{\sphinxstrong{class }\sphinxcode{src.SenderThread.}\sphinxbfcode{LoopSenderThread}}{\emph{packets}, \emph{sleepTime}, \emph{CANData}, \emph{threadName}}{}
Bases: \sphinxcode{PySide.QtCore.QThread}

Spawns a new thread that will send the passed packets in a loop.
\index{\_\_init\_\_() (src.SenderThread.LoopSenderThread method)}

\begin{fulllineitems}
\phantomsection\label{\detokenize{src:src.SenderThread.LoopSenderThread.__init__}}\pysiglinewithargsret{\sphinxbfcode{\_\_init\_\_}}{\emph{packets}, \emph{sleepTime}, \emph{CANData}, \emph{threadName}}{}
\end{fulllineitems}

\index{disable() (src.SenderThread.LoopSenderThread method)}

\begin{fulllineitems}
\phantomsection\label{\detokenize{src:src.SenderThread.LoopSenderThread.disable}}\pysiglinewithargsret{\sphinxbfcode{disable}}{}{}
This sets the \sphinxcode{enabled} flag to False which causes the main loop to terminate.

\end{fulllineitems}

\index{run() (src.SenderThread.LoopSenderThread method)}

\begin{fulllineitems}
\phantomsection\label{\detokenize{src:src.SenderThread.LoopSenderThread.run}}\pysiglinewithargsret{\sphinxbfcode{run}}{}{}
Send the packets in a loop and wait accordingly until the thread is disabled.

\end{fulllineitems}

\index{staticMetaObject (src.SenderThread.LoopSenderThread attribute)}

\begin{fulllineitems}
\phantomsection\label{\detokenize{src:src.SenderThread.LoopSenderThread.staticMetaObject}}\pysigline{\sphinxbfcode{staticMetaObject}\sphinxstrong{ = \textless{}PySide.QtCore.QMetaObject object\textgreater{}}}
\end{fulllineitems}


\end{fulllineitems}



\section{CANalyzat0r.Settings module}
\label{\detokenize{src:canalyzat0r-settings-module}}\label{\detokenize{src:module-src.Settings}}\index{src.Settings (module)}
Created on May 17, 2017

@author: pschmied
\index{APP\_NAME (in module src.Settings)}

\begin{fulllineitems}
\phantomsection\label{\detokenize{src:src.Settings.APP_NAME}}\pysigline{\sphinxcode{src.Settings.}\sphinxbfcode{APP\_NAME}\sphinxstrong{ = ‘CANalyzat0r’}}
The application name

\end{fulllineitems}

\index{APP\_VERSION (in module src.Settings)}

\begin{fulllineitems}
\phantomsection\label{\detokenize{src:src.Settings.APP_VERSION}}\pysigline{\sphinxcode{src.Settings.}\sphinxbfcode{APP\_VERSION}\sphinxstrong{ = ‘1.0’}}
The application version

\end{fulllineitems}

\index{DB\_NAME (in module src.Settings)}

\begin{fulllineitems}
\phantomsection\label{\detokenize{src:src.Settings.DB_NAME}}\pysigline{\sphinxcode{src.Settings.}\sphinxbfcode{DB\_NAME}\sphinxstrong{ = ‘../data/database.db’}}
The relative path of the SQLite database file

\end{fulllineitems}

\index{DB\_PATH (in module src.Settings)}

\begin{fulllineitems}
\phantomsection\label{\detokenize{src:src.Settings.DB_PATH}}\pysigline{\sphinxcode{src.Settings.}\sphinxbfcode{DB\_PATH}\sphinxstrong{ = ‘../data/database.db’}}
Optionally we can specify a different database path

\end{fulllineitems}

\index{FORKME\_PATH (in module src.Settings)}

\begin{fulllineitems}
\phantomsection\label{\detokenize{src:src.Settings.FORKME_PATH}}\pysigline{\sphinxcode{src.Settings.}\sphinxbfcode{FORKME\_PATH}\sphinxstrong{ = ‘./ui/icon/forkme.png’}}
Where to find the “Fork Me” icon

\end{fulllineitems}

\index{GITHUB\_URL (in module src.Settings)}

\begin{fulllineitems}
\phantomsection\label{\detokenize{src:src.Settings.GITHUB_URL}}\pysigline{\sphinxcode{src.Settings.}\sphinxbfcode{GITHUB\_URL}\sphinxstrong{ = ‘https://github.com/SCHUTZWERK-CANalyzat0r/CANalyzat0r’}}
Where to find this project in GitHub

\end{fulllineitems}

\index{ICON\_PATH (in module src.Settings)}

\begin{fulllineitems}
\phantomsection\label{\detokenize{src:src.Settings.ICON_PATH}}\pysigline{\sphinxcode{src.Settings.}\sphinxbfcode{ICON\_PATH}\sphinxstrong{ = ‘./ui/icon/icon.png’}}
Where to find the app icon

\end{fulllineitems}

\index{LOGO\_PATH (in module src.Settings)}

\begin{fulllineitems}
\phantomsection\label{\detokenize{src:src.Settings.LOGO_PATH}}\pysigline{\sphinxcode{src.Settings.}\sphinxbfcode{LOGO\_PATH}\sphinxstrong{ = ‘./ui/icon/swlogo\_small.png’}}
Where to find the company logo

\end{fulllineitems}



\section{CANalyzat0r.SnifferProcess module}
\label{\detokenize{src:module-src.SnifferProcess}}\label{\detokenize{src:canalyzat0r-snifferprocess-module}}\index{src.SnifferProcess (module)}
Created on May 18, 2017

@author: pschmied
\index{SnifferProcess (class in src.SnifferProcess)}

\begin{fulllineitems}
\phantomsection\label{\detokenize{src:src.SnifferProcess.SnifferProcess}}\pysiglinewithargsret{\sphinxstrong{class }\sphinxcode{src.SnifferProcess.}\sphinxbfcode{SnifferProcess}}{\emph{snifferSendPipe}, \emph{sharedEnabledFlag}, \emph{snifferName}, \emph{CANData=None}}{}
Bases: \sphinxcode{multiprocessing.process.Process}

Spawn a new process that will sniff packets from the specified CANData instance.
Captured data will be transmitted via the \sphinxcode{snifferSendPipe}.
\index{\_\_init\_\_() (src.SnifferProcess.SnifferProcess method)}

\begin{fulllineitems}
\phantomsection\label{\detokenize{src:src.SnifferProcess.SnifferProcess.__init__}}\pysiglinewithargsret{\sphinxbfcode{\_\_init\_\_}}{\emph{snifferSendPipe}, \emph{sharedEnabledFlag}, \emph{snifferName}, \emph{CANData=None}}{}
Set the passed parameters.
\begin{quote}\begin{description}
\item[{Parameters}] \leavevmode\begin{itemize}
\item {} 
\sphinxstyleliteralstrong{snifferSendPipe} \textendash{} The multiprocessing pipe to send received data to

\item {} 
\sphinxstyleliteralstrong{sharedEnabledFlag} \textendash{} The multiprocessing value to handle disabling

\item {} 
\sphinxstyleliteralstrong{snifferName} \textendash{} The name of the sniffer process, used for logging

\item {} 
\sphinxstyleliteralstrong{CANData} \textendash{} Optional: The CANData instance to query for data.
If this is not specified, the global interface is being used

\end{itemize}

\end{description}\end{quote}

\end{fulllineitems}

\index{run() (src.SnifferProcess.SnifferProcess method)}

\begin{fulllineitems}
\phantomsection\label{\detokenize{src:src.SnifferProcess.SnifferProcess.run}}\pysiglinewithargsret{\sphinxbfcode{run}}{}{}
As long as the process hasn’t been disabled: Read a frame using {\hyperref[\detokenize{src:src.CANData.CANData.readPacketAsync}]{\sphinxcrossref{\sphinxcode{readPacketAsync()}}}}
and transmit the received pyvit frame via the pipe.

\end{fulllineitems}


\end{fulllineitems}



\section{CANalyzat0r.SnifferTab module}
\label{\detokenize{src:canalyzat0r-sniffertab-module}}\label{\detokenize{src:module-src.SnifferTab}}\index{src.SnifferTab (module)}
Created on May 18, 2017

@author: pschmied
\index{SnifferTab (class in src.SnifferTab)}

\begin{fulllineitems}
\phantomsection\label{\detokenize{src:src.SnifferTab.SnifferTab}}\pysigline{\sphinxstrong{class }\sphinxcode{src.SnifferTab.}\sphinxbfcode{SnifferTab}}
This class handles the logic of the sniffer tab.
Subtabs are being handled in {\hyperref[\detokenize{src:src.SnifferTabElement.SnifferTabElement}]{\sphinxcrossref{\sphinxcode{SnifferTabElement}}}}.
\index{addSniffer() (src.SnifferTab.SnifferTab static method)}

\begin{fulllineitems}
\phantomsection\label{\detokenize{src:src.SnifferTab.SnifferTab.addSniffer}}\pysiglinewithargsret{\sphinxstrong{static }\sphinxbfcode{addSniffer}}{\emph{snifferTabName}}{}
Appends a new sniffer tab to the sub tab bar.
\begin{quote}\begin{description}
\item[{Parameters}] \leavevmode
\sphinxstyleliteralstrong{snifferTabName} \textendash{} The displayed name of the tab. Normally, this corresponds to the
CAN interface the tab is managing

\end{description}\end{quote}

\end{fulllineitems}

\index{clearAndAddPlaceholder() (src.SnifferTab.SnifferTab static method)}

\begin{fulllineitems}
\phantomsection\label{\detokenize{src:src.SnifferTab.SnifferTab.clearAndAddPlaceholder}}\pysiglinewithargsret{\sphinxstrong{static }\sphinxbfcode{clearAndAddPlaceholder}}{}{}
Add a placeholder where normally sniffer tabs are displayed

\end{fulllineitems}

\index{indexInMainTabBar (src.SnifferTab.SnifferTab attribute)}

\begin{fulllineitems}
\phantomsection\label{\detokenize{src:src.SnifferTab.SnifferTab.indexInMainTabBar}}\pysigline{\sphinxbfcode{indexInMainTabBar}\sphinxstrong{ = 1}}
The index of the sniffer tab in the main tab bar

\end{fulllineitems}

\index{logger (src.SnifferTab.SnifferTab attribute)}

\begin{fulllineitems}
\phantomsection\label{\detokenize{src:src.SnifferTab.SnifferTab.logger}}\pysigline{\sphinxbfcode{logger}\sphinxstrong{ = \textless{}logging.Logger object\textgreater{}}}
The tab specific logger

\end{fulllineitems}

\index{prepareUI() (src.SnifferTab.SnifferTab static method)}

\begin{fulllineitems}
\phantomsection\label{\detokenize{src:src.SnifferTab.SnifferTab.prepareUI}}\pysiglinewithargsret{\sphinxstrong{static }\sphinxbfcode{prepareUI}}{}{}
This adds a placeholder of no instance of {\hyperref[\detokenize{src:src.SnifferTabElement.SnifferTabElement}]{\sphinxcrossref{\sphinxcode{SnifferTabElement}}}}
was created previously.

\end{fulllineitems}

\index{removeSniffer() (src.SnifferTab.SnifferTab static method)}

\begin{fulllineitems}
\phantomsection\label{\detokenize{src:src.SnifferTab.SnifferTab.removeSniffer}}\pysiglinewithargsret{\sphinxstrong{static }\sphinxbfcode{removeSniffer}}{\emph{snifferTabElement=None}, \emph{snifferTabName=None}}{}
Remove a sniffer from the sub tab bar. This method gets called from an instance of
{\hyperref[\detokenize{src:src.SnifferTabElement.SnifferTabElement}]{\sphinxcrossref{\sphinxcode{SnifferTabElement}}}} by \sphinxcode{removeSender()}.
One can either specify \sphinxcode{snifferTabElement} \sphinxstyleemphasis{or} \sphinxcode{snifferTabName} to delete a tab. If both are used,
the object parameter is used.
\begin{quote}\begin{description}
\item[{Parameters}] \leavevmode\begin{itemize}
\item {} 
\sphinxstyleliteralstrong{senderTabElement} \textendash{} Optional: The {\hyperref[\detokenize{src:src.SnifferTabElement.SnifferTabElement}]{\sphinxcrossref{\sphinxcode{SnifferTabElement}}}} instance to remove

\item {} 
\sphinxstyleliteralstrong{snifferTabName} \textendash{} Optional: The name of the {\hyperref[\detokenize{src:src.SenderTabElement.SenderTabElement}]{\sphinxcrossref{\sphinxcode{SenderTabElement}}}} instance to remove

\end{itemize}

\end{description}\end{quote}

\end{fulllineitems}

\index{snifferTabs (src.SnifferTab.SnifferTab attribute)}

\begin{fulllineitems}
\phantomsection\label{\detokenize{src:src.SnifferTab.SnifferTab.snifferTabs}}\pysigline{\sphinxbfcode{snifferTabs}\sphinxstrong{ = \{\}}}
Consinsts of all SnifferTabElements, interface names as key

\end{fulllineitems}

\index{toggleActive() (src.SnifferTab.SnifferTab static method)}

\begin{fulllineitems}
\phantomsection\label{\detokenize{src:src.SnifferTab.SnifferTab.toggleActive}}\pysiglinewithargsret{\sphinxstrong{static }\sphinxbfcode{toggleActive}}{}{}
If there is at least one tab sniffing then the tab bar title will be red.

\end{fulllineitems}

\index{updateInterfaceLabel() (src.SnifferTab.SnifferTab class method)}

\begin{fulllineitems}
\phantomsection\label{\detokenize{src:src.SnifferTab.SnifferTab.updateInterfaceLabel}}\pysiglinewithargsret{\sphinxstrong{classmethod }\sphinxbfcode{updateInterfaceLabel}}{}{}
This invokes \sphinxcode{updateInterfaceLabel()} every sniffer tab

\end{fulllineitems}


\end{fulllineitems}



\section{CANalyzat0r.SnifferTabElement module}
\label{\detokenize{src:module-src.SnifferTabElement}}\label{\detokenize{src:canalyzat0r-sniffertabelement-module}}\index{src.SnifferTabElement (module)}
Created on Jun 16, 2017

@author: pschmied
\index{SnifferTabElement (class in src.SnifferTabElement)}

\begin{fulllineitems}
\phantomsection\label{\detokenize{src:src.SnifferTabElement.SnifferTabElement}}\pysiglinewithargsret{\sphinxstrong{class }\sphinxcode{src.SnifferTabElement.}\sphinxbfcode{SnifferTabElement}}{\emph{tabWidget}, \emph{tabName}, \emph{ifaceName=None}}{}
Bases: {\hyperref[\detokenize{src:src.AbstractTab.AbstractTab}]{\sphinxcrossref{\sphinxcode{src.AbstractTab.AbstractTab}}}}

This class handles the logic of the sniffer sub tab.
The main tab is being handled in {\hyperref[\detokenize{src:src.SnifferTab.SnifferTab}]{\sphinxcrossref{\sphinxcode{SnifferTab}}}}.
\index{\_\_init\_\_() (src.SnifferTabElement.SnifferTabElement method)}

\begin{fulllineitems}
\phantomsection\label{\detokenize{src:src.SnifferTabElement.SnifferTabElement.__init__}}\pysiglinewithargsret{\sphinxbfcode{\_\_init\_\_}}{\emph{tabWidget}, \emph{tabName}, \emph{ifaceName=None}}{}
Set parameters and initialize the CANData instance
\begin{quote}\begin{description}
\item[{Parameters}] \leavevmode\begin{itemize}
\item {} 
\sphinxstyleliteralstrong{tabWidget} \textendash{} The element in the tab bar. \sphinxstylestrong{Not} the table widget.

\item {} 
\sphinxstyleliteralstrong{tabName} \textendash{} The name of the tab

\item {} 
\sphinxstyleliteralstrong{ifaceName} \textendash{} Optional: The interface name. If this is None, then the \sphinxcode{tabName} will be used

\end{itemize}

\end{description}\end{quote}

\end{fulllineitems}

\index{addPacket() (src.SnifferTabElement.SnifferTabElement method)}

\begin{fulllineitems}
\phantomsection\label{\detokenize{src:src.SnifferTabElement.SnifferTabElement.addPacket}}\pysiglinewithargsret{\sphinxbfcode{addPacket}}{\emph{valueList}, \emph{addAtFront=True}, \emph{append=True}, \emph{emit=True}, \emph{addToRawDataOnly=False}, \emph{ignorePacketRate=False}}{}
Override the parents method to add packets at front and to update the counter label.
If too much data is received, the data will be added after sniffing to prevent freezes.
Also, only add packets if the data isn’t present in \sphinxcode{self.ignoredPackets}
\begin{quote}\begin{description}
\item[{Parameters}] \leavevmode
\sphinxstyleliteralstrong{ignorePacketRate} \textendash{} Additional optional parameter: Boolean value indicating whether the rate of
packets per second will be ignored or not. This is set to False by Default.
We need to set it to True to process \sphinxcode{self.valueList} after sniffing if too much
data was received.

\end{description}\end{quote}

\end{fulllineitems}

\index{amountThreadsRunning (src.SnifferTabElement.SnifferTabElement attribute)}

\begin{fulllineitems}
\phantomsection\label{\detokenize{src:src.SnifferTabElement.SnifferTabElement.amountThreadsRunning}}\pysigline{\sphinxbfcode{amountThreadsRunning}\sphinxstrong{ = 0}}
Amount of sniffing threads running to display in the status bar

\end{fulllineitems}

\index{clear() (src.SnifferTabElement.SnifferTabElement method)}

\begin{fulllineitems}
\phantomsection\label{\detokenize{src:src.SnifferTabElement.SnifferTabElement.clear}}\pysiglinewithargsret{\sphinxbfcode{clear}}{\emph{returnOldPackets=False}}{}
Clear the currently displayed data on the GUI and in the lists.
\begin{quote}\begin{description}
\item[{Parameters}] \leavevmode
\sphinxstyleliteralstrong{returnOldPackets} \textendash{} Optional: If this is True then the previously displayed data will be returned as
raw data list. Default is False

\end{description}\end{quote}

\end{fulllineitems}

\index{getPacketCount() (src.SnifferTabElement.SnifferTabElement method)}

\begin{fulllineitems}
\phantomsection\label{\detokenize{src:src.SnifferTabElement.SnifferTabElement.getPacketCount}}\pysiglinewithargsret{\sphinxbfcode{getPacketCount}}{}{}
This uses a call to \sphinxcode{/sys/class/net/\textless{}ifaceName\textgreater{}/statistics/rx\_packets} to return the number
of total received packets of the current interface
\begin{quote}\begin{description}
\item[{Returns}] \leavevmode
Received packet count of the current interface as itneger

\end{description}\end{quote}

\end{fulllineitems}

\index{handleInterfaceSettingsDialog() (src.SnifferTabElement.SnifferTabElement method)}

\begin{fulllineitems}
\phantomsection\label{\detokenize{src:src.SnifferTabElement.SnifferTabElement.handleInterfaceSettingsDialog}}\pysiglinewithargsret{\sphinxbfcode{handleInterfaceSettingsDialog}}{\emph{allowOnlyOwnInterface=True}}{}
Override the parents method to only allow the currently set CAN interface

\end{fulllineitems}

\index{handleManageIgnoredPacketsDialog() (src.SnifferTabElement.SnifferTabElement method)}

\begin{fulllineitems}
\phantomsection\label{\detokenize{src:src.SnifferTabElement.SnifferTabElement.handleManageIgnoredPacketsDialog}}\pysiglinewithargsret{\sphinxbfcode{handleManageIgnoredPacketsDialog}}{}{}
Open a dialog to manage ignored packets when sniffing

\end{fulllineitems}

\index{removeSniffer() (src.SnifferTabElement.SnifferTabElement method)}

\begin{fulllineitems}
\phantomsection\label{\detokenize{src:src.SnifferTabElement.SnifferTabElement.removeSniffer}}\pysiglinewithargsret{\sphinxbfcode{removeSniffer}}{}{}
This gets called when associated interface disappears after a re-check.
This stops the sniffer thread and calls the parents method (\sphinxcode{removeSender()})
to remove the sniffer form the tab bar.

\end{fulllineitems}

\index{tabIndex() (src.SnifferTabElement.SnifferTabElement method)}

\begin{fulllineitems}
\phantomsection\label{\detokenize{src:src.SnifferTabElement.SnifferTabElement.tabIndex}}\pysiglinewithargsret{\sphinxbfcode{tabIndex}}{}{}
Get the \sphinxstylestrong{current} tab index of the sub tab element
\begin{quote}\begin{description}
\item[{Returns}] \leavevmode
The tab index of the sniffer tab

\end{description}\end{quote}

\end{fulllineitems}

\index{terminateThreads() (src.SnifferTabElement.SnifferTabElement method)}

\begin{fulllineitems}
\phantomsection\label{\detokenize{src:src.SnifferTabElement.SnifferTabElement.terminateThreads}}\pysiglinewithargsret{\sphinxbfcode{terminateThreads}}{}{}
This stops the processes/threads called \sphinxcode{snifferProcess} and \sphinxcode{itemAdderThread}.
Also, the CANData instance will be set to inactive and GUI elements will be toggled.

\end{fulllineitems}

\index{toggleActive() (src.SnifferTabElement.SnifferTabElement method)}

\begin{fulllineitems}
\phantomsection\label{\detokenize{src:src.SnifferTabElement.SnifferTabElement.toggleActive}}\pysiglinewithargsret{\sphinxbfcode{toggleActive}}{}{}
Toggles the current sub tab to (in)active. This also calls {\hyperref[\detokenize{src:src.SnifferTab.SnifferTab.toggleActive}]{\sphinxcrossref{\sphinxcode{toggleActive()}}}} to manage
the color of the main tab (parent tab).

\end{fulllineitems}

\index{toggleSniffing() (src.SnifferTabElement.SnifferTabElement method)}

\begin{fulllineitems}
\phantomsection\label{\detokenize{src:src.SnifferTabElement.SnifferTabElement.toggleSniffing}}\pysiglinewithargsret{\sphinxbfcode{toggleSniffing}}{}{}
This starts and stops sniffing for a specific sniffer tab.
Instances of {\hyperref[\detokenize{src:src.ItemAdderThread.ItemAdderThread}]{\sphinxcrossref{\sphinxcode{ItemAdderThread}}}} and {\hyperref[\detokenize{src:src.SnifferProcess.SnifferProcess}]{\sphinxcrossref{\sphinxcode{SnifferProcess}}}} are
used to gather and display data.

\end{fulllineitems}

\index{updateStatusBar() (src.SnifferTabElement.SnifferTabElement method)}

\begin{fulllineitems}
\phantomsection\label{\detokenize{src:src.SnifferTabElement.SnifferTabElement.updateStatusBar}}\pysiglinewithargsret{\sphinxbfcode{updateStatusBar}}{}{}
Updates the status bar label to display the correct amount of sniffer tabs (if any)

\end{fulllineitems}


\end{fulllineitems}



\section{CANalyzat0r.Strings module}
\label{\detokenize{src:module-src.Strings}}\label{\detokenize{src:canalyzat0r-strings-module}}\index{src.Strings (module)}
Created on May 17, 2017

@author: pschmied


\section{CANalyzat0r.Toolbox module}
\label{\detokenize{src:module-src.Toolbox}}\label{\detokenize{src:canalyzat0r-toolbox-module}}\index{src.Toolbox (module)}
Created on May 22, 2017

@author: pschmied
\index{Toolbox (class in src.Toolbox)}

\begin{fulllineitems}
\phantomsection\label{\detokenize{src:src.Toolbox.Toolbox}}\pysigline{\sphinxstrong{class }\sphinxcode{src.Toolbox.}\sphinxbfcode{Toolbox}}
This calls offers helpful static methods that every tab can use to unify program logic and simplify code.
\index{askUserConfirmAction() (src.Toolbox.Toolbox static method)}

\begin{fulllineitems}
\phantomsection\label{\detokenize{src:src.Toolbox.Toolbox.askUserConfirmAction}}\pysiglinewithargsret{\sphinxstrong{static }\sphinxbfcode{askUserConfirmAction}}{}{}
Spawns a MessageBox that asks the user to confirm an action
\begin{quote}\begin{description}
\item[{Returns}] \leavevmode
True if the user has clicked on Yes, else False

\end{description}\end{quote}

\end{fulllineitems}

\index{checkProjectIsNone() (src.Toolbox.Toolbox static method)}

\begin{fulllineitems}
\phantomsection\label{\detokenize{src:src.Toolbox.Toolbox.checkProjectIsNone}}\pysiglinewithargsret{\sphinxstrong{static }\sphinxbfcode{checkProjectIsNone}}{\emph{project=-1}}{}
Checks if a project is \sphinxcode{None} and displays a MessageBox if True.
\begin{quote}\begin{description}
\item[{Parameters}] \leavevmode
\sphinxstyleliteralstrong{project} \textendash{} Optional. The project to check. Default: -1 wich causes the global project to be checked

\item[{Returns}] \leavevmode
Boolean value indicating whether the checked project is \sphinxcode{None}

\end{description}\end{quote}

\end{fulllineitems}

\index{getKnownPacketDescription() (src.Toolbox.Toolbox static method)}

\begin{fulllineitems}
\phantomsection\label{\detokenize{src:src.Toolbox.Toolbox.getKnownPacketDescription}}\pysiglinewithargsret{\sphinxstrong{static }\sphinxbfcode{getKnownPacketDescription}}{\emph{CANID}, \emph{data}}{}
Get a description for a known packet. This will use the dictionary defined in
\sphinxcode{Globals} to find data
\begin{quote}\begin{description}
\item[{Parameters}] \leavevmode\begin{itemize}
\item {} 
\sphinxstyleliteralstrong{CANID} \textendash{} CAN ID

\item {} 
\sphinxstyleliteralstrong{data} \textendash{} Data

\end{itemize}

\item[{Returns}] \leavevmode
The description if one can be found, else an empty string

\end{description}\end{quote}

\end{fulllineitems}

\index{getPacketDictIndex() (src.Toolbox.Toolbox static method)}

\begin{fulllineitems}
\phantomsection\label{\detokenize{src:src.Toolbox.Toolbox.getPacketDictIndex}}\pysiglinewithargsret{\sphinxstrong{static }\sphinxbfcode{getPacketDictIndex}}{\emph{CANID}, \emph{data}}{}
Calculates the index of a packet with a specific CAN ID and data in a dictionary.
\begin{quote}\begin{description}
\item[{Parameters}] \leavevmode\begin{itemize}
\item {} 
\sphinxstyleliteralstrong{CANID} \textendash{} CAN ID

\item {} 
\sphinxstyleliteralstrong{data} \textendash{} Data

\end{itemize}

\item[{Returns}] \leavevmode
The index of a packet in a dictionary

\end{description}\end{quote}

\end{fulllineitems}

\index{getSaveFileName() (src.Toolbox.Toolbox static method)}

\begin{fulllineitems}
\phantomsection\label{\detokenize{src:src.Toolbox.Toolbox.getSaveFileName}}\pysiglinewithargsret{\sphinxstrong{static }\sphinxbfcode{getSaveFileName}}{\emph{dialogTitle}}{}
Spawns a “save file dialog” to get a file path to save to.
\begin{quote}\begin{description}
\item[{Parameters}] \leavevmode
\sphinxstyleliteralstrong{dialogTitle} \textendash{} The displayed dialog title

\item[{Returns}] \leavevmode
The file path of the selected file

\end{description}\end{quote}

\end{fulllineitems}

\index{getWorkingDialog() (src.Toolbox.Toolbox static method)}

\begin{fulllineitems}
\phantomsection\label{\detokenize{src:src.Toolbox.Toolbox.getWorkingDialog}}\pysiglinewithargsret{\sphinxstrong{static }\sphinxbfcode{getWorkingDialog}}{\emph{text}}{}
Generates a working dialog object which blocks the UI.
\begin{quote}\begin{description}
\item[{Parameters}] \leavevmode
\sphinxstyleliteralstrong{text} \textendash{} Text to display while working

\item[{Returns}] \leavevmode
The created working dialog widget

\end{description}\end{quote}

\end{fulllineitems}

\index{interfaceDialogWidget (src.Toolbox.Toolbox attribute)}

\begin{fulllineitems}
\phantomsection\label{\detokenize{src:src.Toolbox.Toolbox.interfaceDialogWidget}}\pysigline{\sphinxbfcode{interfaceDialogWidget}\sphinxstrong{ = None}}
The dialog widget to set the interface settings

\end{fulllineitems}

\index{interfaceSettingsDialog() (src.Toolbox.Toolbox static method)}

\begin{fulllineitems}
\phantomsection\label{\detokenize{src:src.Toolbox.Toolbox.interfaceSettingsDialog}}\pysiglinewithargsret{\sphinxstrong{static }\sphinxbfcode{interfaceSettingsDialog}}{\emph{currentCANData}, \emph{CANDataOverrideValues=None}}{}
Handles the logic of the interface settings dialog.
\begin{quote}\begin{description}
\item[{Parameters}] \leavevmode
\sphinxstyleliteralstrong{CANDataOverrideValues} \textendash{} Optional: List of CANData instances that will be selectable instead of all values.

\item[{Returns}] \leavevmode
A new CANData instances with the selected values. None if no editable CANData instance exists

\end{description}\end{quote}

\end{fulllineitems}

\index{interfaceSettingsDialogCheckBoxChanged() (src.Toolbox.Toolbox static method)}

\begin{fulllineitems}
\phantomsection\label{\detokenize{src:src.Toolbox.Toolbox.interfaceSettingsDialogCheckBoxChanged}}\pysiglinewithargsret{\sphinxstrong{static }\sphinxbfcode{interfaceSettingsDialogCheckBoxChanged}}{\emph{state}}{}
Gets called when the use VCAN CheckBox of the interface dialog gets changed to
handle the enabled state of the bitrate SpinBox
\begin{quote}\begin{description}
\item[{Parameters}] \leavevmode
\sphinxstyleliteralstrong{state} \textendash{} Not used, state is determined by \sphinxcode{isChecked}

\end{description}\end{quote}

\end{fulllineitems}

\index{interfaceSettingsDialogComboBoxChanged() (src.Toolbox.Toolbox static method)}

\begin{fulllineitems}
\phantomsection\label{\detokenize{src:src.Toolbox.Toolbox.interfaceSettingsDialogComboBoxChanged}}\pysiglinewithargsret{\sphinxstrong{static }\sphinxbfcode{interfaceSettingsDialogComboBoxChanged}}{}{}
Gets called when the interface ComboBox of the interface dialog gets changed to
pre-populate the GUI elements accordingly.

\end{fulllineitems}

\index{isHexString() (src.Toolbox.Toolbox static method)}

\begin{fulllineitems}
\phantomsection\label{\detokenize{src:src.Toolbox.Toolbox.isHexString}}\pysiglinewithargsret{\sphinxstrong{static }\sphinxbfcode{isHexString}}{\emph{hexString}}{}
Checks if a hexString is a valid hex string of base 16
\begin{quote}\begin{description}
\item[{Parameters}] \leavevmode
\sphinxstyleliteralstrong{hexString} \textendash{} The hex string

\item[{Returns}] \leavevmode
Boolean value indicating the correctness of the hex string

\end{description}\end{quote}

\end{fulllineitems}

\index{logger (src.Toolbox.Toolbox attribute)}

\begin{fulllineitems}
\phantomsection\label{\detokenize{src:src.Toolbox.Toolbox.logger}}\pysigline{\sphinxbfcode{logger}\sphinxstrong{ = \textless{}logging.Logger object\textgreater{}}}
The toolbox also has its own logger instance

\end{fulllineitems}

\index{mp3Processes (src.Toolbox.Toolbox attribute)}

\begin{fulllineitems}
\phantomsection\label{\detokenize{src:src.Toolbox.Toolbox.mp3Processes}}\pysigline{\sphinxbfcode{mp3Processes}\sphinxstrong{ = \{\}}}
Used to keep track of the processes that play mp3 files.
Key = filepath, value: multiprocessing Process object

\end{fulllineitems}

\index{playMP3() (src.Toolbox.Toolbox static method)}

\begin{fulllineitems}
\phantomsection\label{\detokenize{src:src.Toolbox.Toolbox.playMP3}}\pysiglinewithargsret{\sphinxstrong{static }\sphinxbfcode{playMP3}}{\emph{filePath}}{}
Plays an mp3 sound file using ffmpeg
\begin{quote}\begin{description}
\item[{Parameters}] \leavevmode
\sphinxstyleliteralstrong{filePath} \textendash{} Path of the mp3 file

\end{description}\end{quote}

\end{fulllineitems}

\index{populateInterfaceComboBox() (src.Toolbox.Toolbox static method)}

\begin{fulllineitems}
\phantomsection\label{\detokenize{src:src.Toolbox.Toolbox.populateInterfaceComboBox}}\pysiglinewithargsret{\sphinxstrong{static }\sphinxbfcode{populateInterfaceComboBox}}{\emph{comboBoxWidget}, \emph{reselectCurrentItem=True}, \emph{ignoreActiveInstances=False}}{}
Inserts all available interface values into the passed ComboBox widget
\begin{quote}\begin{description}
\item[{Parameters}] \leavevmode\begin{itemize}
\item {} 
\sphinxstyleliteralstrong{comboBoxWidget} \textendash{} The GUI element to fill with items

\item {} 
\sphinxstyleliteralstrong{reselectCurrentItem} \textendash{} Optional: If this is true, the previously selected index will be re-selected
Default: True

\end{itemize}

\end{description}\end{quote}

\end{fulllineitems}

\index{stopMP3() (src.Toolbox.Toolbox static method)}

\begin{fulllineitems}
\phantomsection\label{\detokenize{src:src.Toolbox.Toolbox.stopMP3}}\pysiglinewithargsret{\sphinxstrong{static }\sphinxbfcode{stopMP3}}{\emph{filePath}}{}
Stops the playback of a given mp3 file
:param filePath: Path of the mp3 file

\end{fulllineitems}

\index{tableExtractAllRowData() (src.Toolbox.Toolbox static method)}

\begin{fulllineitems}
\phantomsection\label{\detokenize{src:src.Toolbox.Toolbox.tableExtractAllRowData}}\pysiglinewithargsret{\sphinxstrong{static }\sphinxbfcode{tableExtractAllRowData}}{\emph{table}}{}
Get \sphinxstylestrong{all} contents of a GUI table
\begin{quote}\begin{description}
\item[{Parameters}] \leavevmode
\sphinxstyleliteralstrong{table} \textendash{} The \sphinxcode{QTableView} object to gather data from

\item[{Returns}] \leavevmode
A list of raw row data \textendash{}\textgreater{} List of lists

\end{description}\end{quote}

\end{fulllineitems}

\index{tableExtractSelectedRowData() (src.Toolbox.Toolbox static method)}

\begin{fulllineitems}
\phantomsection\label{\detokenize{src:src.Toolbox.Toolbox.tableExtractSelectedRowData}}\pysiglinewithargsret{\sphinxstrong{static }\sphinxbfcode{tableExtractSelectedRowData}}{\emph{table}}{}
Get the \sphinxstylestrong{selected} contents of a GUI table
\begin{quote}\begin{description}
\item[{Parameters}] \leavevmode
\sphinxstyleliteralstrong{table} \textendash{} The \sphinxcode{QTableView} object to gather data from

\item[{Returns}] \leavevmode
A list of raw row data \textendash{}\textgreater{} List of lists

\end{description}\end{quote}

\end{fulllineitems}

\index{toggleDisabledProjectGUIElements() (src.Toolbox.Toolbox static method)}

\begin{fulllineitems}
\phantomsection\label{\detokenize{src:src.Toolbox.Toolbox.toggleDisabledProjectGUIElements}}\pysiglinewithargsret{\sphinxstrong{static }\sphinxbfcode{toggleDisabledProjectGUIElements}}{}{}
This toggles specific GUI elements that should only be active if a project has been selected

\end{fulllineitems}

\index{toggleDisabledSenderGUIElements() (src.Toolbox.Toolbox static method)}

\begin{fulllineitems}
\phantomsection\label{\detokenize{src:src.Toolbox.Toolbox.toggleDisabledSenderGUIElements}}\pysiglinewithargsret{\sphinxstrong{static }\sphinxbfcode{toggleDisabledSenderGUIElements}}{}{}
This toggles specific GUI elements that should only be active if a CANData instance is present

\end{fulllineitems}

\index{updateCANDataInstances() (src.Toolbox.Toolbox static method)}

\begin{fulllineitems}
\phantomsection\label{\detokenize{src:src.Toolbox.Toolbox.updateCANDataInstances}}\pysiglinewithargsret{\sphinxstrong{static }\sphinxbfcode{updateCANDataInstances}}{\emph{CANDataInstance}}{}
Calls \sphinxcode{updateCANDataInstance} for every tab.
\begin{quote}\begin{description}
\item[{Parameters}] \leavevmode
\sphinxstyleliteralstrong{CANDataInstance} \textendash{} The new CANData instance

\end{description}\end{quote}

\end{fulllineitems}

\index{updateInterfaceLabels() (src.Toolbox.Toolbox static method)}

\begin{fulllineitems}
\phantomsection\label{\detokenize{src:src.Toolbox.Toolbox.updateInterfaceLabels}}\pysiglinewithargsret{\sphinxstrong{static }\sphinxbfcode{updateInterfaceLabels}}{}{}
Calls \sphinxcode{updateInterfaceLabel} for every tab.

\end{fulllineitems}

\index{widgetFromUIFile() (src.Toolbox.Toolbox static method)}

\begin{fulllineitems}
\phantomsection\label{\detokenize{src:src.Toolbox.Toolbox.widgetFromUIFile}}\pysiglinewithargsret{\sphinxstrong{static }\sphinxbfcode{widgetFromUIFile}}{\emph{filePath}}{}
Reads an \sphinxcode{.ui} file and creates a new widget object from it.
\begin{quote}\begin{description}
\item[{Parameters}] \leavevmode
\sphinxstyleliteralstrong{filePath} \textendash{} Where to find the \sphinxcode{.ui} file

\item[{Returns}] \leavevmode
The new created widget

\end{description}\end{quote}

\end{fulllineitems}

\index{yesNoBox() (src.Toolbox.Toolbox static method)}

\begin{fulllineitems}
\phantomsection\label{\detokenize{src:src.Toolbox.Toolbox.yesNoBox}}\pysiglinewithargsret{\sphinxstrong{static }\sphinxbfcode{yesNoBox}}{\emph{title}, \emph{text}}{}
Spawns a MessageBox that asks the user a Yes-No question.
\begin{quote}\begin{description}
\item[{Returns}] \leavevmode
True if the user has clicked on Yes, else False

\end{description}\end{quote}

\end{fulllineitems}


\end{fulllineitems}



\section{CANalyzat0r.mainWindow module}
\label{\detokenize{src:module-src.mainWindow}}\label{\detokenize{src:canalyzat0r-mainwindow-module}}\index{src.mainWindow (module)}

\section{CANalyzat0r.AbstractTab module}
\label{\detokenize{src:module-src.AbstractTab}}\label{\detokenize{src:canalyzat0r-abstracttab-module}}\index{src.AbstractTab (module)}
Created on Jun 26, 2017

@author: pschmied
\index{AbstractTab (class in src.AbstractTab)}

\begin{fulllineitems}
\phantomsection\label{\detokenize{src:src.AbstractTab.AbstractTab}}\pysiglinewithargsret{\sphinxstrong{class }\sphinxcode{src.AbstractTab.}\sphinxbfcode{AbstractTab}}{\emph{tabWidget}, \emph{loggerName}, \emph{readOnlyCols}, \emph{packetTableViewName}, \emph{labelInterfaceValueName=None}, \emph{CANData=None}, \emph{hideTimestampCol=True}, \emph{sendToSenderContextMenu=True}, \emph{saveAsPacketSetContextMenu=True}, \emph{allowTableCopy=True}, \emph{allowTablePaste=True}, \emph{allowTableDelete=True}}{}
This is a base class for \sphinxstyleemphasis{most} tabs. If you’re using a tab that uses the following things, you can use this class:
- Non-static tab - you create instances from it
- a QTableView to display data
- \sphinxcode{rawData} as raw packet list
- Copy, paste and/or delete actions by shortcuts and context menus

Just take care of \sphinxcode{packetTableViewName} and \sphinxcode{labelInterfaceValueName} as they’re necessary for this to work.
\index{CANData (src.AbstractTab.AbstractTab attribute)}

\begin{fulllineitems}
\phantomsection\label{\detokenize{src:src.AbstractTab.AbstractTab.CANData}}\pysigline{\sphinxbfcode{CANData}\sphinxstrong{ = None}}
The tab specific CANData instance

\end{fulllineitems}

\index{\_\_init\_\_() (src.AbstractTab.AbstractTab method)}

\begin{fulllineitems}
\phantomsection\label{\detokenize{src:src.AbstractTab.AbstractTab.__init__}}\pysiglinewithargsret{\sphinxbfcode{\_\_init\_\_}}{\emph{tabWidget}, \emph{loggerName}, \emph{readOnlyCols}, \emph{packetTableViewName}, \emph{labelInterfaceValueName=None}, \emph{CANData=None}, \emph{hideTimestampCol=True}, \emph{sendToSenderContextMenu=True}, \emph{saveAsPacketSetContextMenu=True}, \emph{allowTableCopy=True}, \emph{allowTablePaste=True}, \emph{allowTableDelete=True}}{}
\end{fulllineitems}

\index{active (src.AbstractTab.AbstractTab attribute)}

\begin{fulllineitems}
\phantomsection\label{\detokenize{src:src.AbstractTab.AbstractTab.active}}\pysigline{\sphinxbfcode{active}\sphinxstrong{ = None}}
Whether the tab is currently active (using \sphinxcode{CANData})

\end{fulllineitems}

\index{addPacket() (src.AbstractTab.AbstractTab method)}

\begin{fulllineitems}
\phantomsection\label{\detokenize{src:src.AbstractTab.AbstractTab.addPacket}}\pysiglinewithargsret{\sphinxbfcode{addPacket}}{\emph{valueList}, \emph{addAtFront=False}, \emph{append=True}, \emph{emit=True}, \emph{addToRawDataOnly=False}}{}
Add a packet to the GUI table.
\begin{quote}\begin{description}
\item[{Parameters}] \leavevmode\begin{itemize}
\item {} 
\sphinxstyleliteralstrong{valueList} \textendash{} Packet data as raw value list

\item {} 
\sphinxstyleliteralstrong{addAtFront} \textendash{} Optional. Indicates whether the packets will be inserted at the start of \sphinxcode{rawData}.
Default: False

\item {} 
\sphinxstyleliteralstrong{append} \textendash{} Optional. Indicates whether data will be added to \sphinxcode{self.rawData} or not. Default: True

\item {} 
\sphinxstyleliteralstrong{emit} \textendash{} Optional. Indicates whether the GUI will be notified using signals. Default: True

\item {} 
\sphinxstyleliteralstrong{addToRawDataOnly} \textendash{} Optional. Indicates whether only \sphinxcode{self.rawData} will be updated, ignoring
the packet table model. Default: False

\end{itemize}

\end{description}\end{quote}

\end{fulllineitems}

\index{applyNewKnownPackets() (src.AbstractTab.AbstractTab method)}

\begin{fulllineitems}
\phantomsection\label{\detokenize{src:src.AbstractTab.AbstractTab.applyNewKnownPackets}}\pysiglinewithargsret{\sphinxbfcode{applyNewKnownPackets}}{}{}
Apply new known packets which have been saved in the mean time. This reloads the packets into the GUI table.

\end{fulllineitems}

\index{clear() (src.AbstractTab.AbstractTab method)}

\begin{fulllineitems}
\phantomsection\label{\detokenize{src:src.AbstractTab.AbstractTab.clear}}\pysiglinewithargsret{\sphinxbfcode{clear}}{\emph{returnOldPackets=False}}{}
Clear the currently displayed data on the GUI and in the rawData list
\begin{quote}\begin{description}
\item[{Parameters}] \leavevmode
\sphinxstyleliteralstrong{returnOldPackets} \textendash{} If this is true, then the previously displayed packets will
be returned as raw data list

\item[{Returns}] \leavevmode
Previously displayed packets as raw data list (if returnOldPackets is True), else an empty list

\end{description}\end{quote}

\end{fulllineitems}

\index{handleCellChanged() (src.AbstractTab.AbstractTab method)}

\begin{fulllineitems}
\phantomsection\label{\detokenize{src:src.AbstractTab.AbstractTab.handleCellChanged}}\pysiglinewithargsret{\sphinxbfcode{handleCellChanged}}{\emph{rowIndex}, \emph{colIndex}}{}
To update the rawData element and
to put the length of the changed data field into the length field.
\begin{quote}\begin{description}
\item[{Parameters}] \leavevmode\begin{itemize}
\item {} 
\sphinxstyleliteralstrong{rowIndex} \textendash{} The changed row

\item {} 
\sphinxstyleliteralstrong{colIndex} \textendash{} The changed column

\end{itemize}

\item[{Returns}] \leavevmode


\end{description}\end{quote}

\end{fulllineitems}

\index{handleCopy() (src.AbstractTab.AbstractTab method)}

\begin{fulllineitems}
\phantomsection\label{\detokenize{src:src.AbstractTab.AbstractTab.handleCopy}}\pysiglinewithargsret{\sphinxbfcode{handleCopy}}{}{}
Handle copying \sphinxstylestrong{selected} rows from a GUI table.
This copies the raw data list \sphinxstylestrong{to the clipboard}.

\end{fulllineitems}

\index{handleInterfaceSettingsDialog() (src.AbstractTab.AbstractTab method)}

\begin{fulllineitems}
\phantomsection\label{\detokenize{src:src.AbstractTab.AbstractTab.handleInterfaceSettingsDialog}}\pysiglinewithargsret{\sphinxbfcode{handleInterfaceSettingsDialog}}{\emph{allowOnlyOwnInterface=False}}{}
Open a dialog to change interface settings and set the updated CANData instance.
\begin{quote}\begin{description}
\item[{Parameters}] \leavevmode
\sphinxstyleliteralstrong{allowOnlyOwnInterface} \textendash{} If this is true, you can only edit the CANData instance that is already selected
for the current tab. This is being used for the sniffer tabs.

\end{description}\end{quote}

\end{fulllineitems}

\index{handlePaste() (src.AbstractTab.AbstractTab method)}

\begin{fulllineitems}
\phantomsection\label{\detokenize{src:src.AbstractTab.AbstractTab.handlePaste}}\pysiglinewithargsret{\sphinxbfcode{handlePaste}}{}{}
Handle pasting rows into a GUI table.
Data is being gathered from the clipboard and be of the following types:
\begin{itemize}
\item {} 
Raw data list (list of lists which consist of column data) - Parsing takes place asynchronously

\item {} 
SocketCAN format (see \sphinxcode{SocketCANFormat})

\end{itemize}

\end{fulllineitems}

\index{handleRightCick() (src.AbstractTab.AbstractTab method)}

\begin{fulllineitems}
\phantomsection\label{\detokenize{src:src.AbstractTab.AbstractTab.handleRightCick}}\pysiglinewithargsret{\sphinxbfcode{handleRightCick}}{}{}
This spawns a custom context menu right next to the cursor if a user has right clicked on a table cell.

\end{fulllineitems}

\index{loggerName (src.AbstractTab.AbstractTab attribute)}

\begin{fulllineitems}
\phantomsection\label{\detokenize{src:src.AbstractTab.AbstractTab.loggerName}}\pysigline{\sphinxbfcode{loggerName}\sphinxstrong{ = None}}
The tab specific logger

\end{fulllineitems}

\index{manualAddPacket() (src.AbstractTab.AbstractTab method)}

\begin{fulllineitems}
\phantomsection\label{\detokenize{src:src.AbstractTab.AbstractTab.manualAddPacket}}\pysiglinewithargsret{\sphinxbfcode{manualAddPacket}}{}{}
Manually add an empty packet row to the GUI table. This also updates \sphinxcode{rawData}.

\end{fulllineitems}

\index{packetTableModel (src.AbstractTab.AbstractTab attribute)}

\begin{fulllineitems}
\phantomsection\label{\detokenize{src:src.AbstractTab.AbstractTab.packetTableModel}}\pysigline{\sphinxbfcode{packetTableModel}\sphinxstrong{ = None}}
Custom packet model of the GUI table

\end{fulllineitems}

\index{prepareUI() (src.AbstractTab.AbstractTab method)}

\begin{fulllineitems}
\phantomsection\label{\detokenize{src:src.AbstractTab.AbstractTab.prepareUI}}\pysiglinewithargsret{\sphinxbfcode{prepareUI}}{}{}
Prepare the tab specific GUI elements, add keyboard shortcuts and set a CANData instance

\end{fulllineitems}

\index{rawData (src.AbstractTab.AbstractTab attribute)}

\begin{fulllineitems}
\phantomsection\label{\detokenize{src:src.AbstractTab.AbstractTab.rawData}}\pysigline{\sphinxbfcode{rawData}\sphinxstrong{ = None}}
Raw packet data that corresponds to the data displayed in the GUI table

\end{fulllineitems}

\index{readOnlyCols (src.AbstractTab.AbstractTab attribute)}

\begin{fulllineitems}
\phantomsection\label{\detokenize{src:src.AbstractTab.AbstractTab.readOnlyCols}}\pysigline{\sphinxbfcode{readOnlyCols}\sphinxstrong{ = None}}
Tab specific read only columns as list of indexes

\end{fulllineitems}

\index{removeSelectedPackets() (src.AbstractTab.AbstractTab method)}

\begin{fulllineitems}
\phantomsection\label{\detokenize{src:src.AbstractTab.AbstractTab.removeSelectedPackets}}\pysiglinewithargsret{\sphinxbfcode{removeSelectedPackets}}{}{}
Remove selected rows from a table and also delete those rows from a \sphinxcode{rawData} list.
:return: A list of indexes of the removed rows. None if no rows have been selected

\end{fulllineitems}

\index{setInitialCANData() (src.AbstractTab.AbstractTab method)}

\begin{fulllineitems}
\phantomsection\label{\detokenize{src:src.AbstractTab.AbstractTab.setInitialCANData}}\pysiglinewithargsret{\sphinxbfcode{setInitialCANData}}{}{}
Try to get initial an initial CANData instance.
This method also updates GUI elements.
\begin{quote}\begin{description}
\item[{Returns}] \leavevmode
A boolean value which indicates the success

\end{description}\end{quote}

\end{fulllineitems}

\index{tabWidget (src.AbstractTab.AbstractTab attribute)}

\begin{fulllineitems}
\phantomsection\label{\detokenize{src:src.AbstractTab.AbstractTab.tabWidget}}\pysigline{\sphinxbfcode{tabWidget}\sphinxstrong{ = None}}
The specific GUI tab

\end{fulllineitems}

\index{toggleGUIElements() (src.AbstractTab.AbstractTab method)}

\begin{fulllineitems}
\phantomsection\label{\detokenize{src:src.AbstractTab.AbstractTab.toggleGUIElements}}\pysiglinewithargsret{\sphinxbfcode{toggleGUIElements}}{\emph{state}}{}
\end{fulllineitems}

\index{updateCANDataInstance() (src.AbstractTab.AbstractTab method)}

\begin{fulllineitems}
\phantomsection\label{\detokenize{src:src.AbstractTab.AbstractTab.updateCANDataInstance}}\pysiglinewithargsret{\sphinxbfcode{updateCANDataInstance}}{\emph{CANDataInstance}}{}
Updates the tab specific CANData instance to the passed parameter.
This only takes place if the tab is not active to prevent errors.
This also calls {\hyperref[\detokenize{src:src.AbstractTab.AbstractTab.updateInterfaceLabel}]{\sphinxcrossref{\sphinxcode{updateInterfaceLabel()}}}} to update the label.
\begin{quote}\begin{description}
\item[{Parameters}] \leavevmode
\sphinxstyleliteralstrong{CANDataInstance} \textendash{} The new CANData instance

\end{description}\end{quote}

\end{fulllineitems}

\index{updateInterfaceLabel() (src.AbstractTab.AbstractTab method)}

\begin{fulllineitems}
\phantomsection\label{\detokenize{src:src.AbstractTab.AbstractTab.updateInterfaceLabel}}\pysiglinewithargsret{\sphinxbfcode{updateInterfaceLabel}}{}{}
Set the text of the interface label to the updated value, if the label is present.
Uses the text “None” if no interface is set.

\end{fulllineitems}


\end{fulllineitems}



\section{Module contents}
\label{\detokenize{src:module-src}}\label{\detokenize{src:module-contents}}\index{src (module)}

\chapter{Used libraries and files}
\label{\detokenize{usedlibs::doc}}\label{\detokenize{usedlibs:used-libraries-and-files}}

\section{Libs}
\label{\detokenize{usedlibs:libs}}\begin{itemize}
\item {} 
\sphinxhref{https://github.com/linklayer/pyvit}{pyvit by Eric Evenchick}

\item {} 
\sphinxhref{https://wiki.qt.io/PySide}{PySide: Python for Qt}

\item {} 
{\color{red}\bfseries{}{}`ffmpeg \textless{}https://ffmpeg.org/{}`\_}

\item {} 
\sphinxhref{https://github.com/linux-can/can-utils}{can-utils}

\item {} 
\sphinxhref{http://www.sphinx-doc.org/en/stable/}{Sphinx}

\item {} 
\sphinxhref{https://github.com/rtfd/sphinx\_rtd\_theme}{Sphinx RTD Theme}

\end{itemize}


\section{Misc}
\label{\detokenize{usedlibs:misc}}\begin{itemize}
\item {} 
\sphinxhref{https://freeiconshop.com/icon/car-icon-flat/}{Car Icon Flat}

\item {} 
\sphinxhref{http://www.iconsdb.com/orange-icons/flame-2-icon.html}{Orange flame 2 icon}

\item {} 
\sphinxhref{http://www.iconsdb.com/soylent-red-icons/flame-2-icon.html}{Soylent red flame 2 icon}

\end{itemize}


\chapter{Indices and tables}
\label{\detokenize{index:indices-and-tables}}\begin{itemize}
\item {} 
\DUrole{xref,std,std-ref}{genindex}

\item {} 
\DUrole{xref,std,std-ref}{modindex}

\item {} 
\DUrole{xref,std,std-ref}{search}

\end{itemize}


\renewcommand{\indexname}{Python Module Index}
\begin{sphinxtheindex}
\def\bigletter#1{{\Large\sffamily#1}\nopagebreak\vspace{1mm}}
\bigletter{s}
\item {\sphinxstyleindexentry{src}}\sphinxstyleindexpageref{src:\detokenize{module-src}}
\item {\sphinxstyleindexentry{src.AboutTab}}\sphinxstyleindexpageref{src:\detokenize{module-src.AboutTab}}
\item {\sphinxstyleindexentry{src.AbstractTab}}\sphinxstyleindexpageref{src:\detokenize{module-src.AbstractTab}}
\item {\sphinxstyleindexentry{src.CANalyzat0r}}\sphinxstyleindexpageref{src:\detokenize{module-src.CANalyzat0r}}
\item {\sphinxstyleindexentry{src.CANData}}\sphinxstyleindexpageref{src:\detokenize{module-src.CANData}}
\item {\sphinxstyleindexentry{src.ComparerTab}}\sphinxstyleindexpageref{src:\detokenize{module-src.ComparerTab}}
\item {\sphinxstyleindexentry{src.Database}}\sphinxstyleindexpageref{src:\detokenize{module-src.Database}}
\item {\sphinxstyleindexentry{src.FilterTab}}\sphinxstyleindexpageref{src:\detokenize{module-src.FilterTab}}
\item {\sphinxstyleindexentry{src.FuzzerTab}}\sphinxstyleindexpageref{src:\detokenize{module-src.FuzzerTab}}
\item {\sphinxstyleindexentry{src.Globals}}\sphinxstyleindexpageref{src:\detokenize{module-src.Globals}}
\item {\sphinxstyleindexentry{src.ItemAdderThread}}\sphinxstyleindexpageref{src:\detokenize{module-src.ItemAdderThread}}
\item {\sphinxstyleindexentry{src.KnownPacket}}\sphinxstyleindexpageref{src:\detokenize{module-src.KnownPacket}}
\item {\sphinxstyleindexentry{src.Logger}}\sphinxstyleindexpageref{src:\detokenize{module-src.Logger}}
\item {\sphinxstyleindexentry{src.MainTab}}\sphinxstyleindexpageref{src:\detokenize{module-src.MainTab}}
\item {\sphinxstyleindexentry{src.mainWindow}}\sphinxstyleindexpageref{src:\detokenize{module-src.mainWindow}}
\item {\sphinxstyleindexentry{src.ManagerTab}}\sphinxstyleindexpageref{src:\detokenize{module-src.ManagerTab}}
\item {\sphinxstyleindexentry{src.Packet}}\sphinxstyleindexpageref{src:\detokenize{module-src.Packet}}
\item {\sphinxstyleindexentry{src.PacketsDialog}}\sphinxstyleindexpageref{src:\detokenize{module-src.PacketsDialog}}
\item {\sphinxstyleindexentry{src.PacketSet}}\sphinxstyleindexpageref{src:\detokenize{module-src.PacketSet}}
\item {\sphinxstyleindexentry{src.PacketTableModel}}\sphinxstyleindexpageref{src:\detokenize{module-src.PacketTableModel}}
\item {\sphinxstyleindexentry{src.Project}}\sphinxstyleindexpageref{src:\detokenize{module-src.Project}}
\item {\sphinxstyleindexentry{src.SearcherTab}}\sphinxstyleindexpageref{src:\detokenize{module-src.SearcherTab}}
\item {\sphinxstyleindexentry{src.SenderTab}}\sphinxstyleindexpageref{src:\detokenize{module-src.SenderTab}}
\item {\sphinxstyleindexentry{src.SenderTabElement}}\sphinxstyleindexpageref{src:\detokenize{module-src.SenderTabElement}}
\item {\sphinxstyleindexentry{src.SenderThread}}\sphinxstyleindexpageref{src:\detokenize{module-src.SenderThread}}
\item {\sphinxstyleindexentry{src.Settings}}\sphinxstyleindexpageref{src:\detokenize{module-src.Settings}}
\item {\sphinxstyleindexentry{src.SnifferProcess}}\sphinxstyleindexpageref{src:\detokenize{module-src.SnifferProcess}}
\item {\sphinxstyleindexentry{src.SnifferTab}}\sphinxstyleindexpageref{src:\detokenize{module-src.SnifferTab}}
\item {\sphinxstyleindexentry{src.SnifferTabElement}}\sphinxstyleindexpageref{src:\detokenize{module-src.SnifferTabElement}}
\item {\sphinxstyleindexentry{src.Strings}}\sphinxstyleindexpageref{src:\detokenize{module-src.Strings}}
\item {\sphinxstyleindexentry{src.Toolbox}}\sphinxstyleindexpageref{src:\detokenize{module-src.Toolbox}}
\item {\sphinxstyleindexentry{src.ui}}\sphinxstyleindexpageref{src.ui:\detokenize{module-src.ui}}
\item {\sphinxstyleindexentry{src.ui.mainWindow}}\sphinxstyleindexpageref{src.ui:\detokenize{module-src.ui.mainWindow}}
\end{sphinxtheindex}

\renewcommand{\indexname}{Index}
\printindex
\end{document}